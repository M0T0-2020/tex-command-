\documentclass{jsarticle}
\usepackage{amsthm}
\usepackage{amsmath}
\usepackage{amssymb}
\usepackage{eucal}
\usepackage{amsfonts}
\usepackage{mathrsfs}  


\newtheorem{thm}{定理}
\newtheorem{dfn}[thm]{定義}
\newtheorem{prop}[thm]{命題}
\newtheorem{lem}[thm]{補題}
\newtheorem{cor}[thm]{系}


\begin{document}


\title{課題研究bレポート}
\author{加納基晴}
\date{}
\maketitle

%ボレルカンテリの補題
%\begin{thm}

%Borel-Ccantelli Lamma 

%\end{thm}

%\begin{proof}

%\end{proof}


%近似定理
\begin{thm}
近似定理 \par
$(\Omega, \mathscr{F}, P)$を確率空間, \ $X_{1}, X_{2}, \dots$ を確率変数列とする.\par
${}^\forall A_{1} \in \sigma (\textbf{X}),{}^\forall \varepsilon > 0$ に対して, ある$n \in \mathbb{N}, A_{2} \in \mathcal{F} (X_{1}, X_{2}, \dots , X{n})$が存在して$P(A_{1} \triangle A_{2}) \le 0$ となる. \par (ただし $A \triangle B := (A-B) \cup (B-A)$ )
\end{thm}

\begin{proof}
$\quad$\par
${}^\forall A_{1} \in \sigma (\textbf{X}), {}^\forall \varepsilon > 0$を固定する. \par
$\mathscr{F}_{0} = \displaystyle\bigcup_{n \in  \mathbb{N}} \sigma (X_{1}, X_{2}, \dots, X_{n}) $, $\mathcal{C} = {\lbrace A \in \mathcal{F} | \ {}^\forall \varepsilon > 0 に対して, {}^\exists B \in \mathscr{F}_{0} \ s.t. \ P(A \triangle B) \le \varepsilon \rbrace  }$と定める. \par
$\mathscr{F}_{0}  \subset \mathcal{C}$は明らかだから,$\mathcal{C}$が$\sigma$加法族であることを示せば, $\sigma (\mathscr{F}_{0}) \subset \mathcal{C}$で, $\sigma (\mathscr{F}_{0}) = \sigma (\textbf {X})$であることから,\par
$A_{1} \in \sigma (\textbf{X}) \subset \mathcal{C}$なので,${}^\exists A_{2} \in \mathscr{F}_{0} \ s.t. \ P(A_{1} \triangle A_{2}) \le \varepsilon$となり、定理が成立するのがわかる.\par
%%証明
$\bullet \ \mathcal{C}$が$\sigma$加法族であること示す.\par
$\quad$ $(i)$ $\Omega \in \mathcal{C} \ (\because \ \Omega \in \ \mathscr{F}_{0}) $\par

$\quad$ $(ii)$ ${}^\forall A \in \mathcal{C}に対して, A^c \in \mathcal{C}$\par
$\qquad$ $\because {}^\forall \varepsilon > 0$ を固定する.このとき$B \in \mathcal{F}_{0}$が取れて, $P(A \triangle B) \le \varepsilon $となる.$\mathcal{F}_{0}$の定め方から, \ $B^c \in \mathcal{F}_{0}$ \par $\quad \qquad$であって, $P(A^c \triangle B^c) = P((A^c \cap B) \cup (A \cap B^c)) = P((B-A) \cup (A-B)) = P(A \triangle B) \le \varepsilon$\par $\quad \qquad$ $\therefore A^c \in \mathcal{C}$

$\quad$ $(iii)$ ${}^\forall \lbrace A_{n} \rbrace_{n \in \mathbb{N}} \subset \mathcal{C}, {}^\forall \varepsilon > 0$をとる. $\lbrace B_{n} \rbrace_{n \in \mathbb{N}} \subset \mathcal{F}_{0}$を$P(A_{n} \triangle B_{n}) \le \frac{\varepsilon}{2^{n+1}} $となるようにとる.\par 
$\quad \qquad$また,測度の上からの連続性から ある$N \in \mathbb{N}$が取れて, $P( \displaystyle\bigcup_{n=N+1}^{\infty} A_{n}) \le \frac{\varepsilon}{2}$となる.\par$\quad \qquad$ ここで, $\displaystyle\bigcup_{n=1}^{N} B_{n} \triangle \displaystyle\bigcup_{n=1}^{\infty} A_{n} \subset (\displaystyle\bigcup_{n=1}^{N} B_{n} \triangle A_{n}) \cup \displaystyle\bigcup_{n=N+1}^{\infty} A_{n}$を示せれば,単調性と劣加法性から,\par $\quad \qquad$ $P(\displaystyle\bigcup_{n=1}^{N} B_{n} \triangle \displaystyle\bigcup_{n=1}^{\infty} A_{n}) \le P((\displaystyle\bigcup_{n=1}^{N} B_{n} \triangle A_{n}) \cup \displaystyle\bigcup_{n=N+1}^{\infty} A_{n}) \le  \displaystyle\sum_{n=1}^{N}P(B_{n} \triangle A_{n}) + P(\displaystyle\bigcup_{n=N+1}^{\infty} A_{n})$ \par $\quad \qquad$ $\le \frac{\varepsilon}{2} + \frac{\varepsilon}{2} = \varepsilon $となる. $\displaystyle\bigcup_{n=1}^{N} B_{n} \in \mathcal{F}_{0}$であることから $\displaystyle\bigcup_{n=1}^{\infty} A_{n} \in \mathcal{C}$となる.\par

%%証明の中の証明
$\quad \qquad$ $\bullet \ \displaystyle\bigcup_{n=1}^{N} B_{n} \triangle \displaystyle\bigcup_{n=1}^{\infty} A_{n} \subset (\displaystyle\bigcup_{n=1}^{N} B_{n} \triangle A_{n}) \cup \displaystyle\bigcup_{n=N+1}^{\infty} A_{n}$を示す. \par
$\quad \qquad$ $\because \ \omega \in \displaystyle\bigcup_{n=1}^{N} B_{n} \triangle \displaystyle\bigcup_{n=1}^{\infty} A_{n} \Leftrightarrow (\omega \in \displaystyle\bigcup_{n=1}^{N} B_{n} - \displaystyle\bigcup_{n=1}^{\infty} A_{n}) \lor  (\omega \in \displaystyle\bigcup_{n=1}^{\infty} A_{n} - \displaystyle\bigcup_{n=1}^{N} B_{n})$ \par 
$\qquad \qquad \qquad$
$ \Leftrightarrow (\omega \in \displaystyle\bigcup_{n=1}^{N} B_{n} \cap \displaystyle\bigcap_{n=1}^{\infty} A_{n}^{c}) \lor 
 (\omega \in ( \displaystyle\bigcup_{n=1}^{N} A_{n} \cap \displaystyle\bigcap_{n=1}^{N} B_{n}^{c}) \cup (\displaystyle\bigcup_{n=N+1}^{\infty} A_{n} \cap \displaystyle\bigcap_{n=1}^{N} B_{n}^{c}) ) $\par 
$\qquad \qquad \qquad$
$\Rightarrow (\omega \in \displaystyle\bigcup_{n=1}^{N} B_{n} \cap \displaystyle\bigcap_{n=1}^{N} A_{n}^{c}) \lor (\omega \in ( \displaystyle\bigcup_{n=1}^{N} A_{n} \cap \displaystyle\bigcap_{n=1}^{N} B_{n}^{c}) \cup \displaystyle\bigcup_{n=N+1}^{\infty} A_{n} )$\par 
$\qquad \qquad \qquad$
$\Rightarrow \omega \in \displaystyle\bigcup_{n=1}^{N} (B_{n} \cap  A_{n}^{c}) \lor (\omega \in ( \displaystyle\bigcup_{n=1}^{N} A_{n} \cap B_{n}^{c}) \cup \displaystyle\bigcup_{n=N+1}^{\infty} A_{n} )$
\par 
$\qquad \qquad \qquad$
$\Leftrightarrow \omega \in \displaystyle\bigcup_{n=1}^{N} ( (B_{n} \cap  A_{n}^{c}) \cup (A_{n} \cap  B_{n}^{c}) ) \lor \omega \in \displaystyle\bigcup_{n=N+1}^{\infty} A_{n}$
\par 
$\qquad \qquad \qquad$
$\Leftrightarrow  \omega \in \displaystyle\bigcup_{n=1}^{N} ( B_{n} \triangle A_{n}) \lor \omega \in \displaystyle\bigcup_{n=N+1}^{\infty} A_{n}$
\par 
$\qquad \qquad \qquad$
$\Leftrightarrow \omega \in \displaystyle\bigcup_{n=1}^{N} ( B_{n} \triangle A_{n}) \cup \displaystyle\bigcup_{n=N+1}^{\infty} A_{n}$
$\qquad \quad$ $\therefore \displaystyle\bigcup_{n=1}^{N} B_{n} \triangle \displaystyle\bigcup_{n=1}^{\infty} A_{n} \subset (\displaystyle\bigcup_{n=1}^{N} B_{n} \triangle A_{n}) \cup \displaystyle\bigcup_{n=N+1}^{\infty} A_{n}$ \par
$\quad$$(i) \sim (iii)$より$\mathcal{C}$は$\sigma$加法族である.



\end{proof}

%ゼロワン法則
\begin{thm}
Kolmogorov zero-one law
\end{thm}

$X_{1},X_{2}, \dots を独立な確率変数とする.この時, E\in\delta であるとすればP(E)は0,1のいずれかの値をとる.$

\begin{proof}
$\\ {}^\forall E \in \delta をとる. \  E \in \sigma (\textbf {X}), \ であるから,定理1により各n \in \mathbb{N}に対して,
あるE_{n} \in \sigma (X_{1}, X_{2}, \dots , X_{n}) \ が取れて \ P(E \triangle E_{n}) \to 0 となる.
このことからP(E_{n}) \to P(E),\\ P(E_{n} \cup E) \to P(E)がわかる.\\
\because \\
\bullet P(E_{n}) \to P(E)\\
P(E_{n}) \le P((E_{n}-E) \cup E ) \le P(E_{n}-E) + P(E)からP(E_{n})-P(E) \le P(E_{n}-E) \le P(E_{n} \triangle E) \to 0 \ (n \to \infty). \ 同様にして P(E)-P(E_{n}) \le P(E-E_{n}) \le P(E_{n} \triangle E) \to 0がわかる.\\
\bullet P(E_{n} \cup E) \to P(E) \\
P(E \cup E_{n}) \le P((E_{n}-E) \cup E) \le P(E_{n}-E) + P(E) \le P(E_{n} \triangle E) + P(E)から\\
P(E \cup E_{n})-P(E) \le P(E_{n} \triangle E) \to 0 \ (n \to \infty).
また,E \subset (E \cup E_{n}) \cup (E \triangle E_{n})だからP(E)-P(E \cup E_{n}) \le P(E \triangle E_{n}) \to 0 \ (n \to \infty) \\
この時, E \in \delta だから, E \in \sigma (X_{n+1}, X_{n+2}, \dots) \ である.つまり, EとE_{n}は独立であることがわかる.\\ 
P(E \cap E_{n} ) = P(E)P( E_{n} )であり.\\ 各辺でn \to \infty とすれば, \ P(E) = P(E)^{2} であるから, P(E)=0,1 となることがわかった.
$
\end{proof}

%ボレルカンテリの補題
%\{, \lbrace	\}, \rbrace
\begin{lem}
Borel-Ccantelli Lamma 
\end{lem}
(\textbf{I}), ${\lbrace A_{n} \rbrace }_{n \in \mathbb{N}} \in \mathcal{F}  について, \sum_{n=1}^{\infty} P(A_{n}) < \infty ならば, P(A_{n} \ i.o.) = 0 \ が成立する.$ \par
(\textbf{II}), ${\lbrace A_{n} \rbrace }_{n \in \mathbb{N}} \in \mathcal{F}  について, {\lbrace A_{n} \rbrace }_{n \in \mathbb{N}} が独立かつ,\sum_{n=1}^{\infty}P(A_{n}) = \infty ならば, P(A_{n}  \ i.o.) = 1 \ が成立する.$
\begin{proof}
$\\ \\ \\
(\textbf{I})
$  \par

$P(A_{n} \ i.o.) = P(\displaystyle\lim_{m \to \infty} \displaystyle\bigcup_{n=m}^{\infty}{A_n}) = \displaystyle\lim_{m \to \infty} P(\displaystyle\bigcup_{n=m}^{\infty}{A_n}) \le \displaystyle\lim_{m \to \infty} (\sum_{n=m}^{\infty}P(A_{n})) (\because 二つ目の等号は測度の連続性,不等号には劣加法性を使った)\\
\sum_{n=1}^{\infty} P(A_{n}) < \infty であるから\displaystyle\lim_{m \to \infty} (\sum_{n=m}^{\infty}P(A_{n})) = 0 \ \therefore P(A_{n} \ i.o.) = 0
$

$(\textbf{II})
$ \par
$
{}^\forall m \in \mathbb{N}に対して,P(\displaystyle\bigcap_{n=m}^{\infty}{A_n}^{c})=0を示せば, P({(A_{n} \ i.o.)}^{c}) = P(\bigcup_{m \in \mathbb{N}}\bigcap_{n=m}^{\infty}{A_n}^{c}) = \displaystyle\lim_{m \to \infty} P(\bigcap_{n=m}^{\infty}{A_n}^{c}) = 0,つまりP(A_{n} \ i.o.)=1がわかる.\\
{}^\forall m \in \mathbb{N}を固定する. {\lbrace A_{n} \rbrace }_{n \in \mathbb{N}}は独立なのでP(\displaystyle\bigcap_{n=m}^{\infty}{A_n}^{c}) = \displaystyle\prod_{n=m}^{\infty} P({A_n}^{c}) = \displaystyle\prod_{n=m}^{\infty} (1-P(A_{n}))である.ここで\log{(1-x)} \le -x \ (0 \le x \le 1)を使うと, \ \log{(\displaystyle\prod_{n=m}^{\infty}(1-P(A_{n})))} = \sum_{n=m}^{\infty}\log{(1-P(A_{n}))} \le - \sum_{n=m}^{\infty}P(A_{n}) = -\infty . \  よってP(\displaystyle\bigcap_{n=m}^{\infty}{A_n}^{c}) = 0 \ \ ({}^\forall m \in \mathbb{N})が示せた.
$
\end{proof}

$\\$
いくつか応用例を挙げる.$\\$
(例1)コイントスを考える. \textbf {s} を長さkのH, T (表, 裏)が要素の列とする.
$A_{n} = \lbrace \omega \ ; (\omega_{n},\dots,\omega_{n+k-1}) = \textbf {s}  \rbrace$と定める.

%応用例1
\begin{prop}
$P(A_{n} \ i.o.) = 1$
\end{prop}
\begin{proof}
$B_{1} = \lbrace \omega \ ; (\omega_{1},\dots,\omega_{k}) = \textbf {s}  \rbrace , B_{2} = \lbrace \omega \ ; (\omega_{k+1},\dots,\omega_{2k}) = \textbf {s}  \rbrace , \ \dots$とおく. このとき, ${\lbrace B_{n} \rbrace}_{n \in \mathbb{N}}$は独立となる. また,$\lbrace B_{n} \ i.o. \rbrace \subset \lbrace A_{n} \ i.o.\rbrace$である$( \because \ B_{l} = A_{(l-1)k + 1})$ . \ $P(B_{n}) = P(B_{1}) = \frac{1}{2^{k}} > 0$なので$\displaystyle\sum_{n=1}^{\infty}P(B_{n}) = \infty$.以上のことから定理3$(\textbf{II})$を使うと, $P(B_{n} \ i.o.) = 1 \le P(A_{n} \ i.o.) \ \therefore P(A_{n} \ i.o.)=1$
\end{proof}

$\\$
(例2)再び,コイントスを考える. $
Y_{i}(\omega)= \left \{
\begin{array}{ll}
1 & (\omega_{i}がHのとき) \\
-1 & (\omega_{i}がTのとき)
\end{array}
\right. , Z_{n} = Y_{1} + \dots + Y_{n}$と定める. %$A_{n} = {Z_{n}=0}$とする.
\begin{prop}
$P(Head) \ne \frac{1}{2}$とする. このとき$P(Z_{n} = 0 \ i.o.) = 0$となる.
\end{prop}
\begin{proof}
$\quad$\par
$P(Head) = p$とおく.\par
$\displaystyle\sum_{n=1}^{\infty}P(Z_{n} = 0) < \infty$であることが示せれば, 定理3($\textbf{I}$)から$P(Z_{n} = 0 \ i.o.) = 0$がわかる.
Stirlingの近似\par 公式から,十分大きいnに対して, $^{}_{2n}C_{n} = 2^{2n} \frac{1+\delta_{n}}{\sqrt{\pi n}}$(ただし$ \delta_{n} \downarrow 0$)であり, また,$\ p \neq \frac{1}{2}$なので$2^{2}p(1-p)$\par $< 1$より,ある$0 < \lambda < 1$ が存在して$2^{2}p(1-p) < \frac{1}{\lambda} 2^{2}p(1-p) < 1$ となる.$\delta_{n} \downarrow 0$だから十分大きいn\par に対して は$\delta_{n} < \frac{\lambda}{2^{2}p(1-p)} -1$が成立する.\par 
以上で$N \in \mathbb{N}$ を, $n \ge N $で $\displaystyle P(Z_{2n})=^{}_{2n}C_{n} \ p^{n} (1-p)^{n} = 2^{2n} \frac{1+\delta_{n}}{\sqrt{\pi n}} \ p^{n} (1-p)^{n}$ かつ$\displaystyle\delta_{n} < \frac{\lambda}{2^{2}p(1-p)} -1$\par を満たすようにとる. $\displaystyle a_{n} = 2^{2n} \frac{1+\delta_{n}}{\sqrt{\pi n}} \ p^{n} (1-p)^{n}$とおく.\ $n \ge N$ において $\displaystyle\frac{a_{n+1}}{a_{n}} =  2^{2} \frac{1+\delta_{n+1}}{1+\delta_{n}} \sqrt{\frac{n}{n+1}}$\par $\displaystyle p(1-p) \le 2^{2} \sqrt{\frac{n}{n+1}}\frac{\lambda}{2^{2}p(1-p)} p(1-p)= \lambda \sqrt{\frac{n}{n+1}} \le \lambda $だから, $\displaystyle a_{n+1} \le (1-\lambda)a_{n} \le \dots \le (1-\lambda)^{n+1-N}a_{N}$\par 
$\displaystyle\sum_{n=1}^{\infty}P(Z_{2n} = 0) = \displaystyle\sum_{n=1}^{N}P(Z_{2n} = 0) + \displaystyle\sum_{n=N+1}^{\infty}P(Z_{2n} = 0) \le \displaystyle\sum_{n=1}^{N}P(Z_{2n} = 0) + \displaystyle\sum_{n=N+1}^{\infty} \lambda^{n-N} a_{N}$\par
$\le \displaystyle\sum_{n=1}^{N}P(Z_{2n} = 0) + a_{N} \displaystyle\sum_{n=1}^{\infty} \lambda^{n} =\displaystyle\sum_{n=1}^{N}P(Z_{2n} = 0) + a_{N} \frac{\lambda}{1-\lambda} < \infty \  (\because \ 0 < \lambda < 1)$\par
以上で$\displaystyle\sum_{n=1}^{\infty}P(Z_{n} = 0) = \displaystyle\sum_{n=1}^{\infty}P(Z_{2n} = 0) < \infty$がわかった.
\end{proof}
$\\$

\begin{thm}
$P(Head) = \frac{1}{2}$とする. このとき$P(Z_{n} = 0 \ i.o.) = 1$となる.
\end{thm}
\begin{proof}
$\\$
$n_{1} < n_{2} < \dots$の自然数列とする. また, 各$k \in \mathbb{N}$に対して,$n_{k} < m_{k} < n_{k+1}$となるように$m_{1} < m_{2} < \dots$をとる.
$C_{k} = \lbrace Y_{n_{k}+1} + \dots + Y_{m_{k}} \le -n_{k} \rbrace \bigcap  \lbrace Y_{m_{k}+1} + \dots + Y_{n_{k+1}} \ge m_{k} \rbrace$と定める.$\\$ $Y_{i} = -1, 1$だから$-n \le Z_{n} \le n$となることを使うと,
$\omega \in C_{k}$ に対して,$Z_{m_{k}}(\omega) = (Y_{1} + \dots + Y_{m_{k}})(\omega) = (Y_{1} + \dots + Y_{n_{k}})(\omega) + (Y_{n_{k}+1} + \dots + Y_{m_{k}})(\omega) \le n_{k} - n_{k}=0 \\ $ また$Z_{n_{k+1}}(\omega) = (Y_{1} + \dots + Y_{m_{k}})(\omega) + (Y_{m_{k}+1} + \dots + Y_{n_{k+1}})(\omega) \ge -m_{k} + m_{k} = 0$よって,$\\ \omega \in C_{k}$に対して$Z_{m_{k}}(\omega) \le 0, Z_{n_{k+1}}(\omega) \ge 0$であり, $Z_{n+1} = Z_{n} \pm 1$となることから $\\ C_{k} \subset \lbrace Z_{n} = 0 ; n_{k}+1 \le {}^\exists{n} \le n_{k+1} \rbrace = \displaystyle\bigcup_{n=n_{k}+1}^{n_{k+1}} \lbrace Z_{n}=0 \rbrace \\$
$\lbrace C_{n} \ i.o. \rbrace = \displaystyle\bigcap_{m=1}^{\infty} \displaystyle\bigcup_{k=m}^{\infty} C_{k} \subset \displaystyle\bigcap_{m=1}^{\infty} \displaystyle\bigcup_{k=m}^{\infty} \displaystyle\bigcup_{n=n_{k}+1}^{n_{k+1}} \lbrace Z_{n}=0 \rbrace \subset  \displaystyle\bigcap_{m=1}^{\infty}  \displaystyle\bigcup_{n=n_{m}+1}^{\infty} \lbrace Z_{n}=0 \rbrace = \lbrace Z_{n} =0 \ i.o. \rbrace \\$ Borel-Ccantelli Lammaから \ $\displaystyle\sum_{n=1}^{\infty} P(C_{n}) = \infty$となれば $1 = P( C_{n} \ i.o. ) \le P(Z_{n}=0 \ i.o. )$となる.つまり $\displaystyle\sum_{n=1}^{\infty} P(C_{n}) = \infty$となるような自然数列$\lbrace n_{k} \rbrace , \lbrace m_{k} \rbrace$が取れることを示せばよい.
$\\ \bullet \ {}^\forall \alpha \in (0,1), \ {}^\forall k \in \mathbb{N} $ に対して,${}^\exists \varphi (k) \ge 1 \ s.t. \ P(|Z_{\varphi (k)}| < k) \le \alpha$となる.
$\\ \quad (proof) \ {}^\forall \alpha \in (0,1), \ {}^\forall k \in \mathbb{N} , \  {}^\forall j \in \mathbb{Z}$ を固定する,$P(Z_{n}=j) \rightarrow 0 \ (n \rightarrow \infty)$であるから,$ \displaystyle\sum_{|j|<k} P(Z_{n} = j) \rightarrow 0 \ (n \rightarrow \infty)$となる.よって,$\varphi (k)$を$\displaystyle\sum_{|j|<k} P(Z_{\varphi (k)} = j) \le \alpha$となるように取れる. [証明終り]
%続き
$\\$ ${}^\forall \alpha \in (0,1), \ {}^\forall k \in \mathbb{N}$を固定する. $\lbrace \varphi (k) \rbrace_{k \in \mathbb{N}}$を上で示したものと同様にとる. $n_{k}, m_{k}$を$n_{1}=1, \\ m_{k} = n_{k} + \varphi (n_{k}), \ n_{k+1} = m_{k} + \varphi (m_{k})$とする.
%
$\\ P(C_{k}) = P(Y_{n_{k} + 1} + \dots + Y_{m_{k}} \le -n_{k}) P(Y_{m_{k} + 1} + \dots + Y_{n_{k+1}} \ge m_{k})$  ( $\because \ {\lbrace Y_{i}} \rbrace_{ i \in \mathbb{N}}$は独立)
$\\ P(Head) = \frac{1}{2}$であるから,対象性を使うと$\\ P(\left| Y_{n_{k} + 1} + \dots + Y_{m_{k}} \right| \ge n_{k}) = 2P(Y_{n_{k} + 1} + \dots + Y_{m_{k}} \le -n_{k})$
$\\ P(\left| Y_{m_{k} + 1} + \dots + Y_{n_{k+1}} \right| \ge m_{k}) = 2P(Y_{m_{k} + 1} + \dots + Y_{n_{k+1}} \ge m_{k})$となるから,
$\\ P(C_{k}) = \frac{1}{4} P(\left| Y_{n_{k} + 1} + \dots + Y_{m_{k}} \right| \ge n_{k}) P(\left| Y_{m_{k} + 1} + \dots + Y_{n_{k+1}} \right| \ge m_{k})  \ , \lbrace Y_{i} \rbrace_{ i \in \mathbb{N}}$の同一分布性から
$\\ \qquad  =\frac{1}{4} P(\left| Y_{1} + \dots + Y_{m_{k}-n_{k}} \right| \ge n_{k}) P(\left| Y_{1} + \dots + Y_{n_{k+1}-m_{k}} \right| \ge m_{k})  \ , \varphi (k)$の定め方から,
$\\ \qquad  =\frac{1}{4} P(\left| Y_{1} + \dots + Y_{\varphi(n_{k})} \right| \ge n_{k}) P(\left| Y_{1} + \dots + Y_{\varphi(m_{k})} \right| \ge m_{k}) \ge \frac{1}{4} (1-\alpha)^{2}$
$\\ \qquad \because \displaystyle\sum_{|j|<k} P(Z_{\varphi (k)} = j)$ $\underbrace{=}_{\lbrace Z_{\varphi (k)} = j \rbrace_{|j|<k} は非交和}$ $P( \displaystyle\bigcup_{|j|<k} Z_{\varphi (k)} = j) = P(\left| Y_{1} + \dots + Y_{\varphi(k)} \right| < k) \le \alpha$なので$\\ \qquad \qquad P(\left| Y_{1} + \dots + Y_{\varphi(k)} \right| \ge k) \ge 1- \alpha$
$\\$ 以上で $ \displaystyle\sum_{k=1}^{\infty} P(C_{k}) \ge  \displaystyle\sum_{k=1}^{\infty} \frac{1}{4} (1-\alpha)^{2} = \infty$となって,$\ P(Z_{n} = 0 \ i.o.) = 1$が示せた.
%$\stackrel{=}{as}$
%$\stackrel{=}{\lbrace Z_{\varphi (k)} = j \rbrace_{|j|<k} は非交和} $ 
%$11 \displaystyle{=}_{\lbrace Z_{\varphi (k)} = j \rbrace_{|j|<k} は非交和}  11$
\end{proof}

%RANDOM SIGNS PROBLEM
\begin{thm}
$X_{1}, X_{2}, \dots$を独立確率変数とする.$\\$
このとき, \par $\displaystyle \sum_{k=1}^{n}X_{k}$ が確率収束する $\Leftrightarrow \displaystyle\sum_{k=1}^{n}X_{k}$ が概収束する$\quad$が成立する.
\end{thm}
まず、補題を示す.
%補題
\begin{lem}$N \in \mathbb{N}$を固定する. $X_{1}, X_{2}, \dots, X_{N}$を独立確率変数とし,\ $S_{n} = X_{1} + \dots + X_{n}$とおく.$\\$
${}^\forall \alpha > 0$に対して, $\displaystyle\sup_{1 \le j \le N} P(|S_{N}-S_{j}| > \alpha) = c < 1$となるとき, $\\ \displaystyle P( \sup_{1 \le j \le N} |S_{j}| > 2\alpha ) \le \frac{1}{1-c} P(|S_{N}|>\alpha)$となる.
\end{lem}
\begin{proof}
$\\$
$j^{*} (\omega)を|S_{j}(\omega)| > 2 \alpha$となる$1 \le j \le N$で一番小さいものとする.存在しないときは0とする.ここで$\displaystyle\bigcup_{1 \le j \le N} \lbrace j^{*} = j \rbrace = \emptyset$であるとき$\displaystyle P( \sup_{1 \le j \le N} |S_{j}| > 2\alpha ) = 0$なので,$\displaystyle P( \sup_{1 \le j \le N} |S_{j}| > 2\alpha ) = 0 \le \frac{1}{1-c} P(|S_{N}|>\alpha)$が成立する.よって,$\displaystyle\bigcup_{1 \le j \le N} \lbrace j^{*} = j \rbrace \neq \emptyset$のときを考える. $\\ \displaystyle P(|S_{N}| > \alpha, \ \sup_{1 \le j \le N} |S_{j} | > 2\alpha) = \sum_{j=1}^{N} P(|S_{N}| > \alpha, \ j^{*} = j) \ge \sum_{j=1}^{N} P(|S_{N} - S_{j}| \le \alpha, \ j^{*} = j) \\$
%becuase1
$\displaystyle \because \ \bullet \bigcup_{1 \le j \le N} \lbrace j^{*} = j \rbrace = \lbrace \sup_{1 \le j \le N}|S_{j}| > 2\alpha \rbrace$を示せば,一つ目の等号が成立する.\par
$(\subset)$ \par
$\omega \in$ (左辺)とすれば, $1 \le {}^\exists j \le N \ s.t. \ j^{*}(\omega) = k$だから $|S_{k}(\omega)|>2 \alpha$なので$\displaystyle\sup_{1 \le j \le N}  |S_{j}| \ge |S_{k}(\omega)| > 2\alpha$となって, $\omega \in$ (右辺) \par
$( \supset )$ \par
$\omega \in$ (右辺)とすれば, $\displaystyle\sup_{1 \le j \le N } |S_{j}(\omega)| > 2\alpha$であるから, ${}^\exists \lbrace k_{1}, k_{2}, \dots, k_{K} \rbrace \subset \lbrace 1,2,\dots ,N \rbrace \ s.t. \\ |S_{k_{m}}|>2 \alpha \  (m = 1,2,\dots, K) $となる. $\displaystyle j^{**}(\omega)= \min{ \lbrace k_{1}, k_{2}, \dots, k_{K} \rbrace }$とすれば$j^{*}(\omega) = j^{**}(\omega)$となるから,$\omega \in $(左辺)となる.
$ \\ $
%becuase 2
$\bullet $ 各k $\in \lbrace 1,2,\dots, N \rbrace$ に対して,$\ \lbrace |S_{N}| > \alpha \rbrace \cap \lbrace j^{*} = j \rbrace \supset \lbrace |S_{N}-S_{j}| \le \alpha \rbrace  \cap \lbrace j^{*} = j \rbrace$となるのを示せば2つ目の不等号が示せる.\par
k $\in \lbrace 1,2,\dots, N \rbrace$を固定しておく.$\omega \in $(右辺)をとる. $|S_{N}(\omega) - S_{j}(\omega)| \le \alpha$かつ$j^{*}(\omega) = j$であるから,
$\\$ $|S_{j}(\omega)| - |S_{N}(\omega)| \le \alpha$かつ$|S_{j}(\omega)| > 2\alpha$ $\Leftrightarrow$ $|S_{j}(\omega)| - \alpha \le |S_{N}(\omega)|$かつ$ |S_{j}(\omega)| > 2\alpha$\par $\quad\qquad\qquad\qquad\qquad\qquad\qquad\qquad\qquad \Rightarrow 2\alpha - \alpha = \alpha < |S_{N}(\omega)|$\par
以上で$ \lbrace |S_{N}| > \alpha \rbrace \cap \lbrace j^{*} = j \rbrace \supset \lbrace |S_{N}-S_{j}| \le \alpha \rbrace  \cap \lbrace j^{*} = j \rbrace $
$\displaystyle\\ \ \lbrace j^{*} = j \rbrace = ( \bigcap_{k=1}^{j-1}\lbrace |S_{k}| > 2\alpha \rbrace^{c} ) \cap \lbrace |S_{j}| > 2\alpha \rbrace$なので,$\lbrace j^{*} = j \rbrace \in \sigma (X_{1}, \dots , X_{j})$,
$\\ \lbrace |S_{N} - S_{j} | \le \alpha \rbrace \in \sigma (X_{j+1}, \dots , X_{N})$であるから, $ \lbrace j^{*} = j \rbrace $と$ \lbrace |S_{N} - S_{j} | \le \alpha \rbrace $は独立.
仮定から$ P(|S_{N}-S_{j}| > \alpha) \le c $なので$1 - P(|S_{N}-S_{j}| > \alpha) = P(|S_{N}-S_{j}| \le \alpha) \ge 1-c$
%まとめ
$\\ \displaystyle \sum_{j=1}^{N}P(|S_{N}-S_{j}| \le \alpha , \ j^{*}=j) = \sum_{j=1}^{N} P(|S_{N}-S_{j}| \le \alpha)P( j^{*}=j) \ge (1-c)\sum_{j=1}^{N} P(j^{*}=j) \\ = (1-c) P(\sup_{1 \le j \le N} |S_{j}| > 2\alpha) \\ (1-c) P(\sup_{1 \le j \le N} |S_{j}| > 2\alpha) \le \sum_{j=1}^{N}P(|S_{N}-S_{j}| \le \alpha , \ j^{*}=j)  \le P(|S_{N}| > \alpha, \ \sup_{1 \le j \le N} |S_{j} | > 2\alpha) \\ \le P(|S_{N}| > \alpha) \quad \therefore P(\sup_{1 \le j \le N} |S_{j}| > 2\alpha) \le \frac{1}{1-c} P(|S_{N}| > \alpha)$
\end{proof}
補題 8を使って,定理7の証明をする.
%定理7の証明
\begin{proof}
$\\ (\Leftarrow) \ 概収束するならば確率収束するので成立する.$
$\displaystyle \\ (\Rightarrow) \sum_{k=1}^{n}X_{k}$は確率収束するとする. ここで$\displaystyle \sum_{k=1}^{n}X_{k}$が概収束しないと仮定する.(背理法)
$\\$ここで実数列$\lbrace s_{n} \rbrace$が収束しないとすれば$\lbrace s_{n} \rbrace$はCauchy列でないので
$\\{}^\exists \varepsilon>0 \ s.t. \ {}^\forall N \in \mathbb{N} , \ {}^\forall n,m \ge N \land |s_{n}-s_{m}| > \varepsilon$であるから,
$\displaystyle \\ {}^\exists \varepsilon>0 \ s.t. \ {}^\forall m \in \mathbb{N}, \ \sup_{n>m} |s_{n}-s_{m}| > \varepsilon$となる. $\displaystyle \sum_{k=1}^{n}X_{k}$はほとんど確実にCauchy列でないから,
$\displaystyle \\ {}^\exists \varepsilon >0, \ {}^\exists \delta \in (0,1] \ s.t. \ \left[ {}^\forall m \in \mathbb{N},\ P\left(\sup_{n>m} |\sum_{k=1}^{n}X_{k} - \sum_{k=1}^{m}X_{k}| > \varepsilon \right) \ge \delta  \right]$となる.この$\varepsilon, \ \delta$を固定する.
%Cm,Nを定める.
 $\\ \displaystyle \displaystyle \sum_{k=1}^{n}X_{k}$は確率収束するので$\displaystyle \sum_{k=1}^{N} X_{k} - \sum_{k=1}^{m}X_{k} \stackrel{P}{\rightarrow} 0 $となる. 
%何故ならば
$\displaystyle \\ \because \sum_{k=1}^{n}X_{k} \stackrel{P}{\rightarrow} s$とすると, $\displaystyle P\left( \left| \sum_{k=1}^{N} X_{k} - \sum_{k=1}^{m}X_{k} \right| > \varepsilon\right) = P\left( \left| \sum_{k=1}^{N} X_{k} - s +s - \sum_{k=1}^{m}X_{k} \right| > \varepsilon\right) \le P\left( \left| \sum_{k=1}^{N} X_{k} - s\right| + \left| s - \sum_{k=1}^{m}X_{k} \right| > \varepsilon\right) \le P\left( \left\{ \left| \sum_{k=1}^{N} X_{k} - s\right| > \frac{\varepsilon}{2} \right\} \cup \left\{  \left| s - \sum_{k=1}^{m}X_{k} \right| > \frac{\varepsilon}{2} \right\} \right) \\ \le P\left( \left| \sum_{k=1}^{N} X_{k} - s\right| > \frac{\varepsilon}{2} \right) + P\left( \left| \sum_{k=1}^{m} X_{k} - s\right| > \frac{\varepsilon}{2} \right) \to 0 \ (m,\ N \to \infty)$となるから. $\\$
よって,ある$M \in \mathbb{N}$が存在して, ${}^\forall m, \ N \ge M \ (m < N)$に対して,$\displaystyle P\left( \left| \sum_{k=m+1}^{N} X_{k} \right| > \frac{\varepsilon}{2}\right) < 1$で, $\displaystyle P\left( \left| \sum_{k=m+1}^{N} X_{k} \right| > \frac{\varepsilon}{2}\right) \to 0 \ (m, \ N \to \infty)$この$m, \ N $を固定する.$\\$
$ \displaystyle C_{m, N} = \sup_{m < n \le N} P\left( \left| \sum_{k=n}^{N}X_{k} \right| > \frac{\varepsilon}{2} \right)$とおくと,$ \displaystyle \  C_{m, N} < 1$ かつ$\displaystyle C_{m, N} \to 0 \ (m, \ N \to \infty)$となる.

%
ここで補題 8を使うと,
$\displaystyle \\ P\left(\sup_{ m < n \le N } \left| \sum_{ k = m+1 }^{n} X_{k} \right| > \varepsilon \right) \le \frac{1}{1-C_{m, N}} P\left( \left| \sum_{ k = m+1 }^{N} X_{k} \right| > \frac{\varepsilon}{2} \right)$ とかけて,
まず$\displaystyle N \to \infty$とすると,$\\$
単調性から, $\displaystyle \lim_{N \to \infty} P\left(\sup_{ m < n \le N } \left| \sum_{ k = m+1 }^{n} X_{k} \right| > \varepsilon \right) =   P\left( \lim_{N \to \infty} \sup_{ m < n \le N } \left| \sum_{ k = m+1 }^{n} X_{k} \right| > \varepsilon \right)=   P\left( \sup_{ m < n} \left| \sum_{ k = m+1 }^{n} X_{k} \right| > \varepsilon \right)$ 
$\displaystyle \\ \le \lim_{N \to \infty} \frac{1}{1-C_{m, N}} P\left( \left| \sum_{ k = m+1 }^{N} X_{k} \right| > \frac{\varepsilon}{2} \right)$, $\ \displaystyle \lim_{m \to \infty} \lim_{N \to \infty} \frac{1}{1-C_{m, N}} P\left( \left| \sum_{ k = m+1 }^{N} X_{k} \right| > \frac{\varepsilon}{2} \right)$だから,
$\displaystyle \\ \lim_{m \to \infty}  P\left( \sup_{ m < n} \left| \sum_{ k = m+1 }^{n} X_{k} \right| > \varepsilon \right) = 0$
 これは$\displaystyle {}^\forall m \in \mathbb{N},\ P\left(\sup_{n>m} |\sum_{k=1}^{n}X_{k} - \sum_{k=1}^{m}X_{k}| > \varepsilon \right) \ge \delta > 0$に矛盾する. 背理法により $\displaystyle \sum_{k=1}^{n}X_{k}$は概収束することがわかった.
\end{proof}

%系
\begin{cor}
$\displaystyle \\ E[X_{k}] = 0 \ ( {}^\forall k \in \mathbb{N} ), \ \sum_{k=1}^{\infty}E[X_{k}^{2}] < \infty$とする.このとき$\displaystyle \sum_{k=1}^{n}X_{ k }$は確率収束する.
\end{cor}
\begin{proof}
$\displaystyle \\ X_{1}, X_{2}, \dots$は独立なので,$\displaystyle \sum_{k=1}^{n}X_{ k }$が確率収束することを示せば定理8から$\displaystyle \sum_{k=1}^{n}X_{ k }$は確率収束する.
$\displaystyle \\ \sum_{k=1}^{\infty}E[X_{k}^{2}] = s^{2}$ (ただし $s \ge 0$ )とする. ${}^\forall \varepsilon > 0$ に対してChebyshevの不等式から $\displaystyle \\ P\left( \left| \sum_{k=1}^{n}X_{ k } - s \right| > \varepsilon \right) \le \frac{1}{ \varepsilon ^ {2} } E\left[ \left| \sum_{k=1}^{n}X_{ k } - s \right|^{2} \right]$ となる.
$\displaystyle E\left[ \left| \sum_{k=1}^{n}X_{ k } - s \right|^{2} \right] \to 0 \ (n \to \infty )$を示したい. $\displaystyle \\ E\left[ \left| \sum_{k=1}^{n}X_{ k } - s \right|^{2} \right] = E\left[ \left| \sum_{k=1}^{n}X_{ k } \right|^{2} \right] -2sE\left[ \sum_{k=1}^{n}X_{ k } \right] + s^{2} \\ = E\left[ \sum_{k=1}^{n}X_{ k }^{2} + 2 \sum_{i<j}X_{ i }X_{ j }  \right]-2sE\left[ \sum_{k=1}^{n}X_{ k } \right] + s^{2} $ここで,$\displaystyle X_{1}, X_{2}, \dots$は独立だから 
$\displaystyle \\ \sum_{i<j} E\left[ X_{ i }X_{ j }  \right] = \sum_{i<j} E\left[ X_{ i } \right] E\left[X_{ j }  \right]$が成立する. また$\displaystyle E[X_{k}] = 0 \ ( {}^\forall k \in \mathbb{N} )$なので
$\displaystyle \\ = \sum_{k=1}^{n} E\left[ {X_{ k }}^{2} \right] -2s\sum_{k=1}^{n} E\left[ X_{ k } \right] + s^{2} \to s^{2} -2s^{2}+s^{2}=0 \ (n \to \infty) \\ \sum_{k=1}^{n}X_{ k }$が確率収束することがわかったので$\displaystyle \sum_{k=1}^{n}X_{ k }$は確率収束する.
\end{proof}


%独立確率変数に対する大数の法則
\begin{thm}
独立確率変数に対する大数の法則$\\$
$X_{1}, X_{2}, \dots$を独立確率変数とする. $E\left[ X_{k} \right] = 0, \ E\left[ X_{k}^{2} \right] < \infty \ \left( {}^\forall k \in \mathbb{N} \right)$であるとする.
正数列 $\\ \lbrace b_{n} \rbrace_{ n \in \mathbb{N} }$が$b_{n} \uparrow \infty$かつ$\displaystyle\sum_{k=1}^{\infty} E\left[ \frac{X_{k}^{2}}{b_{k}^{2}} \right] < \infty$を満たすとき,$\displaystyle\frac{X_{1} + \dots + X_{n}}{b_{n}} \stackrel{a.s.}{\longrightarrow}  0 $が成立する.
\end{thm}
証明の前に一つ補題を示す.
\begin{lem}
Kronecker's Lemma
$\\$ $x_{1}, x_{2}, \dots$を$\displaystyle\sum_{k=1}^{n} x_{k} \to s < \infty$を満たす実数列とする.このとき, $b_{n} \uparrow \infty$となる整数列$\lbrace b_{n} \rbrace_{n \in \mathbb{N}}$が取れて, $\displaystyle\frac{1}{b_{n}} \sum_{k=1}^{n} b_{k} x_{k} \to 0 $となる.
\end{lem}
\begin{proof}
$\\$
$\displaystyle r_{n} = \sum_{k=n+1}^{\infty}x_{k}, \ r_{0} = s$とおく.このとき$x_{n} = r_{n-1} - r_{n}, \ n = 1,2, \dots$.また,$\displaystyle\sum_{k=1}^{n} b_{k}x_{k} = \\ \sum_{k=1}^{n} b_{k} (r_{k-1}-r_{k}) = \sum_{k=0}^{n-1} b_{k+1} r_{k} - \sum_{k=1}^{n} b_{k} r_{k}  = \sum_{k=1}^{n-1} (b_{k+1}-b_{k})r_{k} + b_{1}s - b_{n}r_{n}$となるから
$\\ \displaystyle\left| \sum_{k=1}^{n} b_{k}x_{k} \right| \le \sum_{k=1}^{n-1}(b_{k+1}-b_{k}) |r_{k}| + b_{1}|s| + b_{n}|r_{n}| \ $   ($ \because$ 三角不等式, $b_{n}$は単調増加なので$b_{n+1}-b_{n} \ge 0$)$\\$
ここで${}^\forall \varepsilon >0$をとる. $\displaystyle\sum_{k=1}^{\infty}x_{k}$は収束するから $r_{k}$の定め方から $N \in \mathbb{N}$を${}^\forall n \ge N$に対して, $|r_{k}| \le \varepsilon$となるように取れる.このNを固定する. $\displaystyle \tilde{r} := \max \lbrace |r_{1}|, \dots, |r_{N-1}|, \varepsilon \rbrace$とする.
$n > N$において,$\\$
$\displaystyle \sum_{k=1}^{n-1}(b_{k+1}-b_{k}) |r_{k}| \le \sum_{k=1}^{N-1}(b_{k+1}-b_{k}) |r_{k}| + \varepsilon \sum_{k=N}^{n-1}(b_{k+1}-b_{k}) \le \tilde{r} (b_{N}-b_{1}) + \varepsilon (b_{n}-b_{N}) \ よって$ $\\$
$\displaystyle\left| \frac{ \sum_{k=1}^{n} b_{k}x_{k} }{b_{n}}\right| \le \frac{1}{b_{n}} ( \tilde{r}(b_{N}-b_{1}) + \varepsilon ( b_{n} -b_{N} ) + b_{1}|s| + b_{n}\varepsilon ) \to \varepsilon $つまり$\displaystyle \varlimsup_{n \to \infty} \left| \frac{ \sum_{k=1}^{n} b_{k}x_{k} }{b_{n}}\right| \le \varepsilon$となるから$\displaystyle \lim_{n \to \infty} \left| \frac{ \sum_{k=1}^{n} b_{k}x_{k} }{b_{n}}\right| \le \varepsilon$がわかった.ここで$\varepsilon \downarrow 0$とすれば,$\displaystyle\frac{1}{b_{n}} \sum_{k=1}^{n} b_{k} x_{k} \to 0 $が示された.
\end{proof}
この補題を使って定理7を証明する.
\begin{proof}
$\\$
Kronecker's Lemmaにより, $\displaystyle\sum_{k=1}^{n}\frac{X_{k}}{b_{k}}$がほとんど確実に収束すれば,$\displaystyle \frac{1}{b_{k}} \sum_{k=1}^{n} b_{k} \frac{X_{k}}{b_{k}} = \frac{1}{b_{k}} \sum_{k=1}^{n} X_{k} \stackrel{a.s.}{\longrightarrow}  0$となる.
\end{proof}


\end{document}
\documentclass{jsarticle}
\usepackage{amsthm}
\usepackage{amsmath}
\usepackage{amssymb}
\usepackage{eucal}
\usepackage{amsfonts}
\usepackage{mathrsfs}  


\newtheorem{thm}{定理}
\newtheorem{dfn}[thm]{定義}
\newtheorem{prop}[thm]{命題}
\newtheorem{lem}[thm]{補題}
\newtheorem{cor}[thm]{系}


\begin{document}


\title{課題研究bレポート}
\author{加納基晴}
\date{}
\maketitle

%ボレルカンテリの補題
%\begin{thm}

%Borel-Ccantelli Lamma 

%\end{thm}

%\begin{proof}

%\end{proof}

%section 末尾事象とinfinte often
以下では$(\Omega, \ \mathcal{F}, \ P)を確率空間とする.$

\section{末尾事象とinfinte often}
\begin{dfn} 
末尾事象 (tail event)
$\\ X_{1}, X_{2}, \dots$を確率変数列とする.
$\\ E \in \sigma (X_{1}, X_{2}, \dots)$が末尾事象(tail event)であるとは
$E \in \sigma (X_{n}, X_{n+1},\dots) \quad ({}^\forall n \in \mathbb{N}) \\ $が成立することである.
$ \\$末尾加法族(tail  $\sigma$-field) $\delta$を
$\displaystyle \delta = \bigcap_{n \in \mathbb{N}} \sigma (X_{n}, X_{n+1}, \dots)$と定める.
$\\$ 定義から末尾加法族の元 $E \in \delta$ は末尾事象となる.
\end{dfn}

%近似定理
\begin{thm}
近似定理 
$\\ X_{1}, X_{2}, \dots$ を確率変数列とする.
$\displaystyle \\ {}^\forall A_{1} \in \sigma (\textbf{X}),{}^\forall \varepsilon > 0$ に対して, ある$n \in \mathbb{N}, A_{2} \in \mathcal{F} (X_{1}, X_{2}, \dots , X{n})$が存在して$P(A_{1} \triangle A_{2}) \le 0$ となる. $\\$ (ただし $A \triangle B := (A-B) \cup (B-A)$ )
\end{thm}

\begin{proof}
$\\$
${}^\forall A_{1} \in \sigma (\textbf{X}), {}^\forall \varepsilon > 0$を固定する. 
$\displaystyle \\ \mathscr{F}_{0} = \displaystyle\bigcup_{n \in  \mathbb{N}} \sigma (X_{1}, X_{2}, \dots, X_{n}) $, $\mathcal{C} = {\lbrace A \in \mathcal{F} | \ {}^\forall \varepsilon > 0 に対して, {}^\exists B \in \mathscr{F}_{0} \ s.t. \ P(A \triangle B) \le \varepsilon \rbrace  }$と定める. 
$\displaystyle \\ \mathscr{F}_{0}  \subset \mathcal{C}$は明らかだから,$\mathcal{C}$が$\sigma$加法族であることを示せば, $\sigma (\mathscr{F}_{0}) \subset \mathcal{C}$で, $\sigma (\mathscr{F}_{0}) = \sigma (\textbf {X})$であることから,
$\displaystyle \\ A_{1} \in \sigma (\textbf{X}) \subset \mathcal{C}$なので,${}^\exists A_{2} \in \mathscr{F}_{0} \ s.t. \ P(A_{1} \triangle A_{2}) \le \varepsilon$となり、定理が成立するのがわかる.
%%証明
$\displaystyle \\ \bullet \ \mathcal{C}$が$\sigma$加法族であること示す.\par
$\quad$ $(i)$ $\Omega \in \mathcal{C} \ (\because \ \Omega \in \ \mathscr{F}_{0}) $\par

$\quad$ $(ii)$ ${}^\forall A \in \mathcal{C}に対して, A^c \in \mathcal{C}$\par
$\qquad$ $\because {}^\forall \varepsilon > 0$ を固定する.このとき$B \in \mathcal{F}_{0}$が取れて, $P(A \triangle B) \le \varepsilon $となる.$\mathcal{F}_{0}$の定め方から, \ $B^c \in \mathcal{F}_{0}$ \par $\quad \qquad$であって, $P(A^c \triangle B^c) = P((A^c \cap B) \cup (A \cap B^c)) = P((B-A) \cup (A-B)) = P(A \triangle B) \le \varepsilon$\par $\quad \qquad$ $\therefore A^c \in \mathcal{C}$

$\quad$ $(iii)$ ${}^\forall \lbrace A_{n} \rbrace_{n \in \mathbb{N}} \subset \mathcal{C}, {}^\forall \varepsilon > 0$をとる. $\lbrace B_{n} \rbrace_{n \in \mathbb{N}} \subset \mathcal{F}_{0}$を$P(A_{n} \triangle B_{n}) \le \frac{\varepsilon}{2^{n+1}} $となるようにとる.\par 
$\quad \qquad$また,測度の上からの連続性から ある$N \in \mathbb{N}$が取れて, $P( \displaystyle\bigcup_{n=N+1}^{\infty} A_{n}) \le \frac{\varepsilon}{2}$となる.\par$\quad \qquad$ ここで, $\displaystyle\bigcup_{n=1}^{N} B_{n} \triangle \displaystyle\bigcup_{n=1}^{\infty} A_{n} \subset (\displaystyle\bigcup_{n=1}^{N} B_{n} \triangle A_{n}) \cup \displaystyle\bigcup_{n=N+1}^{\infty} A_{n}$を示せれば,単調性と劣加法性から,\par $\quad \qquad$ $P(\displaystyle\bigcup_{n=1}^{N} B_{n} \triangle \displaystyle\bigcup_{n=1}^{\infty} A_{n}) \le P((\displaystyle\bigcup_{n=1}^{N} B_{n} \triangle A_{n}) \cup \displaystyle\bigcup_{n=N+1}^{\infty} A_{n}) \le  \displaystyle\sum_{n=1}^{N}P(B_{n} \triangle A_{n}) + P(\displaystyle\bigcup_{n=N+1}^{\infty} A_{n})$ \par $\quad \qquad$ $\le \frac{\varepsilon}{2} + \frac{\varepsilon}{2} = \varepsilon $となる. $\displaystyle\bigcup_{n=1}^{N} B_{n} \in \mathcal{F}_{0}$であることから $\displaystyle\bigcup_{n=1}^{\infty} A_{n} \in \mathcal{C}$となる.

%%証明の中の証明
$\displaystyle \\ \quad$ $\bullet \ \displaystyle\bigcup_{n=1}^{N} B_{n} \triangle \displaystyle\bigcup_{n=1}^{\infty} A_{n} \subset (\displaystyle\bigcup_{n=1}^{N} B_{n} \triangle A_{n}) \cup \displaystyle\bigcup_{n=N+1}^{\infty} A_{n}$を示す. \par
$\quad \qquad$ $\because \ \omega \in \displaystyle\bigcup_{n=1}^{N} B_{n} \triangle \displaystyle\bigcup_{n=1}^{\infty} A_{n} \Leftrightarrow (\omega \in \displaystyle\bigcup_{n=1}^{N} B_{n} - \displaystyle\bigcup_{n=1}^{\infty} A_{n}) \lor  (\omega \in \displaystyle\bigcup_{n=1}^{\infty} A_{n} - \displaystyle\bigcup_{n=1}^{N} B_{n})$ \par 
$\qquad \qquad \qquad$
$ \Leftrightarrow (\omega \in \displaystyle\bigcup_{n=1}^{N} B_{n} \cap \displaystyle\bigcap_{n=1}^{\infty} A_{n}^{c}) \lor 
 (\omega \in ( \displaystyle\bigcup_{n=1}^{N} A_{n} \cap \displaystyle\bigcap_{n=1}^{N} B_{n}^{c}) \cup (\displaystyle\bigcup_{n=N+1}^{\infty} A_{n} \cap \displaystyle\bigcap_{n=1}^{N} B_{n}^{c}) ) $\par 
$\qquad \qquad \qquad$
$\Rightarrow (\omega \in \displaystyle\bigcup_{n=1}^{N} B_{n} \cap \displaystyle\bigcap_{n=1}^{N} A_{n}^{c}) \lor (\omega \in ( \displaystyle\bigcup_{n=1}^{N} A_{n} \cap \displaystyle\bigcap_{n=1}^{N} B_{n}^{c}) \cup \displaystyle\bigcup_{n=N+1}^{\infty} A_{n} )$\par 
$\qquad \qquad \qquad$
$\Rightarrow \omega \in \displaystyle\bigcup_{n=1}^{N} (B_{n} \cap  A_{n}^{c}) \lor (\omega \in ( \displaystyle\bigcup_{n=1}^{N} A_{n} \cap B_{n}^{c}) \cup \displaystyle\bigcup_{n=N+1}^{\infty} A_{n} )$
\par 
$\qquad \qquad \qquad$
$\Leftrightarrow \omega \in \displaystyle\bigcup_{n=1}^{N} ( (B_{n} \cap  A_{n}^{c}) \cup (A_{n} \cap  B_{n}^{c}) ) \lor \omega \in \displaystyle\bigcup_{n=N+1}^{\infty} A_{n}$
\par 
$\qquad \qquad \qquad$
$\Leftrightarrow  \omega \in \displaystyle\bigcup_{n=1}^{N} ( B_{n} \triangle A_{n}) \lor \omega \in \displaystyle\bigcup_{n=N+1}^{\infty} A_{n}$
\par 
$\qquad \qquad \qquad$
$\Leftrightarrow \omega \in \displaystyle\bigcup_{n=1}^{N} ( B_{n} \triangle A_{n}) \cup \displaystyle\bigcup_{n=N+1}^{\infty} A_{n}$
$\qquad \quad$ $\therefore \displaystyle\bigcup_{n=1}^{N} B_{n} \triangle \displaystyle\bigcup_{n=1}^{\infty} A_{n} \subset (\displaystyle\bigcup_{n=1}^{N} B_{n} \triangle A_{n}) \cup \displaystyle\bigcup_{n=N+1}^{\infty} A_{n}$ \par
$\quad$$(i) \sim (iii)$より$\mathcal{C}$は$\sigma$加法族である.



\end{proof}

%ゼロワン法則
\begin{thm}
Kolmogorov zero-one law
\end{thm}

$X_{1},X_{2}, \dots を独立な確率変数とする.この時, E\in\delta であるとすればP(E)は0,1のいずれかの値をとる.$

\begin{proof}
$\\ {}^\forall E \in \delta をとる. \  E \in \sigma (\textbf {X}), \ であるから,定理2により各n \in \mathbb{N}に対して,
あるE_{n} \in \sigma (X_{1}, X_{2}, \dots , X_{n}) \ が取れて \ P(E \triangle E_{n}) \to 0 となる.
このことからP(E_{n}) \to P(E),\\ P(E_{n} \cup E) \to P(E)がわかる.\\
\because \\
\bullet P(E_{n}) \to P(E)\\
P(E_{n}) \le P((E_{n}-E) \cup E ) \le P(E_{n}-E) + P(E)からP(E_{n})-P(E) \le P(E_{n}-E) \le P(E_{n} \triangle E) \to 0 \ (n \to \infty). \ 同様にして P(E)-P(E_{n}) \le P(E-E_{n}) \le P(E_{n} \triangle E) \to 0がわかる.\\
\bullet P(E_{n} \cup E) \to P(E) \\
P(E \cup E_{n}) \le P((E_{n}-E) \cup E) \le P(E_{n}-E) + P(E) \le P(E_{n} \triangle E) + P(E)から\\
P(E \cup E_{n})-P(E) \le P(E_{n} \triangle E) \to 0 \ (n \to \infty).
また,E \subset (E \cup E_{n}) \cup (E \triangle E_{n})だからP(E)-P(E \cup E_{n}) \le P(E \triangle E_{n}) \to 0 \ (n \to \infty) \\
この時, E \in \delta だから, E \in \sigma (X_{n+1}, X_{n+2}, \dots) \ である.つまり, EとE_{n}は独立であることがわかる.\\ 
P(E \cap E_{n} ) = P(E)P( E_{n} )であり.\\ 各辺でn \to \infty とすれば, \ P(E) = P(E)^{2} であるから, P(E)=0,1 となることがわかった.
$
\end{proof}

%infinite often
\begin{dfn} 
infinite often
$\\ \left\{ A_{n} \right\}_{n \in \mathbb{N}} \subset \mathscr{F}$とする. $\left\{ A_{n} \ i.o. \right\}$を
 $\displaystyle \left\{ A_{n} \ i.o. \right\} = \lim_{m \to \infty} \bigcup_{n>m} A_{n}$と定める.
 $\\  \left\{ A_{n} \ i.o. \right\}$は $\left\{ \omega \ | \ \omega \in A_{n} となるnが無限個存在する. \right\}$ともかける.
 $\\ \displaystyle \because \omega \in \lim_{m \to \infty} \bigcup_{n>m} A_{n}$を任意にとる. $1$に対して$\omega \in A_{n_{1}}$となる$n_{1} \in \mathbb{N}$をとる.続いて $n_{1}$に対して$\omega \in A_{n_{2}}$となる$n_{2} \in \mathbb{N}$をとる.これを続ければ $\omega \in A_{n}$となる$n$の列として$\left\{ n_{k} \right\}_{k \in \mathbb{N}}$が取れるので, $\omega \in \left\{ \omega ' \ | \ \omega ' \in A_{n} となるnが無限個存在する. \right\}$である.
 $\\ \omega \in \left\{ \omega \ | \ \omega \in A_{n} となるnが無限個存在する. \right\}$を任意にとる.このとき任意の$m \in \mathbb{N}$に対して
 $\\ \omega \in A_{n}$ となる$n>m$が無限個存在する.よって $\displaystyle \omega \in \lim_{m \to \infty} \bigcup_{n>m} A_{n}$
 
\end{dfn}


%ボレルカンテリの補題
%\{, \lbrace	\}, \rbrace
\begin{lem}
Borel-Ccantelli Lamma 
\end{lem}
(\textbf{I}), ${\lbrace A_{n} \rbrace }_{n \in \mathbb{N}} \in \mathcal{F}  について, \sum_{n=1}^{\infty} P(A_{n}) < \infty ならば, P(A_{n} \ i.o.) = 0 \ が成立する.$ \par
(\textbf{II}), ${\lbrace A_{n} \rbrace }_{n \in \mathbb{N}} \in \mathcal{F}  について, {\lbrace A_{n} \rbrace }_{n \in \mathbb{N}} が独立かつ,\sum_{n=1}^{\infty}P(A_{n}) = \infty ならば, P(A_{n}  \ i.o.) = 1 \ が成立する.$
\begin{proof}
$\\ (\textbf{I}) $ 
$\\ P(A_{n} \ i.o.) = P(\displaystyle\lim_{m \to \infty} \displaystyle\bigcup_{n=m}^{\infty}{A_n}) = \displaystyle\lim_{m \to \infty} P(\displaystyle\bigcup_{n=m}^{\infty}{A_n}) \le \displaystyle\lim_{m \to \infty} (\sum_{n=m}^{\infty}P(A_{n})) (\because 二つ目の等号は測度の連続性,不等号には劣加法性を使った)\\
\sum_{n=1}^{\infty} P(A_{n}) < \infty であるから\displaystyle\lim_{m \to \infty} (\sum_{n=m}^{\infty}P(A_{n})) = 0 \ \therefore P(A_{n} \ i.o.) = 0
$

$(\textbf{II})
$ \par
$
{}^\forall m \in \mathbb{N}に対して,P(\displaystyle\bigcap_{n=m}^{\infty}{A_n}^{c})=0を示せば, P({(A_{n} \ i.o.)}^{c}) = P(\bigcup_{m \in \mathbb{N}}\bigcap_{n=m}^{\infty}{A_n}^{c}) = \displaystyle\lim_{m \to \infty} P(\bigcap_{n=m}^{\infty}{A_n}^{c}) = 0,つまりP(A_{n} \ i.o.)=1がわかる.\\
{}^\forall m \in \mathbb{N}を固定する. {\lbrace A_{n} \rbrace }_{n \in \mathbb{N}}は独立なのでP(\displaystyle\bigcap_{n=m}^{\infty}{A_n}^{c}) = \displaystyle\prod_{n=m}^{\infty} P({A_n}^{c}) = \displaystyle\prod_{n=m}^{\infty} (1-P(A_{n}))である.ここで\log{(1-x)} \le -x \ (0 \le x \le 1)を使うと, \ \log{(\displaystyle\prod_{n=m}^{\infty}(1-P(A_{n})))} = \sum_{n=m}^{\infty}\log{(1-P(A_{n}))} \le - \sum_{n=m}^{\infty}P(A_{n}) = -\infty . \  よってP(\displaystyle\bigcap_{n=m}^{\infty}{A_n}^{c}) = 0 \ \ ({}^\forall m \in \mathbb{N})が示せた.
$
\end{proof}

$\\$
いくつか応用例を挙げる.$\\$
(例1)コイントスを考える. \textbf {s} を長さkのH, T (表, 裏)が要素の列とする.
$A_{n} = \lbrace \omega \ ; (\omega_{n},\dots,\omega_{n+k-1}) = \textbf {s}  \rbrace$と定める.

%応用例1
\begin{prop}
$P(A_{n} \ i.o.) = 1$
\end{prop}
\begin{proof}
$B_{1} = \lbrace \omega \ ; (\omega_{1},\dots,\omega_{k}) = \textbf {s}  \rbrace , B_{2} = \lbrace \omega \ ; (\omega_{k+1},\dots,\omega_{2k}) = \textbf {s}  \rbrace , \ \dots$とおく. このとき, ${\lbrace B_{n} \rbrace}_{n \in \mathbb{N}}$は独立となる. また,$\lbrace B_{n} \ i.o. \rbrace \subset \lbrace A_{n} \ i.o.\rbrace$である$( \because \ B_{l} = A_{(l-1)k + 1})$ . \ $P(B_{n}) = P(B_{1}) = \frac{1}{2^{k}} > 0$なので$\displaystyle\sum_{n=1}^{\infty}P(B_{n}) = \infty$.以上のことから補題5$(\textbf{II})$を使うと, $P(B_{n} \ i.o.) = 1 \le P(A_{n} \ i.o.) \ \therefore P(A_{n} \ i.o.)=1$
\end{proof}

$\\$
(例2)再び,コイントスを考える. $
Y_{i}(\omega)= \left \{
\begin{array}{ll}
1 & (\omega_{i}がHのとき) \\
-1 & (\omega_{i}がTのとき)
\end{array}
\right. , Z_{n} = Y_{1} + \dots + Y_{n}$と定める. %$A_{n} = {Z_{n}=0}$とする.
\begin{prop}
$P(Head) \ne \frac{1}{2}$とする. このとき$P(Z_{n} = 0 \ i.o.) = 0$となる.
\end{prop}
\begin{proof}
$\quad$\par
$P(Head) = p$とおく.\par
$\displaystyle\sum_{n=1}^{\infty}P(Z_{n} = 0) < \infty$であることが示せれば, 補題5($\textbf{I}$)から$P(Z_{n} = 0 \ i.o.) = 0$がわかる.
Stirlingの近似\par 公式から,十分大きいnに対して, $^{}_{2n}C_{n} = 2^{2n} \frac{1+\delta_{n}}{\sqrt{\pi n}}$(ただし$ \delta_{n} \downarrow 0$)であり, また,$\ p \neq \frac{1}{2}$なので$2^{2}p(1-p)$\par $< 1$より,ある$0 < \lambda < 1$ が存在して$2^{2}p(1-p) < \frac{1}{\lambda} 2^{2}p(1-p) < 1$ となる.$\delta_{n} \downarrow 0$だから十分大きいn\par に対して は$\delta_{n} < \frac{\lambda}{2^{2}p(1-p)} -1$が成立する.\par 
以上で$N \in \mathbb{N}$ を, $n \ge N $で $\displaystyle P(Z_{2n})=^{}_{2n}C_{n} \ p^{n} (1-p)^{n} = 2^{2n} \frac{1+\delta_{n}}{\sqrt{\pi n}} \ p^{n} (1-p)^{n}$ かつ$\displaystyle\delta_{n} < \frac{\lambda}{2^{2}p(1-p)} -1$\par を満たすようにとる. $\displaystyle a_{n} = 2^{2n} \frac{1+\delta_{n}}{\sqrt{\pi n}} \ p^{n} (1-p)^{n}$とおく.\ $n \ge N$ において $\displaystyle\frac{a_{n+1}}{a_{n}} =  2^{2} \frac{1+\delta_{n+1}}{1+\delta_{n}} \sqrt{\frac{n}{n+1}}$\par $\displaystyle p(1-p) \le 2^{2} \sqrt{\frac{n}{n+1}}\frac{\lambda}{2^{2}p(1-p)} p(1-p)= \lambda \sqrt{\frac{n}{n+1}} \le \lambda $だから, $\displaystyle a_{n+1} \le (1-\lambda)a_{n} \le \dots \le (1-\lambda)^{n+1-N}a_{N}$\par 
$\displaystyle\sum_{n=1}^{\infty}P(Z_{2n} = 0) = \displaystyle\sum_{n=1}^{N}P(Z_{2n} = 0) + \displaystyle\sum_{n=N+1}^{\infty}P(Z_{2n} = 0) \le \displaystyle\sum_{n=1}^{N}P(Z_{2n} = 0) + \displaystyle\sum_{n=N+1}^{\infty} \lambda^{n-N} a_{N}$\par
$\le \displaystyle\sum_{n=1}^{N}P(Z_{2n} = 0) + a_{N} \displaystyle\sum_{n=1}^{\infty} \lambda^{n} =\displaystyle\sum_{n=1}^{N}P(Z_{2n} = 0) + a_{N} \frac{\lambda}{1-\lambda} < \infty \  (\because \ 0 < \lambda < 1)$\par
以上で$\displaystyle\sum_{n=1}^{\infty}P(Z_{n} = 0) = \displaystyle\sum_{n=1}^{\infty}P(Z_{2n} = 0) < \infty$がわかった.
\end{proof}
$\\$

\begin{thm}
$P(Head) = \frac{1}{2}$とする. このとき$P(Z_{n} = 0 \ i.o.) = 1$となる.
\end{thm}
\begin{proof}
$\\$
$n_{1} < n_{2} < \dots$の自然数列とする. また, 各$k \in \mathbb{N}$に対して,$n_{k} < m_{k} < n_{k+1}$となるように$m_{1} < m_{2} < \dots$をとる.
$C_{k} = \lbrace Y_{n_{k}+1} + \dots + Y_{m_{k}} \le -n_{k} \rbrace \bigcap  \lbrace Y_{m_{k}+1} + \dots + Y_{n_{k+1}} \ge m_{k} \rbrace$と定める.$\\$ $Y_{i} = -1, 1$だから$-n \le Z_{n} \le n$となることを使うと,
$\omega \in C_{k}$ に対して,$Z_{m_{k}}(\omega) = (Y_{1} + \dots + Y_{m_{k}})(\omega) = (Y_{1} + \dots + Y_{n_{k}})(\omega) + (Y_{n_{k}+1} + \dots + Y_{m_{k}})(\omega) \le n_{k} - n_{k}=0 \\ $ また$Z_{n_{k+1}}(\omega) = (Y_{1} + \dots + Y_{m_{k}})(\omega) + (Y_{m_{k}+1} + \dots + Y_{n_{k+1}})(\omega) \ge -m_{k} + m_{k} = 0$よって,$\\ \omega \in C_{k}$に対して$Z_{m_{k}}(\omega) \le 0, Z_{n_{k+1}}(\omega) \ge 0$であり, $Z_{n+1} = Z_{n} \pm 1$となることから $\\ C_{k} \subset \lbrace Z_{n} = 0 ; n_{k}+1 \le {}^\exists{n} \le n_{k+1} \rbrace = \displaystyle\bigcup_{n=n_{k}+1}^{n_{k+1}} \lbrace Z_{n}=0 \rbrace \\$
$\lbrace C_{n} \ i.o. \rbrace = \displaystyle\bigcap_{m=1}^{\infty} \displaystyle\bigcup_{k=m}^{\infty} C_{k} \subset \displaystyle\bigcap_{m=1}^{\infty} \displaystyle\bigcup_{k=m}^{\infty} \displaystyle\bigcup_{n=n_{k}+1}^{n_{k+1}} \lbrace Z_{n}=0 \rbrace \subset  \displaystyle\bigcap_{m=1}^{\infty}  \displaystyle\bigcup_{n=n_{m}+1}^{\infty} \lbrace Z_{n}=0 \rbrace = \lbrace Z_{n} =0 \ i.o. \rbrace \\$ Borel-Ccantelli Lammaから \ $\displaystyle\sum_{n=1}^{\infty} P(C_{n}) = \infty$となれば $1 = P( C_{n} \ i.o. ) \le P(Z_{n}=0 \ i.o. )$となる.つまり $\displaystyle\sum_{n=1}^{\infty} P(C_{n}) = \infty$となるような自然数列$\lbrace n_{k} \rbrace , \lbrace m_{k} \rbrace$が取れることを示せばよい.
$\\ \bullet \ {}^\forall \alpha \in (0,1), \ {}^\forall k \in \mathbb{N} $ に対して,${}^\exists \varphi (k) \ge 1 \ s.t. \ P(|Z_{\varphi (k)}| < k) \le \alpha$となる.
$\\ \quad (proof) \ {}^\forall \alpha \in (0,1), \ {}^\forall k \in \mathbb{N} , \  {}^\forall j \in \mathbb{Z}$ を固定する,$P(Z_{n}=j) \rightarrow 0 \ (n \rightarrow \infty)$であるから,$ \displaystyle\sum_{|j|<k} P(Z_{n} = j) \rightarrow 0 \ (n \rightarrow \infty)$となる.よって,$\varphi (k)$を$\displaystyle\sum_{|j|<k} P(Z_{\varphi (k)} = j) \le \alpha$となるように取れる. [証明終り]
%続き
$\\$ ${}^\forall \alpha \in (0,1), \ {}^\forall k \in \mathbb{N}$を固定する. $\lbrace \varphi (k) \rbrace_{k \in \mathbb{N}}$を上で示したものと同様にとる. $n_{k}, m_{k}$を$n_{1}=1, \\ m_{k} = n_{k} + \varphi (n_{k}), \ n_{k+1} = m_{k} + \varphi (m_{k})$とする.
%
$\\ P(C_{k}) = P(Y_{n_{k} + 1} + \dots + Y_{m_{k}} \le -n_{k}) P(Y_{m_{k} + 1} + \dots + Y_{n_{k+1}} \ge m_{k})$  ( $\because \ {\lbrace Y_{i}} \rbrace_{ i \in \mathbb{N}}$は独立)
$\\ P(Head) = \frac{1}{2}$であるから,対象性を使うと$\\ P(\left| Y_{n_{k} + 1} + \dots + Y_{m_{k}} \right| \ge n_{k}) = 2P(Y_{n_{k} + 1} + \dots + Y_{m_{k}} \le -n_{k})$
$\\ P(\left| Y_{m_{k} + 1} + \dots + Y_{n_{k+1}} \right| \ge m_{k}) = 2P(Y_{m_{k} + 1} + \dots + Y_{n_{k+1}} \ge m_{k})$となるから,
$\\ P(C_{k}) = \frac{1}{4} P(\left| Y_{n_{k} + 1} + \dots + Y_{m_{k}} \right| \ge n_{k}) P(\left| Y_{m_{k} + 1} + \dots + Y_{n_{k+1}} \right| \ge m_{k})  \ , \lbrace Y_{i} \rbrace_{ i \in \mathbb{N}}$の同一分布性から
$\\ \qquad  =\frac{1}{4} P(\left| Y_{1} + \dots + Y_{m_{k}-n_{k}} \right| \ge n_{k}) P(\left| Y_{1} + \dots + Y_{n_{k+1}-m_{k}} \right| \ge m_{k})  \ , \varphi (k)$の定め方から,
$\\ \qquad  =\frac{1}{4} P(\left| Y_{1} + \dots + Y_{\varphi(n_{k})} \right| \ge n_{k}) P(\left| Y_{1} + \dots + Y_{\varphi(m_{k})} \right| \ge m_{k}) \ge \frac{1}{4} (1-\alpha)^{2}$
$\\ \qquad \because \displaystyle\sum_{|j|<k} P(Z_{\varphi (k)} = j)$ $\underbrace{=}_{\lbrace Z_{\varphi (k)} = j \rbrace_{|j|<k} は非交和}$ $P( \displaystyle\bigcup_{|j|<k} Z_{\varphi (k)} = j) = P(\left| Y_{1} + \dots + Y_{\varphi(k)} \right| < k) \le \alpha$なので$\\ \qquad \qquad P(\left| Y_{1} + \dots + Y_{\varphi(k)} \right| \ge k) \ge 1- \alpha$
$\\$ 以上で $ \displaystyle\sum_{k=1}^{\infty} P(C_{k}) \ge  \displaystyle\sum_{k=1}^{\infty} \frac{1}{4} (1-\alpha)^{2} = \infty$となって,$\ P(Z_{n} = 0 \ i.o.) = 1$が示せた.
%$\stackrel{=}{as}$
%$\stackrel{=}{\lbrace Z_{\varphi (k)} = j \rbrace_{|j|<k} は非交和} $ 
%$11 \displaystyle{=}_{\lbrace Z_{\varphi (k)} = j \rbrace_{|j|<k} は非交和}  11$
\end{proof}

%section 独立確率変数に対する大数の法則
\section{独立確率変数に対する大数の法則}

%RANDOM SIGNS PROBLEM
\begin{thm}
$X_{1}, X_{2}, \dots$を独立確率変数とする.$\\$
このとき, \par $\displaystyle \sum_{k=1}^{n}X_{k}$ が確率収束する $\Leftrightarrow \displaystyle\sum_{k=1}^{n}X_{k}$ が概収束する$\quad$が成立する.
\end{thm}
まず、補題を示す.
%補題
\begin{lem}$N \in \mathbb{N}$を固定する. $X_{1}, X_{2}, \dots, X_{N}$を独立確率変数とし,\ $S_{n} = X_{1} + \dots + X_{n}$とおく.$\\$
${}^\forall \alpha > 0$に対して, $\displaystyle\sup_{1 \le j \le N} P(|S_{N}-S_{j}| > \alpha) = c < 1$となるとき, $\\ \displaystyle P( \sup_{1 \le j \le N} |S_{j}| > 2\alpha ) \le \frac{1}{1-c} P(|S_{N}|>\alpha)$となる.
\end{lem}
\begin{proof}
$\\$
$j^{*} (\omega)を|S_{j}(\omega)| > 2 \alpha$となる$1 \le j \le N$で一番小さいものとする.存在しないときは0とする.ここで$\displaystyle\bigcup_{1 \le j \le N} \lbrace j^{*} = j \rbrace = \emptyset$であるとき$\displaystyle P( \sup_{1 \le j \le N} |S_{j}| > 2\alpha ) = 0$なので,$\displaystyle P( \sup_{1 \le j \le N} |S_{j}| > 2\alpha ) = 0 \le \frac{1}{1-c} P(|S_{N}|>\alpha)$が成立する.よって,$\displaystyle\bigcup_{1 \le j \le N} \lbrace j^{*} = j \rbrace \neq \emptyset$のときを考える. $\\ \displaystyle P(|S_{N}| > \alpha, \ \sup_{1 \le j \le N} |S_{j} | > 2\alpha) = \sum_{j=1}^{N} P(|S_{N}| > \alpha, \ j^{*} = j) \ge \sum_{j=1}^{N} P(|S_{N} - S_{j}| \le \alpha, \ j^{*} = j) \\$
%becuase1
$\displaystyle \because \ \bullet \bigcup_{1 \le j \le N} \lbrace j^{*} = j \rbrace = \lbrace \sup_{1 \le j \le N}|S_{j}| > 2\alpha \rbrace$を示せば,一つ目の等号が成立する.\par
$(\subset)$ \par
$\omega \in$ (左辺)とすれば, $1 \le {}^\exists j \le N \ s.t. \ j^{*}(\omega) = k$だから $|S_{k}(\omega)|>2 \alpha$なので$\displaystyle\sup_{1 \le j \le N}  |S_{j}| \ge |S_{k}(\omega)| > 2\alpha$となって, $\omega \in$ (右辺) \par
$( \supset )$ \par
$\omega \in$ (右辺)とすれば, $\displaystyle\sup_{1 \le j \le N } |S_{j}(\omega)| > 2\alpha$であるから, ${}^\exists \lbrace k_{1}, k_{2}, \dots, k_{K} \rbrace \subset \lbrace 1,2,\dots ,N \rbrace \ s.t. \\ |S_{k_{m}}|>2 \alpha \  (m = 1,2,\dots, K) $となる. $\displaystyle j^{**}(\omega)= \min{ \lbrace k_{1}, k_{2}, \dots, k_{K} \rbrace }$とすれば$j^{*}(\omega) = j^{**}(\omega)$となるから,$\omega \in $(左辺)となる.
$ \\ $
%becuase 2
$\bullet $ 各k $\in \lbrace 1,2,\dots, N \rbrace$ に対して,$\ \lbrace |S_{N}| > \alpha \rbrace \cap \lbrace j^{*} = j \rbrace \supset \lbrace |S_{N}-S_{j}| \le \alpha \rbrace  \cap \lbrace j^{*} = j \rbrace$となるのを示せば2つ目の不等号が示せる.\par
k $\in \lbrace 1,2,\dots, N \rbrace$を固定しておく.$\omega \in $(右辺)をとる. $|S_{N}(\omega) - S_{j}(\omega)| \le \alpha$かつ$j^{*}(\omega) = j$であるから,
$\\$ $|S_{j}(\omega)| - |S_{N}(\omega)| \le \alpha$かつ$|S_{j}(\omega)| > 2\alpha$ $\Leftrightarrow$ $|S_{j}(\omega)| - \alpha \le |S_{N}(\omega)|$かつ$ |S_{j}(\omega)| > 2\alpha$\par $\quad\qquad\qquad\qquad\qquad\qquad\qquad\qquad\qquad \Rightarrow 2\alpha - \alpha = \alpha < |S_{N}(\omega)|$\par
以上で$ \lbrace |S_{N}| > \alpha \rbrace \cap \lbrace j^{*} = j \rbrace \supset \lbrace |S_{N}-S_{j}| \le \alpha \rbrace  \cap \lbrace j^{*} = j \rbrace $
$\displaystyle\\ \ \lbrace j^{*} = j \rbrace = ( \bigcap_{k=1}^{j-1}\lbrace |S_{k}| > 2\alpha \rbrace^{c} ) \cap \lbrace |S_{j}| > 2\alpha \rbrace$なので,$\lbrace j^{*} = j \rbrace \in \sigma (X_{1}, \dots , X_{j})$,
$\\ \lbrace |S_{N} - S_{j} | \le \alpha \rbrace \in \sigma (X_{j+1}, \dots , X_{N})$であるから, $ \lbrace j^{*} = j \rbrace $と$ \lbrace |S_{N} - S_{j} | \le \alpha \rbrace $は独立.
仮定から$ P(|S_{N}-S_{j}| > \alpha) \le c $なので$1 - P(|S_{N}-S_{j}| > \alpha) = P(|S_{N}-S_{j}| \le \alpha) \ge 1-c$
%まとめ
$\\ \displaystyle \sum_{j=1}^{N}P(|S_{N}-S_{j}| \le \alpha , \ j^{*}=j) = \sum_{j=1}^{N} P(|S_{N}-S_{j}| \le \alpha)P( j^{*}=j) \ge (1-c)\sum_{j=1}^{N} P(j^{*}=j) \\ = (1-c) P(\sup_{1 \le j \le N} |S_{j}| > 2\alpha) \\ (1-c) P(\sup_{1 \le j \le N} |S_{j}| > 2\alpha) \le \sum_{j=1}^{N}P(|S_{N}-S_{j}| \le \alpha , \ j^{*}=j)  \le P(|S_{N}| > \alpha, \ \sup_{1 \le j \le N} |S_{j} | > 2\alpha) \\ \le P(|S_{N}| > \alpha) \quad \therefore P(\sup_{1 \le j \le N} |S_{j}| > 2\alpha) \le \frac{1}{1-c} P(|S_{N}| > \alpha)$
\end{proof}
補題 10を使って,定理9の証明をする.
%定理9の証明
\begin{proof}
$\\ (\Leftarrow) \ 概収束するならば確率収束するので成立する.$
$\displaystyle \\ (\Rightarrow) \sum_{k=1}^{n}X_{k}$は確率収束するとする. ここで$\displaystyle \sum_{k=1}^{n}X_{k}$が概収束しないと仮定する.(背理法)
$\\$ここで実数列$\lbrace s_{n} \rbrace$が収束しないとすれば$\lbrace s_{n} \rbrace$はCauchy列でないので
$\\{}^\exists \varepsilon>0 \ s.t. \ {}^\forall N \in \mathbb{N} , \ {}^\forall n,m \ge N \land |s_{n}-s_{m}| > \varepsilon$であるから,
$\displaystyle \\ {}^\exists \varepsilon>0 \ s.t. \ {}^\forall m \in \mathbb{N}, \ \sup_{n>m} |s_{n}-s_{m}| > \varepsilon$となる. $\displaystyle \sum_{k=1}^{n}X_{k}$はほとんど確実にCauchy列でないから,
$\displaystyle \\ {}^\exists \varepsilon >0, \ {}^\exists \delta \in (0,1] \ s.t. \ \left[ {}^\forall m \in \mathbb{N},\ P\left(\sup_{n>m} |\sum_{k=1}^{n}X_{k} - \sum_{k=1}^{m}X_{k}| > \varepsilon \right) \ge \delta  \right]$となる.この$\varepsilon, \ \delta$を固定する.
%Cm,Nを定める.
 $\\ \displaystyle \displaystyle \sum_{k=1}^{n}X_{k}$は確率収束するので$\displaystyle \sum_{k=1}^{N} X_{k} - \sum_{k=1}^{m}X_{k} \stackrel{P}{\rightarrow} 0 $となる. 
%何故ならば
$\displaystyle \\ \because \sum_{k=1}^{n}X_{k} \stackrel{P}{\rightarrow} s$とすると, $\displaystyle P\left( \left| \sum_{k=1}^{N} X_{k} - \sum_{k=1}^{m}X_{k} \right| > \varepsilon\right) = P\left( \left| \sum_{k=1}^{N} X_{k} - s +s - \sum_{k=1}^{m}X_{k} \right| > \varepsilon\right) \le P\left( \left| \sum_{k=1}^{N} X_{k} - s\right| + \left| s - \sum_{k=1}^{m}X_{k} \right| > \varepsilon\right) \le P\left( \left\{ \left| \sum_{k=1}^{N} X_{k} - s\right| > \frac{\varepsilon}{2} \right\} \cup \left\{  \left| s - \sum_{k=1}^{m}X_{k} \right| > \frac{\varepsilon}{2} \right\} \right) \\ \le P\left( \left| \sum_{k=1}^{N} X_{k} - s\right| > \frac{\varepsilon}{2} \right) + P\left( \left| \sum_{k=1}^{m} X_{k} - s\right| > \frac{\varepsilon}{2} \right) \to 0 \ (m,\ N \to \infty)$となるから. $\\$
よって,ある$M \in \mathbb{N}$が存在して, ${}^\forall m, \ N \ge M \ (m < N)$に対して,$\displaystyle P\left( \left| \sum_{k=m+1}^{N} X_{k} \right| > \frac{\varepsilon}{2}\right) < 1$で, $\displaystyle P\left( \left| \sum_{k=m+1}^{N} X_{k} \right| > \frac{\varepsilon}{2}\right) \to 0 \ (m, \ N \to \infty)$この$m, \ N $を固定する.$\\$
$ \displaystyle C_{m, N} = \sup_{m < n \le N} P\left( \left| \sum_{k=n}^{N}X_{k} \right| > \frac{\varepsilon}{2} \right)$とおくと,$ \displaystyle \  C_{m, N} < 1$ かつ$\displaystyle C_{m, N} \to 0 \ (m, \ N \to \infty)$となる.

%
ここで補題 8を使うと,
$\displaystyle \\ P\left(\sup_{ m < n \le N } \left| \sum_{ k = m+1 }^{n} X_{k} \right| > \varepsilon \right) \le \frac{1}{1-C_{m, N}} P\left( \left| \sum_{ k = m+1 }^{N} X_{k} \right| > \frac{\varepsilon}{2} \right)$ とかけて,
まず$\displaystyle N \to \infty$とすると,$\\$
単調性から, $\displaystyle \lim_{N \to \infty} P\left(\sup_{ m < n \le N } \left| \sum_{ k = m+1 }^{n} X_{k} \right| > \varepsilon \right) =   P\left( \lim_{N \to \infty} \sup_{ m < n \le N } \left| \sum_{ k = m+1 }^{n} X_{k} \right| > \varepsilon \right)=   P\left( \sup_{ m < n} \left| \sum_{ k = m+1 }^{n} X_{k} \right| > \varepsilon \right)$ 
$\displaystyle \\ \le \lim_{N \to \infty} \frac{1}{1-C_{m, N}} P\left( \left| \sum_{ k = m+1 }^{N} X_{k} \right| > \frac{\varepsilon}{2} \right)$, $\ \displaystyle \lim_{m \to \infty} \lim_{N \to \infty} \frac{1}{1-C_{m, N}} P\left( \left| \sum_{ k = m+1 }^{N} X_{k} \right| > \frac{\varepsilon}{2} \right)$だから,
$\displaystyle \\ \lim_{m \to \infty}  P\left( \sup_{ m < n} \left| \sum_{ k = m+1 }^{n} X_{k} \right| > \varepsilon \right) = 0$
 これは$\displaystyle {}^\forall m \in \mathbb{N},\ P\left(\sup_{n>m} |\sum_{k=1}^{n}X_{k} - \sum_{k=1}^{m}X_{k}| > \varepsilon \right) \ge \delta > 0$に矛盾する. 背理法により $\displaystyle \sum_{k=1}^{n}X_{k}$は概収束することがわかった.
\end{proof}

%系
\begin{cor}
$\displaystyle \\ E[X_{k}] = 0 \ ( {}^\forall k \in \mathbb{N} ), \ \sum_{k=1}^{\infty}E[X_{k}^{2}] < \infty$とする.このとき$\displaystyle \sum_{k=1}^{n}X_{ k }$は概収束する.
\end{cor}
\begin{proof}
$\displaystyle \\ X_{1}, X_{2}, \dots$は独立なので,$\displaystyle \sum_{k=1}^{n}X_{ k }$が確率収束することを示せば定理9から$\displaystyle \sum_{k=1}^{n}X_{ k }$は概収束する.
$\displaystyle \\ \sum_{k=1}^{\infty}E[X_{k}^{2}] = s^{2}$ (ただし $s \ge 0$ )とする. ${}^\forall \varepsilon > 0$ に対してChebyshevの不等式から $\displaystyle \\ P\left( \left| \sum_{k=1}^{n}X_{ k } - s \right| > \varepsilon \right) \le \frac{1}{ \varepsilon ^ {2} } E\left[ \left| \sum_{k=1}^{n}X_{ k } - s \right|^{2} \right]$ となる.
$\displaystyle E\left[ \left| \sum_{k=1}^{n}X_{ k } - s \right|^{2} \right] \to 0 \ (n \to \infty )$を示したい. $\displaystyle \\ E\left[ \left| \sum_{k=1}^{n}X_{ k } - s \right|^{2} \right] = E\left[ \left| \sum_{k=1}^{n}X_{ k } \right|^{2} \right] -2sE\left[ \sum_{k=1}^{n}X_{ k } \right] + s^{2} \\ = E\left[ \sum_{k=1}^{n}X_{ k }^{2} + 2 \sum_{i<j}X_{ i }X_{ j }  \right]-2sE\left[ \sum_{k=1}^{n}X_{ k } \right] + s^{2} $ここで,$\displaystyle X_{1}, X_{2}, \dots$は独立だから 
$\displaystyle \\ \sum_{i<j} E\left[ X_{ i }X_{ j }  \right] = \sum_{i<j} E\left[ X_{ i } \right] E\left[X_{ j }  \right]$が成立する. また$\displaystyle E[X_{k}] = 0 \ ( {}^\forall k \in \mathbb{N} )$なので
$\displaystyle \\ = \sum_{k=1}^{n} E\left[ {X_{ k }}^{2} \right] -2s\sum_{k=1}^{n} E\left[ X_{ k } \right] + s^{2} \to s^{2} -2s^{2}+s^{2}=0 \ (n \to \infty) \\ \sum_{k=1}^{n}X_{ k }$が確率収束することがわかったので$\displaystyle \sum_{k=1}^{n}X_{ k }$は概収束する.
\end{proof}


%独立確率変数に対する大数の法則
\begin{thm}
独立確率変数に対する大数の法則$\\$
$X_{1}, X_{2}, \dots$を独立確率変数とする. $E\left[ X_{k} \right] = 0, \ E\left[ X_{k}^{2} \right] < \infty \ \left( {}^\forall k \in \mathbb{N} \right)$であるとする.
正数列 $\\ \lbrace b_{n} \rbrace_{ n \in \mathbb{N} }$が$b_{n} \uparrow \infty$かつ$\displaystyle\sum_{k=1}^{\infty} E\left[ \frac{X_{k}^{2}}{b_{k}^{2}} \right] < \infty$を満たすとき,$\displaystyle\frac{X_{1} + \dots + X_{n}}{b_{n}} \stackrel{a.s.}{\longrightarrow}  0 $が成立する.
\end{thm}
証明の前に一つ補題を示す.
\begin{lem}
Kronecker's Lemma
$\\$ $x_{1}, x_{2}, \dots$を$\displaystyle\sum_{k=1}^{n} x_{k} \to s < \infty$を満たす実数列とする.このとき, $b_{n} \uparrow \infty$となる整数列$\lbrace b_{n} \rbrace_{n \in \mathbb{N}}$が取れて, $\displaystyle\frac{1}{b_{n}} \sum_{k=1}^{n} b_{k} x_{k} \to 0 $となる.
\end{lem}
\begin{proof}
$\\$
$\displaystyle r_{n} = \sum_{k=n+1}^{\infty}x_{k}, \ r_{0} = s$とおく.このとき$x_{n} = r_{n-1} - r_{n}, \ n = 1,2, \dots$.また,$\displaystyle\sum_{k=1}^{n} b_{k}x_{k} = \\ \sum_{k=1}^{n} b_{k} (r_{k-1}-r_{k}) = \sum_{k=0}^{n-1} b_{k+1} r_{k} - \sum_{k=1}^{n} b_{k} r_{k}  = \sum_{k=1}^{n-1} (b_{k+1}-b_{k})r_{k} + b_{1}s - b_{n}r_{n}$となるから
$\\ \displaystyle\left| \sum_{k=1}^{n} b_{k}x_{k} \right| \le \sum_{k=1}^{n-1}(b_{k+1}-b_{k}) |r_{k}| + b_{1}|s| + b_{n}|r_{n}| \ $   ($ \because$ 三角不等式, $b_{n}$は単調増加なので$b_{n+1}-b_{n} \ge 0$)$\\$
ここで${}^\forall \varepsilon >0$をとる. $\displaystyle\sum_{k=1}^{\infty}x_{k}$は収束するから $r_{k}$の定め方から $N \in \mathbb{N}$を${}^\forall n \ge N$に対して, $|r_{k}| \le \varepsilon$となるように取れる.このNを固定する. $\displaystyle \tilde{r} := \max \lbrace |r_{1}|, \dots, |r_{N-1}|, \varepsilon \rbrace$とする.
$n > N$において,$\\$
$\displaystyle \sum_{k=1}^{n-1}(b_{k+1}-b_{k}) |r_{k}| \le \sum_{k=1}^{N-1}(b_{k+1}-b_{k}) |r_{k}| + \varepsilon \sum_{k=N}^{n-1}(b_{k+1}-b_{k}) \le \tilde{r} (b_{N}-b_{1}) + \varepsilon (b_{n}-b_{N}) \ よって$ $\\$
$\displaystyle\left| \frac{ \sum_{k=1}^{n} b_{k}x_{k} }{b_{n}}\right| \le \frac{1}{b_{n}} ( \tilde{r}(b_{N}-b_{1}) + \varepsilon ( b_{n} -b_{N} ) + b_{1}|s| + b_{n}\varepsilon ) \to \varepsilon $つまり$\displaystyle \varlimsup_{n \to \infty} \left| \frac{ \sum_{k=1}^{n} b_{k}x_{k} }{b_{n}}\right| \le \varepsilon$となるから$\displaystyle \lim_{n \to \infty} \left| \frac{ \sum_{k=1}^{n} b_{k}x_{k} }{b_{n}}\right| \le \varepsilon$がわかった.ここで$\varepsilon \downarrow 0$とすれば,$\displaystyle\frac{1}{b_{n}} \sum_{k=1}^{n} b_{k} x_{k} \to 0 $が示された.
\end{proof}
この補題を使って定理12を証明する.
\begin{proof}
$\\$
Kronecker's Lemmaにより, $\displaystyle\sum_{k=1}^{n}\frac{X_{k}}{b_{k}}$がほとんど確実に収束すれば,$\displaystyle \frac{1}{b_{k}} \sum_{k=1}^{n} b_{k} \frac{X_{k}}{b_{k}} = \frac{1}{b_{k}} \sum_{k=1}^{n} X_{k} \stackrel{a.s.}{\longrightarrow}  0$となる.仮定から,
$X_{1}, X_{2}, \dots$は独立確率変数,
$\displaystyle E\left[ \frac{X_{k}}{b_{k}} \right] = 0, \ \sum_{k=1}^{\infty} E\left[ \frac{X_{k}^{2}}{b_{k}^{2}} \right] < \infty$であるから,系 9から $\displaystyle\sum_{k=1}^{n}\frac{X_{k}}{b_{k}}$は概収束する. 
$\displaystyle \ \therefore \frac{X_{1} + \dots + X_{n}}{b_{n}} \stackrel{a.s.}{\longrightarrow}  0 $
\end{proof}


%section 再帰状態と格子状に分布する確率変数
\section{再帰状態と格子状に分布する確率変数}
以下では$X_{1}, X_{2}, \dots$を独立同一分布に従う確率変数列とする. $\displaystyle S_{n} = \sum_{k=1}^{n} X_{k}$と定める.
%定義 
\begin{dfn} 
再帰状態 (recurrent state)
$\\ x \in \mathbb{R}$ とする. $x$が再帰状態(recurrent state)であるとは, $x$の任意の開近傍$I$に対して $P(S_{n} \in I \ i.o.)=1$
となることである.
\end{dfn}

\begin{dfn} 
格子上に分布する確率変数 
$\\ X$が格子$\displaystyle L_{d} = \left\{ nd \ | \ n \in \mathbb{Z} \right\}, \ (d >0)$上に分布するとは, $\displaystyle \sum_{n \in \mathbb{Z}} P(X = nd)=1$かつ$\displaystyle \\ {}^\exists l >d \ s.t. \sum_{n \in \mathbb{Z}} P(X = nl)=1$とならないことである.
$\\ X$が格子上に分布しないとき, $L_{0} = \mathbb{R}$とかいて, $X$は$L_{0}$上に分布するという.
\end{dfn}

%定理3.33
\begin{thm}
$X_{1}, X_{2}, \dots$を$L_{d} \ (d \ge 0)$上に分布する確率変数列とする.
$\\$このとき$\displaystyle L_{d}$に含まれる状態は全て再帰的または全て非再帰的である.
\end{thm}
\begin{proof}
$\displaystyle \\ G = \left\{ x \in L_{d} | \ xは再帰的\right\}$
とおくと, $G$は閉集合となる.($G$が空のときは全ての状態が非再帰的なので$G \neq \emptyset$とする.)
$\displaystyle \\ \because {}^\forall \left\{ x_{n} \right\}_{n=1}^{\infty} \subset G$をとって $x_{n} \to x $とする.このとき$x \in G$を示したい.
$\displaystyle \\ x$の開近傍$I$を任意にとる.$I$に対してnを十分大きく取れば$x_{n} \in I$となる.この$n$を固定する.$I$は$x_{n}$の近傍でもあるから,$P(S_{n} \in I \ i.o.) = 1 \ \therefore x \in G$
$\displaystyle \\ y \in \mathbb{R}$が$y$の任意の近傍$I$に対して$k \in \mathbb{N}$が存在して$P(S_{k} \in I) >0$となるときyは候補状態であるとする.
$x$が再帰的かつ$y$が候補状態$\Rightarrow \ x-y$は再帰的である. 
$\displaystyle \\ {}^\forall \varepsilon >0$をとって,\ $k \in \mathbb{N}$を$P(\left| S_{k} - y \right| < \varepsilon) >0$を満たすようにとる.
$\\ x$は再帰的より$\displaystyle P(\bigcap_{m \in \mathbb{N} } \bigcup_{n > m} \left\{ \left| S_{n} - x \right| < \varepsilon \right\}) = 1$となるから,
$\displaystyle  \\
0=P(\bigcup_{m \in \mathbb{N} } \bigcap_{n > m} \left\{ \left| S_{n} - x \right| \ge \varepsilon \right\}) \ge 
P( \left| S_{k} - y \right| < \varepsilon, \ \bigcup_{m \in \mathbb{N} } \bigcap_{n > m} \left\{ \left| S_{k+n} - S_{k} - (x-y) \right| \ge 2\varepsilon \right\}) \\
= P(\left| S_{k} - y \right| < \varepsilon)P(\bigcup_{m \in \mathbb{N} } \bigcap_{n > m} \left\{ \left| S_{k+n} - S_{k} - (x-y) \right| \ge 2\varepsilon \right\})= P(\left| S_{k} - y \right| < \varepsilon)P(\bigcup_{m \in \mathbb{N} } \bigcap_{n > m} \left\{ \left| S_{n} - (x-y) \right| \ge 2\varepsilon \right\}) \\
$
$\displaystyle \because \displaystyle {}^\forall \omega \in \left\{ \left| S_{k} - y \right|  < \varepsilon \right\} \cap \bigcup_{m \in \mathbb{N} } \bigcap_{n > m} \left\{ \left| S_{k+n} - S_{k} - (x-y) \right| \ge 2\varepsilon \right\}$をとると,$\displaystyle {}^\exists m \in \mathbb{N} \ s.t. \\ {}^\forall n \ge m, \ \left| S_{k+n}(\omega) - S_{k}(\omega) -(x-y) \right| \ge 2 \varepsilon$となる.
$\displaystyle 2\varepsilon \le \left| S_{k+n}(\omega) -  x \right| + \left| S_{k}(\omega) -y \right| <  \left| S_{k+n}(\omega) -  x \right|  + \varepsilon$から
$\displaystyle \varepsilon \le \left| S_{k+n}(\omega) -  x \right| \ ({}^\forall n \ge m), \ N=k+m$とおけば, \\
$\displaystyle {}^\forall n \ge N$に対して, $ \left| S_{n}(\omega) -  x \right| \ge \varepsilon$ なので $\displaystyle \omega \in \bigcup_{m \in \mathbb{N} } \bigcap_{n > m} \left\{ \left| S_{n} - x \right| \ge \varepsilon \right\} \\ \therefore \bigcup_{m \in \mathbb{N} } \bigcap_{n > m} \left\{ \left| S_{n} - x \right| \ge \varepsilon \right\} \supset \left\{ \left| S_{k} - y \right|  < \varepsilon \right\} \cap \bigcup_{m \in \mathbb{N} } \bigcap_{n > m} \left\{ \left| S_{k+n} - S_{k} - (x-y) \right| \ge 2\varepsilon \right\}$ 
$\\$また $X_{1}, X_{2}, \dots$は独立なので,$\ S_{k}$と$\displaystyle S_{k+n} - S_{k} = \sum_{m=k+1}^{k+n} X_{m}$は独立.\ また同一分布性から
$\displaystyle \\ P(\bigcup_{m \in \mathbb{N} } \bigcap_{n > m} \left\{ \left| S_{k+n} - S_{k} - (x-y) \right| \ge 2\varepsilon \right\})= P(\bigcup_{m \in \mathbb{N} } \bigcap_{n > m} \left\{ \left| S_{n} - (x-y) \right| \ge 2\varepsilon \right\})$も成立する.
$\displaystyle \\ P(\left| S_{k} - y \right| < \varepsilon) >0$なので$P(\bigcup_{m \in \mathbb{N} } \bigcap_{n > m} \left\{ \left| S_{n} - (x-y) \right| \ge 2\varepsilon \right\}) = 0$つまり $\displaystyle \\ P(\left\{ \left| S_{n} - (x-y) \right| < 2\varepsilon \right\} \ i.o.) = 1 \quad I_{\varepsilon} = (x-y -2\varepsilon, x-y + 2\varepsilon)$とおけば,$P( \left| S_{n} \right| \in I_{\varepsilon}  \ i.o.) = 1$
$\\ \varepsilon >0$は任意だったから$x-y$は再帰的である.
$\displaystyle \\ x \in G$は候補状態である
%x in G$は候補状態.
$\because {}^\forall x \in G$をとる.\ $x$の開近傍$I$を任意にとる. $\displaystyle P(S_{n} \in I \ i.o.) = 1$であるから,${}^\forall k \in \mathbb{N}$に対して$P(S_{k} \in I )= 0$であるとすれば \ $\displaystyle \sum_{k=1}^{\infty} P(S_{k} \in I) < \infty$よりBorel-Ccantelli Lammaから$P(S_{k} \in I \ i.o.)= 0$これは$\displaystyle P(S_{n} \in I \ i.o.) = 1$に矛盾する.よって${}^\exists m \in \mathbb{N} \ s.t. \ P(S_{k} \in I ) > 0$となる.
$\\$よって\ $x-x =0 \in G$である.
このことから$G$は群である.$G$が$\mathbb{R}$上で閉なので$G$は$\mathbb{R}$上の閉部分群である. 
%ここで$\mathbb{R}$上の閉部分群はある格子$L_{l} \ (l \ge 0)$である.
%$X_{1}, X_{2},\dots$は$L_{d}$上に分布するので$S_{n} \in L_{d}$から任意の候補状態$y$は$y \in L_{d}$となる.よって
全ての候補状態$y$に対して $0-y = y \in G$となる.
%$L_{d} \subset G$となる.
%$S_{n} \in L_{d}$から $G \subset L_{d}$である.
%d>0
$\displaystyle \\ \bullet d>0$のとき\ $P(X_{1} = nd) >0$ かつ$P(X_{1}=(n+1)d)>0$となる$n \in \mathbb{Z}$が存在しないとき
$\\ 0 \in G$なので0は候補状態なので
$\displaystyle \sum_{n \in \mathbb{Z}} P(X_{1} = (2d) n)=1$となって, $d$の最大性に反する. 
よってある$n \in \mathbb{Z}$が取れて$nd, (n+1)d \in G$となる. $G$は群なので$(n+1)d - nd = d \in G$このことから$L_{d} \subset G$である.
%d=0
$\displaystyle \\ \bullet d=0$のとき\ このとき$G$に対して${}^\exists l > 0 \ s.t. \ G=\left\{ nl | n \in \mathbb{Z}  \right\}となると仮定する(背理法).$候補状態は$G$の元なので$\displaystyle \sum_{n \in \mathbb{Z}}P(X_{1}=nl)=1$となり,これは$d=0$に矛盾する.よって$G=\mathbb{R} = L_{0}$
$\\$以上で$d \ge 0$に対して$L_{d} = G$となり, $L_{d}$の全ての状態は再帰的となる.
\end{proof}

%定理3.34
\begin{thm}
$\displaystyle X_{1}, X_{2}, \dots$を$L_{d}$上に分布する確率変数列とする(ただし$d \ge 0$). 
$\\ (i)$もし, 有界区間$J \subset \mathbb{R}$が存在して $J \cap L_{d} \neq \emptyset$かつ $\displaystyle \sum_{n=1}^{\infty}P(S_{n} \in J) < \infty$を満たせば,再帰状態は存在しない.$\\ (ii)$もし, 有界区間$\displaystyle J \subset \mathbb{R}$で$0 < {}^\forall \varepsilon < \frac{\left|\left|J\right|\right|}{2}$に対して, ${}^\exists x \in \mathbb{R} \ s.t. \ I = (x-\varepsilon, x+\varepsilon) \subset J$かつ$\displaystyle \sum_{n=1}^{\infty} P(S_{n} \in I) = \infty$となるものが存在すれば, $L_{d}$の全ての状態は再帰状態である.
\end{thm}
\begin{proof}
$\\$ $(i) \ J \cap L_{d} \neq \emptyset$かつ $\displaystyle \sum_{n=1}^{\infty}P(S_{n} \in J) < \infty$を満たす有界区間$J \subset \mathbb{R}$がとれたとする.
Borel-Canteli's Lemma$\\$から$P(S_{n} \in J \ i.o.)=0$となって, $L_{d}$は少なくとも再帰的でない状態が含まれる.$\displaystyle \\ \because x \in L{d} \cap J$をとれば, $x$の開近傍$I \subset J$がとれて, $P(S_{n} \in I \ i.o.) \le P(S_{n} \in J \ i.o.) = 0$となって$x$は再帰的でない$L_{d}$の元である.
$\displaystyle \\ $定理16から$L_{d}$の元は全て再帰状態にはならない.つまり再帰状態は存在しない.
%(ii)の証明
$\displaystyle \\ \\ (ii)$長さ$l$の有界区間$\displaystyle J \subset \mathbb{R}$で$0 < {}^\forall \varepsilon < \frac{l}{2}$に対して, ${}^\exists x \in \mathbb{R} \ s.t. \ I = (x-\varepsilon, x+\varepsilon) \subset J$かつ$\displaystyle \sum_{n=1}^{\infty} P(S_{n} \in I) = \infty$となるものがとれたとする. $0 \in L_{d}$なので,0が再帰的であることがわかれば$\\$定理16から$L_{d}$の全ての状態が再帰的である.
%
$A_{k} = \begin{cases} \left\{ S_{k} \in I, \ S_{n+k} \notin I \ n=1,2,\dots \right\} & (k \geq 1) \\ \left\{ S_{n} \notin I \ n=1,2,\dots \right\} & (k = 0) \end{cases} \\$と定めると,$\displaystyle \bigcup_{m \in \mathbb{N}} \bigcap_{n>m} \left\{ S_{n} \notin I \right\} = \bigcup_{k=0}^{\infty} A_{k}$となる.
%
$\\ \displaystyle \because (\subset) {}^\forall \omega \in$ (左辺) とする. このとき${}^\exists m \in \mathbb{N}$がとれて, $S_{n}(\omega) \notin I \ ({}^\forall n > m)$となる. $1 \le i \le m-1$の中で$S_{i}(\omega) \in I$となるものが存在するときその最大値を$k$とすれば, $S_{k}(\omega) \in I, \ S_{n}(\omega) \notin I \ ({}^\forall n \ge k)$が成立する.
よって, $\omega \in A_{k}$となる. $S_{i} \notin I \ (1 \le {}^\forall i \le m-1)$のときは, $\omega \in \left\{ S_{n} \notin I \ n=1,2,\dots \right\} = A_{0}$ 以上で $\\ \displaystyle \omega \in$ (右辺)
%
$\\ (\supset)$ $ {}^\forall \omega \in$ (右辺)とする. ${}^\exists k \in \mathbb{N} \ s.t. \ \omega \in A_{k}$となる.
$\\ k \ge 1$のとき $\displaystyle \omega \in A_{k} \subset \left\{ S_{n+k} \notin n=1,2,\dots \right\} = \bigcap_{n=k+1}^{\infty} \left\{ S_{n} \notin I \right\} ,\ k=0$のとき $\displaystyle \omega \in A_{0} = \bigcap_{n=1}^{\infty} \left\{ S_{n}\notin I \right\}$
定め方から$A_{0}, A_{1}, \dots$は非交和なので, $\displaystyle P(\bigcup_{m \in \mathbb{N}} \bigcap_{n>m} \left\{ S_{n} \notin I \right\} ) = P(\bigcup_{k=0}^{\infty} A_{k}) = \sum_{k=0}^{\infty} P(A_{k})$
$\\ k \ge 1$のとき, $P(A_{k}) \ge P(S_{k} \in I, \ \left| S_{n+k} - S_{k} \right| \ge 2\varepsilon, \ n=1,2,\dots)$
$\displaystyle \\ \because \omega \in \left\{ S_{k} \in I \right\} \cap \left\{ \left| S_{n+k} - S_{k} \right| \ge 2\varepsilon, \ n=1,2,\dots \right\}$を任意にとる.
$\displaystyle \\ S_{k}(\omega)\in I$かつ$ ( S_{n+k}(\omega) - S_{k}(\omega) \le -2\varepsilon$または$2\varepsilon \le S_{n+k}(\omega) - S_{k}(\omega), \ n=1,2,\dots )$
$\\ I$の定め方から, $x-\varepsilon < S_{k} < x+\varepsilon $だから
$\\ \Rightarrow $
$\displaystyle S_{k}(\omega)\in I$かつ$ ( S_{n+k}(\omega) \le (x+\varepsilon) -2\varepsilon$または$2\varepsilon +(x-\varepsilon) \le S_{n+k}(\omega), \ n=1,2,\dots )$
$\displaystyle \\ \Leftrightarrow $
$\displaystyle S_{k}(\omega)\in I$かつ$ ( S_{n+k}(\omega) \le x -\varepsilon$または$x+\varepsilon \le S_{n+k}(\omega), \ n=1,2,\dots )$
$\displaystyle \\ \Leftrightarrow $
$\displaystyle S_{k}(\omega)\in I$かつ$ ( S_{n+k}(\omega) \notin I, \ n=1,2,\dots )$
$\Leftrightarrow \displaystyle \omega \in A_{k}$ 
$\\ \\$
$\displaystyle P(S_{k}\in I, \ \left| S_{n+k} -S_{k} \right| \ge 2\varepsilon, \ n=1,2,\dots) = P(S_{k}\in I) P( \left| S_{n+k} -S_{k} \right| \ge 2\varepsilon, \ n=1,2,\dots) \quad \because$独立性 
$\\$
$ = P(S_{k}\in I) P( \left| S_{n} \right| \ge 2\varepsilon, \ n=1,2,\dots) \quad \because$同一分布性
$\\ \displaystyle P(  \bigcup_{m \in \mathbb{N}} \bigcap_{n>m} \left\{ S_{n} \notin I \right\} ) = \sum_{k=0}^{\infty} P(A_{k}) \ge P(A_{0}) + \sum_{k=1}^{\infty} P(A_{k}) \\ \ge P(A_{0}) + P( \left| S_{n} \right| \ge 2\varepsilon, \ n=1,2,\dots) \sum_{k=1}^{\infty} P(S_{k}\in I) \quad $ここで, $\displaystyle \sum_{k=1}^{\infty} P(S_{k}\in I) = \infty$であるから
$\\ P( \left| S_{n} \right| \ge 2\varepsilon, \ n=1,2,\dots)  = 0$. 以上で $\displaystyle 0 < {}^\forall \varepsilon < \frac{l}{2} \ , P( \left| S_{n} \right| \ge 2\varepsilon, \ n=1,2,\dots)  = 0 \ $となる $(*)$
$\\ \displaystyle 0 < {}^\forall \varepsilon < \frac{l}{2}$を新しく固定する. $I = (-\varepsilon, \ \varepsilon)$として, $\left\{ A_{k} \right\}_{k=0}^{\infty}$を先と同様にとる. $I_{\delta} = (-\delta, \delta)$ (ただし, $\delta < \varepsilon$)とする. ${}^\forall k \ge 1$として $\displaystyle A_{k} = \lim_{\delta \uparrow \varepsilon} \left\{ S_{k} \in I_{\delta}, \ S_{n+k} \notin I \ n=1,2,\dots \right\}$となるから
$\\ \displaystyle P(A_{k}) = P(\lim_{\delta \uparrow \varepsilon} \left\{ S_{k} \in I_{\delta}, \ S_{n+k} \notin I \ n=1,2,\dots \right\})$ 連続性から
$\\ \displaystyle \qquad \quad  = \lim_{\delta \uparrow \varepsilon} P(\left\{ S_{k} \in I_{\delta}, \ S_{n+k} \notin I \ n=1,2,\dots \right\})$
$\\ \displaystyle P(S_{k} \in I_{\delta}, \ S_{n+k} \notin I \ n=1,2,\dots) \le P(S_{k}\in I_{\delta}, \ \left| S_{n+k} - S_{k} \right| \ge \varepsilon - \delta \ n = 1,2,\dots )$となる.
%because 
$\\ \displaystyle \because \ \omega \in \left\{ S_{k} \in I_{\delta}, \ S_{n+k} \notin I \ n=1,2,\dots \right\}$を任意のとる. $\\ \displaystyle -\delta < S_{k}(\omega) < \delta, \ S_{n+k}(\omega) \le -\varepsilon または\ \varepsilon \le S_{n+k}(\omega) \ (n =1,2,\dots)$となる.
$\\ S_{n+k}(\omega) - S_{k}(\omega) \le -\varepsilon - S_{k}(\omega) < -\varepsilon + \delta$または $\varepsilon - \delta < \varepsilon - S_{k}(\omega) \le S_{n+k}(\omega) - S_{k}(\omega)$
$\\$よって $\displaystyle \left| S_{n+k}(\omega) - S_{k}(\omega) \right| \ge \varepsilon - \delta \quad \therefore \ \omega \in \left\{ S_{k} \in I_{\delta}, \ \left| S_{n+k} - S_{k} \right| \ge \varepsilon - \delta \ n = 1,2,\dots \right\}  $
%
$\\$
独立性と同一分布性から
$\displaystyle \\ P(S_{k}\in I_{\delta}, \ \left| S_{n+k} - S_{k} \right| \ge \varepsilon - \delta \ n = 1,2,\dots ) = P(S_{k}\in I_{\delta}) P(\left| S_{n+k} - S_{k} \right| \ge \varepsilon - \delta \ n = 1,2,\dots)$ 
$\\ \qquad  = P(S_{k}\in I_{\delta}) P(\left| S_{n} \right| \ge \varepsilon - \delta \ n = 1,2,\dots) = 0$
$\\ \because \displaystyle  0 < \frac{\varepsilon - \delta}{2} < \frac{l}{2}$なので$(*)$より $P(\left| S_{n} \right| \ge \varepsilon - \delta \ n = 1,2,\dots) =0 \\$
以上で$P(A_{k})=0 \ ({}^\forall k \ge 1)$となる. 
$\\ 0< \varepsilon<\frac{l}{2}$だから$0<\frac{\varepsilon}{2} < \frac{l}{2}$なので $P(A_{0}) = P(S_{n} \notin I \ n=1,2,\dots) = P(\left| S_{n} \right| \ge \varepsilon \ n=1,2,\dots) = 0$
これまでのことから$0 < {}^\forall \varepsilon < \frac{l}{2}, \ I= (-\varepsilon, \varepsilon)$に対して, $P (S_{n} \in I \ i.o.)=1 $ 
$\\$よって,0は再帰的となる. $\therefore L_{d}$の全ての状態は再帰的である.
\end{proof}

\begin{cor}
$\\ \displaystyle \left\{ S_{n} \right\}_{n \in \mathbb{N}}$に対して, 次の$(i), \ (ii)$ のいずれかが成立する.
$\\  (i) \ L_{d} \cap I \neq \emptyset$ を満たす全ての有界区間$I$に対して, $P(S_{n} \in I \ i.o.)=1$ 
$\\ (ii) \ L_{d} \cap I \neq \emptyset$を満たす全ての有界区間$I$ に対して, $P(S_{n} \in I  \ i.o.)=0$
\end{cor}
\begin{proof}
$\\ L_{d} \cap I \neq \emptyset$となる全ての有界区間$I$に対して$\displaystyle \sum_{n=1}^{\infty} P(S_{n} \in I)= \infty$となるとき, 定理17から$L_{d}$の全ての状態は再帰的である.つまり $L_{d} \cap I \neq \emptyset$となる任意の有界区間$I$とすれば, $ {}^\forall x \in L_{d} \cap I$として 
$ P(S_{n} \in I \ i.o.) \ge P(S_{n} = x \ i.o.) =1 \quad (\because \ x$は再帰的)となる.
$\\ L_{d} \cap I \neq \emptyset$となる有界区間$I$が存在して$\displaystyle \sum_{n=1}^{\infty} P(S_{n} \in I) < \infty$となるとき, 定理17から再帰状態は存在しない. よって任意の有界区間$I \ ( L_{d} \cap I \neq \emptyset )$ に対して $I$を含む開区間$J$とすれば $P(S_{n}\in I \ i.o.) \le P(S_{n} \in J \ i.o. ) =0$ ( $\because \ J$は $L_{d}$のある元の開近傍)
\end{proof}

\begin{dfn}
$\\ \displaystyle \left\{ S_{n} \right\}_{n \in \mathbb{N}}$が系$18 \ (i)$を満たすとき, 再帰的という.また, 系$18 \ (ii)$を満たすとき, 非再帰的という.
\end{dfn}

%定理
\begin{thm}
$\displaystyle \\ E\left[ X_{1} \right] = 0$となるとき,$\displaystyle \ S_{1}, S_{2}, \dots$は再帰的となる.
\end{thm}

証明の前にまず命題を一つ示す.
%命題
\begin{prop}
$\displaystyle \\ I$が長さ$a$の区間であるとき, 
$\displaystyle \sum_{n=1}^{\infty} P \left(S_{n} \in I \right) \le 1 + \sum_{n=1}^{\infty} P \left( \left| S_{n} \right| \le a  \right)$となる.
\end{prop}
\begin{proof}
$\displaystyle \\ I$を長さ$a$の任意の区間としてとる.
$\displaystyle N := \sum_{n=1}^{\infty} 1_{ \left\{ S_{n} \in I \right\} }$とおくと, $N$は和 $S_{1}, S_{2}, \dots$の中で$I$に含まれるものを数えたものとなる.
$\displaystyle \\ n^{*} (\omega) := \begin{cases} S_{n}(\omega) \in I となる最小の n \in \mathbb{N} \\ \infty & (もしS_{n}(\omega) \notin I \ , n = 1,2,\dots ) \end{cases}$とおく. $\displaystyle \\ \left\{ n^{*} = k \right\}$ 上では $1_{ \left\{ S_{n} \in I \right\} } = 0$ $({}^\forall n < k)$ , $1_{ \left\{ S_{k} \in I \right\} } = 1$ となるから,
$\displaystyle \\ N = \sum_{n=1}^{k-1} 1_{ \left\{ S_{n} \in I \right\} } + 1_{ \left\{ S_{k} \in I \right\} } + \sum_{n=k+1}^{\infty} 1_{ \left\{ S_{n} \in I \right\} } = 1 + \sum_{n=1}^{\infty} 1_{ \left\{ S_{k+n} \in I \right\} } = 1 + \sum_{n=1}^{\infty} 1_{ \left\{ S_{k+n} - S_{k} \in I - S_{k} \right\} } \\ \le 1 + \sum_{n=1}^{\infty} 1_{ \left\{ S_{k+n} - S_{k} \in \left[-a,a\right] \right\} }$
$\displaystyle \\ \quad \because {}^\forall \omega \in \left\{ n^{*} = k \right\}$に対して, $I - S_{k}(\omega) \subset [-a, a] $となる. つまり $1_{ \left\{ I - S_{k}(\omega)  \right\} } \le 1_{ \left[ -a, a \right]}$が成立する.
$\\$なぜなら, $\overline{I} = \left[ I_{1}, I_{1}+a \right]$とかくと$(I_{1} \in \mathbb{R})$ 
$\\ -I_{1} - a \le -S_{k}(\omega) \le -I_{1}$ であって, $I - I_{1} - a = \left[-a, 0\right], \ I - I_{1}= \left[0,a\right]$であるから,
$\\I - S_{k}(\omega) \subset \left[-a, a\right]$だから, $1_{ \left\{ I - S_{k}  \right\} } \le 1_{ \left[ -a, a \right]}$となる.
$\\ \left\{ n^{*} = k \right\} \in \sigma (X_{1}, X_{2}, \dots, X_{k}) $なので, $\left\{ n^{*} = k \right\}$と$\left\{ S_{k+n} - S_{k} \in [-a,a] \right\}$は独立となる.
$\displaystyle \\ \int_{ \left\{ n^{*} = k \right\} }{N(\omega) P (d\omega)} \le \int_{ \left\{ n^{*} = k \right\} } { \left( 1 + \sum_{n=1}^{\infty} 1_{ \left\{ (S_{k+n} - S_{k})(\omega) \in \left[-a,a\right] \right\}  } \right) P (d\omega) } \ $ 単調収束定理から, 
$\displaystyle \\ = P( n^{*} = k ) + \sum_{n=1}^{\infty} \int_{ \left\{ n^{*} = k \right\} } {1_{ \left\{ S_{k+n}(\omega) - S_{k}(\omega) \in \left[-a,a\right] \right\}  } P (d\omega) } = P( n^{*} = k ) + \sum_{n=1}^{\infty} P\left( \left\{ n^{*}=k \right\} \cap  \left\{ S_{k+n} - S_{k} \in \left[-a,a\right] \right\}  \right) \\ =  P( n^{*} = k ) +  \sum_{n=1}^{\infty} P(n^{*} = k) P (S_{k+n} - S_{k} \in \left[-a,a\right]) \qquad ( \because$  独立性から$) \ $また同一分布性から 
$\displaystyle \\ = P( n^{*} = k ) + P(n^{*} = k) \sum_{n=1}^{\infty} P ( \left| S_{n} \right| \le a)$ ここで両辺$k$で和をとると, 左辺は単調収束定理を使って,
$\displaystyle \\ \sum_{k=1}^{\infty} \int_{ \left\{ n^{*} = k \right\} }{N(\omega) P (d\omega)} = \int_{ \Omega }{\sum_{n=1}^{\infty} 1_{ \left\{ S_{n}(\omega) \in I \right\} } P (d\omega)} = \sum_{n=1}^{\infty} P(S_{n} \in I )$となって,
$\displaystyle \\ \sum_{n=1}^{\infty} P(S_{n} \in I ) \le 1 + \sum_{n=1}^{\infty} P ( \left| S_{n} \right| \le a)$が成立する.
\end{proof}
 
 命題21を使って, 定理20を証明する.
 \begin{proof}
 $\displaystyle \\ {}^\forall M \in \mathbb{N}$をとる.
 \end{proof}
 
 \end{document}
\documentclass{jsarticle}
\usepackage{amsthm}
\usepackage{amsmath}
\usepackage{amssymb}
\usepackage{eucal}
\usepackage{amsfonts}
\usepackage{mathrsfs}  


\newtheorem{thm}{定理}
\newtheorem{dfn}[thm]{定義}
\newtheorem{prop}[thm]{命題}
\newtheorem{lem}[thm]{補題}
\newtheorem{cor}[thm]{系}


\begin{document}


\title{課題研究 レポート}
\author{M$0$T$0$}
\date{}
\maketitle

%ボレルカンテリの補題
%\begin{thm}

%Borel-Cantelli Lamma 

%\end{thm}

%\begin{proof}

%\end{proof}

%section 末尾事象とinfinte often
以下では$(\Omega, \ \mathcal{F}, \ P)を確率空間とする.$

\section{末尾事象とinfinte often}
\begin{dfn} 
末尾事象 (tail event)
$\\X_{1}, X_{2}, \dots$を確率変数列とする.
$\\E \in \sigma (X_{1}, X_{2}, \dots)$が末尾事象(tail event)であるとは
$\\$任意の$n \in \mathbb{N}$に対して, $E \in \sigma (X_{n}, X_{n+1},\dots)$が成立することである.
$ \\$末尾加法族(tail  $\sigma$-field) $\delta$を
$\displaystyle \delta = \bigcap_{n \in \mathbb{N}} \sigma (X_{n}, X_{n+1}, \dots)$と定める.
$\\$ 定義から末尾加法族の元 $E \in \delta$ は末尾事象となる.
\end{dfn}

\begin{prop}
$X_{1}, X_{2}\dots$を確率変数とする.
このとき$\textbf{X} = (X_{1}, X_{2}, \dots)$とすると, 
$\\$任意の$B \in \mathfrak{B}_{\infty}$に対して, $\left\{ \textbf{X} \in B \right\} \in \mathcal{F}$となる.
\end{prop}
\begin{proof}
$$\displaystyle E = \bigcup_{n\in \mathbb{N}} \left\{ B_{1} \times \dots \times B_{n} \times \mathbb{R} \times \mathbb{R} \times \dots \subset \mathbb{R}^{\infty} | \, B_{1}, B_{2}, \dots , B_{n} \in \mathfrak{B}_{1} \right\}$$ と定める.
$\\ S \in E$を任意に固定する.このとき$\, n \in \mathbb{N}$と $ B_{1} , \dots , B_{n} \in \mathfrak{B}_{1}$がとれて, $S = B_{1} \times \dots \times B_{n} \times \mathbb{R} \times \dots$ となる.
各$k=1,2,\dots, n$について $\left\{ X_{k} \in B_{k}  \right\} \in \mathcal{F}$なので $\displaystyle \left\{ \textbf{X} \in S \right\} = \bigcap_{k=1}^n \left\{ X_{k} \in B_{k}  \right\} \in \mathcal{F}$ が成立する.
$\\$ここで$$\displaystyle \mathcal{C} = \left\{ C \in \mathfrak{B}_{\infty} | \,  \left\{ \textbf{X} \in C \right\} \in \mathcal{F} \right\}$$とすると、$\mathcal{C}$は$\sigma$加法族になる.
$\\ \because$
$\\ (i) \left\{ \textbf{X} \in \mathbb{R}^{\infty} \right\} = \Omega \in \mathcal{F}$, $\left\{ \textbf{X} \in \emptyset \right\} = \emptyset \in \mathcal{F}$つまり$\Omega , \, \emptyset \in \mathcal{C}$
$\\ (ii) \, C \in \mathcal{C}$とする.このとき $ \left\{ \textbf{X} \in C^{c} \right\} = \left\{ \textbf{X} \in C \right\}^{c} \in \mathcal{F} \,$となるから $C^{c} \in \mathcal{C}$
$\displaystyle\\ (iii) \, \left\{ C_{k} \right\}_{k \in \mathbb{N}} $を任意にとる. $\displaystyle \left\{ \textbf{X} \in \bigcup_{k \in \mathbb{N}} C_{k} \right\} =  \bigcup_{k \in \mathbb{N}} \left\{ \textbf{X} \in C_{k} \right\} \in \mathcal{F}$となって, $\displaystyle \bigcup_{k \in \mathbb{N}} C_{k} \in \mathcal{F}$がわかる.
$\\$
以上で $\mathcal{C}$は$E$を含む$\sigma$加法族であることがわかった. $\sigma (E) = \mathfrak{B}_{\infty}$なので, $\mathcal{C} = \mathfrak{B}_{\infty}$ である.
$\\$つまり, 任意の$B \in \mathfrak{B}_{\infty}$に対して, $ B \in \mathcal{C}$ だから $\left\{ \textbf{X} \in B \right\} \in \mathcal{F}$が成立する.
\end{proof}


%近似定理
\begin{thm}
近似定理 
$\\ X_{1}, X_{2}, \dots$ を確率変数列とする.
$\\$任意の$\displaystyle A_{1} \in \sigma (\textbf{X}), \,  \varepsilon > 0$ に対して, ある$n \in \mathbb{N}, A_{2} \in \mathcal{F} (X_{1}, X_{2}, \dots , X_{n})$が存在して$P(A_{1} \triangle A_{2}) \le 0$ となる. (ただし $A \triangle B := (A-B) \cup (B-A)$ )
\end{thm}

\begin{proof}
$\\$任意の$ A_{1} \in \sigma (\textbf{X})$,  $\varepsilon > 0$を固定する. 
$$\displaystyle \\ \mathscr{F}_{0} = \displaystyle\bigcup_{n \in  \mathbb{N}} \sigma (X_{1}, X_{2}, \dots, X_{n}) $$
$$\mathcal{C} = {\lbrace A \in \mathcal{F} | \ 任意の\, \varepsilon > 0 に対して,P(A \triangle B) \le \varepsilon \,となるB \in \mathscr{F}_{0} \, が存在する\  \rbrace  }$$
と定める. 
$\\$任意の$A \in \mathscr{F}_{0}$をとると, ある$n \in \mathbb{N}, \, B \in \mathfrak{B}_{n}$が取れて, $A = \left\{ (X_{1}, \dots, X_{n}) \in B \right\}$とかける.
$\\$このとき $X_{1}, X_{2}\dots$は確率変数なので$A \in \mathcal{F}$となる.
$\\$また定め方から$A\triangle A = \emptyset$だから$P(A\triangle A)=0$である.
よって, $A \in \mathcal{C}$となる.
つまり$\mathscr{F}_{0}  \subset \mathcal{C}$が成立する.
$\\$ここで $\mathcal{C}$が$\sigma$加法族であることを示せば, $\sigma (\mathscr{F}_{0}) \subset \mathcal{C}$で, $\sigma (\mathscr{F}_{0}) = \sigma (\textbf {X})$であることから,
$\displaystyle\\ A_{1} \in \sigma (\textbf{X}) \subset \mathcal{C}$なので, $P(A_{1} \triangle A_{2}) \le \varepsilon$となるような$A_{2} \in \mathscr{F}_{0}$が存在することを示せる.
%%証明
$\displaystyle \\ \bullet \ \mathcal{C}$が$\sigma$加法族であること示す.\par
$\quad$ $(i)$ $\Omega \in \mathcal{C} \ (\because \ \Omega \in \ \mathscr{F}_{0}) $\par

$\quad$ $(ii)$ $A \in \mathcal{C}とすると, A^c \in \mathcal{C}$\par
$\qquad$ $\because  \varepsilon > 0$ を任意に固定する.このとき$B \in \mathcal{F}_{0}$が取れて, $P(A \triangle B) \le \varepsilon $となる. $\mathcal{F}_{0}$の定め方から,$\\ \qquad \qquad B^c \in \mathcal{F}_{0}$ であって, $P(A^c \triangle B^c) = P((A^c \cap B) \cup (A \cap B^c)) = P((B-A) \cup (A-B)) = P(A \triangle B) \le \varepsilon$\par $\quad \qquad$ $\therefore A^c \in \mathcal{C}$

$\quad$ $(iii)$ $ \lbrace A_{n} \rbrace_{n \in \mathbb{N}} \subset \mathcal{C},  \varepsilon > 0$を任意にとる. $\lbrace B_{n} \rbrace_{n \in \mathbb{N}} \subset \mathcal{F}_{0}$を$P(A_{n} \triangle B_{n}) \le \frac{\varepsilon}{2^{n+1}} $となるようにとる.\par 
$\quad \qquad$また,測度の上からの連続性からある$N \in \mathbb{N}$が取れて, $P( \displaystyle\bigcup_{n=N+1}^{\infty} A_{n}) \le \frac{\varepsilon}{2}$となる.\par$\quad \qquad$ ここで, $\displaystyle\bigcup_{n=1}^{N} B_{n} \triangle \displaystyle\bigcup_{n=1}^{\infty} A_{n} \subset (\displaystyle\bigcup_{n=1}^{N} B_{n} \triangle A_{n}) \cup \displaystyle\bigcup_{n=N+1}^{\infty} A_{n}$を示せれば,単調性と劣加法性から,\par $\quad \qquad$ $P(\displaystyle\bigcup_{n=1}^{N} B_{n} \triangle \displaystyle\bigcup_{n=1}^{\infty} A_{n}) \le P((\displaystyle\bigcup_{n=1}^{N} B_{n} \triangle A_{n}) \cup \displaystyle\bigcup_{n=N+1}^{\infty} A_{n}) \le  \displaystyle\sum_{n=1}^{N}P(B_{n} \triangle A_{n}) + P(\displaystyle\bigcup_{n=N+1}^{\infty} A_{n})$ \par $\quad \qquad$ $\le \frac{\varepsilon}{2} + \frac{\varepsilon}{2} = \varepsilon $となる. $\displaystyle\bigcup_{n=1}^{N} B_{n} \in \mathcal{F}_{0}$であることから $\displaystyle\bigcup_{n=1}^{\infty} A_{n} \in \mathcal{C}$となる.

%%証明の中の証明
$\displaystyle \\ \quad \bullet \ \displaystyle\bigcup_{n=1}^{N} B_{n} \triangle \displaystyle\bigcup_{n=1}^{\infty} A_{n} \subset (\displaystyle\bigcup_{n=1}^{N} B_{n} \triangle A_{n}) \cup \displaystyle\bigcup_{n=N+1}^{\infty} A_{n}$を示す. \par
$\quad \qquad$ $\because \ \omega \in \displaystyle\bigcup_{n=1}^{N} B_{n} \triangle \displaystyle\bigcup_{n=1}^{\infty} A_{n} \Leftrightarrow (\omega \in \displaystyle\bigcup_{n=1}^{N} B_{n} - \displaystyle\bigcup_{n=1}^{\infty} A_{n}) \lor  (\omega \in \displaystyle\bigcup_{n=1}^{\infty} A_{n} - \displaystyle\bigcup_{n=1}^{N} B_{n})$ \par 
$\qquad \qquad \qquad$
$ \Leftrightarrow (\omega \in \displaystyle\bigcup_{n=1}^{N} B_{n} \cap \displaystyle\bigcap_{n=1}^{\infty} A_{n}^{c}) \lor 
 (\omega \in ( \displaystyle\bigcup_{n=1}^{N} A_{n} \cap \displaystyle\bigcap_{n=1}^{N} B_{n}^{c}) \cup (\displaystyle\bigcup_{n=N+1}^{\infty} A_{n} \cap \displaystyle\bigcap_{n=1}^{N} B_{n}^{c}) ) $\par 
$\qquad \qquad \qquad$
$\Rightarrow (\omega \in \displaystyle\bigcup_{n=1}^{N} B_{n} \cap \displaystyle\bigcap_{n=1}^{N} A_{n}^{c}) \lor (\omega \in ( \displaystyle\bigcup_{n=1}^{N} A_{n} \cap \displaystyle\bigcap_{n=1}^{N} B_{n}^{c}) \cup \displaystyle\bigcup_{n=N+1}^{\infty} A_{n} )$\par 
$\qquad \qquad \qquad$
$\Rightarrow \omega \in \displaystyle\bigcup_{n=1}^{N} (B_{n} \cap  A_{n}^{c}) \lor (\omega \in ( \displaystyle\bigcup_{n=1}^{N} A_{n} \cap B_{n}^{c}) \cup \displaystyle\bigcup_{n=N+1}^{\infty} A_{n} )$
\par 
$\qquad \qquad \qquad$
$\Leftrightarrow \omega \in \displaystyle\bigcup_{n=1}^{N} ( (B_{n} \cap  A_{n}^{c}) \cup (A_{n} \cap  B_{n}^{c}) ) \lor \omega \in \displaystyle\bigcup_{n=N+1}^{\infty} A_{n}$
\par 
$\qquad \qquad \qquad$
$\Leftrightarrow  \omega \in \displaystyle\bigcup_{n=1}^{N} ( B_{n} \triangle A_{n}) \lor \omega \in \displaystyle\bigcup_{n=N+1}^{\infty} A_{n}$
\par 
$\qquad \qquad \qquad$
$\Leftrightarrow \omega \in \displaystyle\bigcup_{n=1}^{N} ( B_{n} \triangle A_{n}) \cup \displaystyle\bigcup_{n=N+1}^{\infty} A_{n}$
$\qquad \quad$ $\therefore \displaystyle\bigcup_{n=1}^{N} B_{n} \triangle \displaystyle\bigcup_{n=1}^{\infty} A_{n} \subset (\displaystyle\bigcup_{n=1}^{N} B_{n} \triangle A_{n}) \cup \displaystyle\bigcup_{n=N+1}^{\infty} A_{n}$ \par
$\quad$$(i) \sim (iii)$より$\mathcal{C}$は$\sigma$加法族である.



\end{proof}

%ゼロワン法則
\begin{thm}
Kolmogorov zero-one law
\end{thm}

$X_{1},X_{2}, \dots$ を独立な確率変数とする.この時, $E\in\delta$ であるとすれば$P(E)$は$0$, $1$のいずれかの値をとる.

\begin{proof}
$\\  E \in \delta$ を任意とる. $E \in \sigma (\textbf {X})$ であるから,定理3により各$n \in \mathbb{N}$ に対して,
ある$E_{n} \in \sigma (X_{1}, X_{2}, \dots , X_{n})$ が取れて$P(E \triangle E_{n}) \to 0$ となる.
このことから$P(E_{n}) \to P(E), \, P(E_{n} \cup E) \to P(E)$がわかる.
$\\ \because $ 
$\\ \bullet P(E_{n}) \to P(E)\\$ $$
P(E_{n}) \le P((E_{n}-E) \cup E ) \le P(E_{n}-E) + P(E)$$から$$P(E_{n})-P(E) \le P(E_{n}-E) \le P(E_{n} \triangle E) \to 0 \ (n \to \infty)$$ 同様にして 
$$P(E)-P(E_{n}) \le P(E-E_{n}) \le P(E_{n} \triangle E) \to 0$$がわかる. 
$\\ \bullet P(E_{n} \cup E) \to P(E)$
$$\\ P(E \cup E_{n}) \le P((E_{n}-E) \cup E) \le P(E_{n}-E) + P(E) \le P(E_{n} \triangle E) + P(E)$$から
$$P(E \cup E_{n})-P(E) \le P(E_{n} \triangle E) \to 0 \ (n \to \infty)$$
また, $E \subset (E \cup E_{n}) \cup (E \triangle E_{n})$だから$$P(E)-P(E \cup E_{n}) \le P(E \triangle E_{n}) \to 0 \ (n \to \infty)$$
この時, $E \in \delta$ だから, $E \in \sigma (X_{n+1}, X_{n+2}, \dots)$ である.つまり, $E$ と$E_{n}$ は独立である.
$\displaystyle \\ P(E \cap E_{n} ) = P(E)P( E_{n} )$ だから,各辺で$n \to \infty$ とすれば, $P(E) = P(E)^{2}$. つまり $P(E)=0,1$ となることがわかった.

\end{proof}

%infinite often
\begin{dfn} 
infinite often
$\\ \left\{ A_{n} \right\}_{n \in \mathbb{N}} \subset \mathscr{F}$とする. $\left\{ A_{n} \ i.o. \right\}$を
 $\displaystyle \left\{ A_{n} \ i.o. \right\} = \lim_{m \to \infty} \bigcup_{n>m} A_{n}$と定める.
$\\ \quad$
 $\\  \left\{ A_{n} \ i.o. \right\}$は $\left\{ \omega \ | \ \omega \in A_{n} となるnが無限個存在する. \right\}$ともかける.
 $\\ \displaystyle \because \omega \in \lim_{m \to \infty} \bigcup_{n>m} A_{n}$を任意にとる. $1$に対して$\omega \in A_{n_{1}}$となる$n_{1} \in \mathbb{N}$をとる.続いて $n_{1}$に対して$\omega \in A_{n_{2}}$となる$n_{2} \in \mathbb{N}$をとる.これを続ければ $\omega \in A_{n}$となる$n$の列として$\left\{ n_{k} \right\}_{k \in \mathbb{N}}$が取れるので, $\omega \in \left\{ \omega ' \ | \ \omega ' \in A_{n} となるnが無限個存在する. \right\}$である.
 $\\ \omega \in \left\{ \omega \ | \ \omega \in A_{n} となるnが無限個存在する. \right\}$を任意にとる.このとき任意の$m \in \mathbb{N}$に対して
 $\\ \omega \in A_{n}$ となる$n>m$が無限個存在する.よって $\displaystyle \omega \in \lim_{m \to \infty} \bigcup_{n>m} A_{n}$
 
\end{dfn}


%ボレルカンテリの補題
%\{, \lbrace	\}, \rbrace
\begin{lem}
Borel-Cantelli Lamma 
\end{lem}
$\displaystyle (\textbf{I}), \ {\lbrace A_{n} \rbrace }_{n \in \mathbb{N}} \in \mathcal{F}$ について, $\displaystyle \sum_{n=1}^{\infty} P(A_{n}) < \infty $ ならば, $\displaystyle P(A_{n} \ i.o.) = 0$が成立する.
 \par
$\displaystyle (\textbf{II}), \ {\lbrace A_{n} \rbrace }_{n \in \mathbb{N}} \in \mathcal{F}$ について, $\displaystyle {\lbrace A_{n} \rbrace }_{n \in \mathbb{N}}$ が独立かつ $\displaystyle \sum_{n=1}^{\infty}P(A_{n}) = \infty$ ならば, $\displaystyle P(A_{n}  \ i.o.) = 1$ が成立する.
\begin{proof}
$\\ (\textbf{I}) $ 
$$\\ P(A_{n} \ i.o.) = P(\displaystyle\lim_{m \to \infty} \displaystyle\bigcup_{n=m}^{\infty}{A_n}) = \displaystyle\lim_{m \to \infty} P(\displaystyle\bigcup_{n=m}^{\infty}{A_n}) \le \displaystyle\lim_{m \to \infty} (\sum_{n=m}^{\infty}P(A_{n}))$$ $(\because$ 二つ目の等号は測度の連続性,不等号には劣加法性を使った)
$\displaystyle \\ \sum_{n=1}^{\infty} P(A_{n}) < \infty$ であるから$\displaystyle\lim_{m \to \infty} (\sum_{n=m}^{\infty}P(A_{n})) = 0 \ \therefore P(A_{n} \ i.o.) = 0
\\(\textbf{II})$
$\\$任意の$ m \in \mathbb{N}$に対して, $\displaystyle P(\bigcap_{n=m}^{\infty}{A_n}^{c})=0$ を示せば
$$\displaystyle \\ P({(A_{n} \ i.o.)}^{c}) = P(\bigcup_{m \in \mathbb{N}}\bigcap_{n=m}^{\infty}{A_n}^{c}) = \lim_{m \to \infty} P(\bigcap_{n=m}^{\infty}{A_n}^{c}) = 0$$
つまり$\displaystyle P(A_{n} \ i.o.)=1$ がわかる.
$\\$任意の$m \in \mathbb{N}$を固定する. $\displaystyle \left\{ A_{n}  \right\}_{n \in \mathbb{N}}$は独立なので
$$\displaystyle P(\bigcap_{n=m}^{\infty}{A_n}^{c}) = \prod_{n=m}^{\infty} P({A_n}^{c}) = \prod_{n=m}^{\infty} (1-P(A_{n}))$$
である.ここで$\log{(1-x)} \le -x \ (0 \le x \le 1)$を使うと
 $$\displaystyle \log{(\prod_{n=m}^{\infty}(1-P(A_{n})))} = \sum_{n=m}^{\infty}\log{(1-P(A_{n}))} \le - \sum_{n=m}^{\infty}P(A_{n}) = -\infty $$
よって$\displaystyle P(\bigcap_{n=m}^{\infty}{A_n}^{c}) = 0$ が示せた.
\end{proof}

$\\$
いくつか応用例を挙げる.$\\$
(例1)コイントスを考える. \textbf {s} を長さ$k$のH, T (表, 裏)が要素の列とする.
$A_{n} = \left\{ \omega \ ; (\omega_{n},\dots,\omega_{n+k-1}) = \textbf {s}  \right\}$と定める.

%応用例1
\begin{prop}
$P(A_{n} \ i.o.) = 1$
\end{prop}
\begin{proof}
$\\ B_{1} = \left\{ \omega \ ; (\omega_{1},\dots,\omega_{k}) = \textbf {s}  \right\} , B_{2} = \left\{ \omega \ ; (\omega_{k+1},\dots,\omega_{2k}) = \textbf {s}  \right\} , \ \dots$とおく. 
$\\$このとき, $\left\{ B_{n} \right\} _{n \in \mathbb{N}} $は独立となる. 
$\\ \because $任意に$i, j \in \mathbb{N}, \, (i \neq j)$をとる. $B_{i}, B_{j}$が独立であることを示したい.
$\\$ 考えている確率はコイントスなので, $\textbf{s}=(s_{1}, \dots ,s_{k})$とかくと,
$$\displaystyle P(B_{i} \cap B_{j}) = P((\omega_{k(i-1)+1},\dots,\omega_{ki}) = \textbf {s}, (\omega_{k(j-1)+1},\dots,\omega_{kj}) = \textbf {s})$$
$$= \prod_{m=1}^{k} P(\omega_{k(i-1)+m}=s_{m})\prod_{l=1}^{k} P(\omega_{k(j-1)+l}=s_{l}) = P((\omega_{k(i-1)+1},\dots,\omega_{ki}) = \textbf {s})P((\omega_{k(j-1)+1},\dots,\omega_{kj}) = \textbf {s})$$
$$=P(B_{i})P(B_{l})$$
$\\$また,$\left\{ B_{n} \ i.o. \right\} \subset \left\{ A_{n} \ i.o.\right\}$である. $( \because \ B_{l} = A_{(l-1)k + 1})$ また, $\displaystyle P(B_{n}) = P(B_{1}) = \frac{1}{2^{k}} > 0$ なので 
$\displaystyle \sum_{n=1}^{\infty}P(B_{n}) = \infty$.以上のことから補題6$(\textbf{II})$を使うと, $P(B_{n} \ i.o.) = 1 \le P(A_{n} \ i.o.) \ \therefore P(A_{n} \ i.o.)=1$
\end{proof}

$\\$
(例2)再び,コイントスを考える. $
Y_{i}(\omega)= \left \{
\begin{array}{ll}
1 & (\omega_{i}がHのとき) \\
-1 & (\omega_{i}がTのとき)
\end{array}
\right. , Z_{n} = Y_{1} + \dots + Y_{n}$と定める. %$A_{n} = {Z_{n}=0}$とする.
\begin{prop}
$P(Head) \ne \frac{1}{2}$とする. このとき$P(Z_{n} = 0 \ i.o.) = 0$となる.
\end{prop}
\begin{proof}
$\\$
$P(Head) = p$とおく.
$\\ \displaystyle\sum_{n=1}^{\infty}P(Z_{n} = 0) < \infty$であることが示せれば, 補題6($\textbf{I}$)から$P(Z_{n} = 0 \ i.o.) = 0$がわかる.
$\\$Stirlingの近似公式から,十分大きい$n$に対して, $\displaystyle ^{}_{2n}C_{n} = 2^{2n} \frac{1+\delta_{n}}{\sqrt{\pi n}}$(ただし$\displaystyle \delta_{n} \downarrow 0$)であり, また,
$\displaystyle p \neq \frac{1}{2}$なので$2^{2}p(1-p)< 1$より,ある$0 < \lambda < 1$ が存在して$\displaystyle 2^{2}p(1-p) < \frac{1}{\lambda} 2^{2}p(1-p) < 1$ となる.
$\\ \displaystyle \delta_{n} \downarrow 0$だから十分大きい$n$に対して は$\displaystyle \delta_{n} < \frac{\lambda}{2^{2}p(1-p)} -1$が成立する.
$\\$
ここで$N \in \mathbb{N}$ を $n \ge N $で 
$$\displaystyle P(Z_{2n})=^{}_{2n}C_{n} \ p^{n} (1-p)^{n} = 2^{2n} \frac{1+\delta_{n}}{\sqrt{\pi n}} \ p^{n} (1-p)^{n}$$
かつ
$\displaystyle\delta_{n} < \frac{\lambda}{2^{2}p(1-p)} -1$
を満たすようにとる.
$\\ \displaystyle a_{n} = 2^{2n} \frac{1+\delta_{n}}{\sqrt{\pi n}} \ p^{n} (1-p)^{n}$とおく.\ $n \ge N$ において
$$ \frac{a_{n+1}}{a_{n}} =  2^{2} \frac{1+\delta_{n+1}}{1+\delta_{n}} \sqrt{\frac{n}{n+1}} p(1-p) \le 2^{2} \sqrt{\frac{n}{n+1}}\frac{\lambda}{2^{2}p(1-p)} p(1-p)= \lambda \sqrt{\frac{n}{n+1}} \le \lambda$$
だから $\displaystyle a_{n+1} \le (1-\lambda)a_{n} \le \dots \le (1-\lambda)^{n+1-N}a_{N}$が成立する.
$$\sum_{n=1}^{\infty}P(Z_{2n} = 0) = \sum_{n=1}^{N}P(Z_{2n} = 0) + \sum_{n=N+1}^{\infty}P(Z_{2n} = 0) \le \sum_{n=1}^{N}P(Z_{2n} = 0) + \sum_{n=N+1}^{\infty} \lambda^{n-N} a_{N}$$
$\\$
$$ \le \sum_{n=1}^{N}P(Z_{2n} = 0) + a_{N} \sum_{n=1}^{\infty} \lambda^{n} = \sum_{n=1}^{N}P(Z_{2n} = 0) + a_{N} \frac{\lambda}{1-\lambda} < \infty \  (\because \ 0 < \lambda < 1)$$
以上で$\displaystyle\sum_{n=1}^{\infty}P(Z_{n} = 0) = \displaystyle\sum_{n=1}^{\infty}P(Z_{2n} = 0) < \infty$がわかった.
\end{proof}
$\\$

\begin{thm}
$\displaystyle P(Head) = \frac{1}{2}$とする. このとき$P(Z_{n} = 0 \ i.o.) = 1$となる.
\end{thm}
\begin{proof}
$\\$
$n_{1} < n_{2} < \dots$の自然数列とする. また, 各$k \in \mathbb{N}$に対して,$n_{k} < m_{k} < n_{k+1}$となるように$m_{1} < m_{2} < \dots$をとる.
$$C_{k} = \lbrace Y_{n_{k}+1} + \dots + Y_{m_{k}} \le -n_{k} \rbrace \bigcap  \lbrace Y_{m_{k}+1} + \dots + Y_{n_{k+1}} \ge m_{k} \rbrace$$
と定める.
$\\$ 
$Y_{i} = -1, 1$だから$-n \le Z_{n} \le n$となることを使うと,
任意の $\omega \in C_{k}$ に対して,
$$Z_{m_{k}}(\omega) = (Y_{1} + \dots + Y_{m_{k}})(\omega) = (Y_{1} + \dots + Y_{n_{k}})(\omega) + (Y_{n_{k}+1} + \dots + Y_{m_{k}})(\omega) \le n_{k} - n_{k}=0 $$
また
$$Z_{n_{k+1}}(\omega) = (Y_{1} + \dots + Y_{m_{k}})(\omega) + (Y_{m_{k}+1} + \dots + Y_{n_{k+1}})(\omega) \ge -m_{k} + m_{k} = 0$$
つまり任意の $\omega \in C_{k}$に対して 
$Z_{m_{k}}(\omega) \le 0, \, Z_{n_{k+1}}(\omega) \ge 0$が成立する. $Z_{n+1} = Z_{n} \pm 1$となることから 
$$ C_{k} \subset \left\{ \omega \in \Omega ; Z_{n}(\omega) = 0 となるn \in \mathbb{N} が少なくとも一つ存在する. (ただし n_{k}+1 \le n \le n_{k+1} ) \right\}$$
$$= \bigcup_{n=n_{k}+1}^{n_{k+1}} \lbrace Z_{n}=0 \rbrace$$
よって
$$\lbrace C_{n} \ i.o. \rbrace = \bigcap_{m=1}^{\infty} \displaystyle\bigcup_{k=m}^{\infty} C_{k} \subset \displaystyle\bigcap_{m=1}^{\infty} \displaystyle\bigcup_{k=m}^{\infty} \bigcup_{n=n_{k}+1}^{n_{k+1}} \lbrace Z_{n}=0 \rbrace \subset  \bigcap_{m=1}^{\infty}  \bigcup_{n=n_{m}+1}^{\infty} \lbrace Z_{n}=0 \rbrace = \lbrace Z_{n} =0 \ i.o. \rbrace $$
Borel-Cantelli Lammaから \ $\displaystyle\sum_{n=1}^{\infty} P(C_{n}) = \infty$となれば $1 = P( C_{n} \ i.o. ) \le P(Z_{n}=0 \ i.o. )$となる.
$\\$つまり $\displaystyle\sum_{n=1}^{\infty} P(C_{n}) = \infty$となるような自然数列$\lbrace n_{k} \rbrace , \lbrace m_{k} \rbrace$が取れることを示せばよい.
$\\ \bullet$ 任意の$\alpha \in (0,1), \, k \in \mathbb{N} $ に対して, $P(|Z_{\varphi (k)}| < k) \le \alpha$となる $\varphi (k) \ge 1$がとれる.
$\\(proof)$ 任意の $\alpha \in (0,1), \ k \in \mathbb{N},\ j \in \mathbb{Z}$ を固定する,
$P(Z_{n}=j) \rightarrow 0 \ (n \rightarrow \infty)$であるから,
$\\ \displaystyle\sum_{|j|<k} P(Z_{n} = j) \rightarrow 0 \ (n \rightarrow \infty)$となる.よって,$\varphi (k)$を$\displaystyle\sum_{|j|<k} P(Z_{\varphi (k)} = j) \le \alpha$となるように取れる. $\\$[証明終り]
%続き
$\\$任意に $\alpha \in (0,1), \ k \in \mathbb{N}$を固定する. $\lbrace \varphi (k) \rbrace_{k \in \mathbb{N}}$を上で示したものと同様にとる. 
$\\ n_{k}, m_{k}$を$n_{1}=1, m_{k} = n_{k} + \varphi (n_{k}), \ n_{k+1} = m_{k} + \varphi (m_{k})$とする.
%
$\\ P(C_{k}) = P(Y_{n_{k} + 1} + \dots + Y_{m_{k}} \le -n_{k}) P(Y_{m_{k} + 1} + \dots + Y_{n_{k+1}} \ge m_{k})$  ( $\because \ {\lbrace Y_{i}} \rbrace_{ i \in \mathbb{N}}$は独立)
$\\ P(Head) = \frac{1}{2}$であるから,対象性を使うと
$$ P(\left| Y_{n_{k} + 1} + \dots + Y_{m_{k}} \right| \ge n_{k}) = P(Y_{n_{k} + 1} + \dots + Y_{m_{k}} \ge n_{k} ) + P( Y_{n_{k} + 1} + \dots + Y_{m_{k}} \le -n_{k} ) $$ 
$$ = P(Y_{n_{k} + 1} + \dots + Y_{m_{k}} \le -n_{k} ) + P( Y_{n_{k} + 1} + \dots + Y_{m_{k}} \le -n_{k} ) = 2P(Y_{n_{k} + 1} + \dots + Y_{m_{k}} \le -n_{k})$$
同様にして
$ P(\left| Y_{m_{k} + 1} + \dots + Y_{n_{k+1}} \right| \ge m_{k}) = 2P(Y_{m_{k} + 1} + \dots + Y_{n_{k+1}} \ge m_{k})$となるから
$$ P(C_{k}) = \frac{1}{4} P(\left| Y_{n_{k} + 1} + \dots + Y_{m_{k}} \right| \ge n_{k}) P(\left| Y_{m_{k} + 1} + \dots + Y_{n_{k+1}} \right| \ge m_{k})$$
$\lbrace Y_{i} \rbrace_{ i \in \mathbb{N}}$の同一分布性から
$$ =\frac{1}{4} P(\left| Y_{1} + \dots + Y_{m_{k}-n_{k}} \right| \ge n_{k}) P(\left| Y_{1} + \dots + Y_{n_{k+1}-m_{k}} \right| \ge m_{k})$$ 
$\varphi (k)$の定め方から
$$  =\frac{1}{4} P(\left| Y_{1} + \dots + Y_{\varphi(n_{k})} \right| \ge n_{k}) P(\left| Y_{1} + \dots + Y_{\varphi(m_{k})} \right| \ge m_{k}) \ge \frac{1}{4} (1-\alpha)^{2}$$
$\\ \qquad \because$ $\lbrace Z_{\varphi (k)} = j \rbrace_{ \left\{ |j|<k \right\} }$は非交和だから
$$\displaystyle\sum_{|j|<k} P(Z_{\varphi (k)} = j) = P( \bigcup_{|j|<k} Z_{\varphi (k)} = j) = P(\left| Y_{1} + \dots + Y_{\varphi(k)} \right| < k) \le \alpha$$
$\qquad$よって $P(\left| Y_{1} + \dots + Y_{\varphi(k)} \right| \ge k) \ge 1- \alpha$
$\\$ 以上で $ \displaystyle\sum_{k=1}^{\infty} P(C_{k}) \ge  \displaystyle\sum_{k=1}^{\infty} \frac{1}{4} (1-\alpha)^{2} = \infty$となって,$\ P(Z_{n} = 0 \ i.o.) = 1$が示せた.
%$\stackrel{=}{as}$
%$\stackrel{=}{\lbrace Z_{\varphi (k)} = j \rbrace_{|j|<k} は非交和} $ 
%$11 \displaystyle{=}_{\lbrace Z_{\varphi (k)} = j \rbrace_{|j|<k} は非交和}  11$
\end{proof}

%section 独立確率変数に対する大数の法則
\section{独立確率変数に対する大数の法則}

%RANDOM SIGNS PROBLEM
\begin{thm}
$X_{1}, X_{2}, \dots$を独立確率変数とする.$\\$
このとき, \par $\displaystyle \sum_{k=1}^{n}X_{k}$ が確率収束する. $\Leftrightarrow \displaystyle\sum_{k=1}^{n}X_{k}$ が概収束する.
\end{thm}
まず補題を示す.
%補題
\begin{lem}$N \in \mathbb{N}$を固定する. $X_{1}, X_{2}, \dots, X_{N}$を独立確率変数とし,\ $S_{n} = X_{1} + \dots + X_{n}$とおく.
$\\$任意の$\alpha > 0$に対して, $\displaystyle\sup_{1 \le j \le N} P(|S_{N}-S_{j}| > \alpha) = c < 1$となるとき, $\\ \displaystyle P( \sup_{1 \le j \le N} |S_{j}| > 2\alpha ) \le \frac{1}{1-c} P(|S_{N}|>\alpha)$が成立する.
\end{lem}
\begin{proof}
$\\$
$j^{*} (\omega)を|S_{j}(\omega)| > 2 \alpha$となる$1 \le j \le N$で一番小さいものとする.存在しないときは$0$とする.
$\\$ここで$\displaystyle \bigcup_{1 \le j \le N} \left\{ j^{*} = j \right\} = \emptyset$であるとき$\displaystyle P\left( \sup_{1 \le j \le N} |S_{j}| > 2\alpha \right) = 0$なので
$\\ \displaystyle P\left( \sup_{1 \le j \le N} |S_{j}| > 2\alpha \right) = 0 \le \frac{1}{1-c} P(|S_{N}|>\alpha)$が成立する.
$\\$よって $\displaystyle\bigcup_{1 \le j \le N} \lbrace j^{*} = j \rbrace \neq \emptyset$のときを考える. 
$\\ \displaystyle P(|S_{N}| > \alpha, \ \sup_{1 \le j \le N} |S_{j} | > 2\alpha) = \sum_{j=1}^{N} P(|S_{N}| > \alpha, \ j^{*} = j) \ge \sum_{j=1}^{N} P(|S_{N} - S_{j}| \le \alpha, \ j^{*} = j)$ が成立する.
%becuase1
$\\ \displaystyle \because \ \bullet \bigcup_{1 \le j \le N} \lbrace j^{*} = j \rbrace = \lbrace \sup_{1 \le j \le N}|S_{j}| > 2\alpha \rbrace$を示せば,一つ目の等号が成立する.\par
$(\subset)$ \par
$\omega \in$ (左辺)を任意にとる.このとき $j^{*}(\omega) = k$となる$j \in \left\{ 1, \dots , N  \right\} $が存在するから, $|S_{k}(\omega)|>2 \alpha$より
$\\ \displaystyle\sup_{1 \le j \le N}  |S_{j}| \ge |S_{k}(\omega)| > 2\alpha$となって, $\omega \in$ (右辺) \par
$( \supset )$ \par
$\omega \in$ (右辺)を任意にとる.このとき $\displaystyle\sup_{1 \le j \le N } |S_{j}(\omega)| > 2\alpha$であるから, 
$\\ K \in \left\{1,2, \dots , N  \right\}, \, \lbrace k_{1}, k_{2}, \dots, k_{K} \rbrace \subset \lbrace 1,2,\dots ,N \rbrace$ が存在して
$|S_{k_{m}}|>2 \alpha \  (m = 1,2,\dots, K) $となる.
$\displaystyle j^{**}(\omega)= \min{ \lbrace k_{1}, k_{2}, \dots, k_{K} \rbrace }$とすれば$j^{*}(\omega) = j^{**}(\omega)$となるから $\omega \in $(左辺)となる.
$ \\ $
%becuase 2
$\bullet $ 各$k$ $\in \lbrace 1,2,\dots, N \rbrace$ に対して,$\ \lbrace |S_{N}| > \alpha \rbrace \cap \lbrace j^{*} = j \rbrace \supset \lbrace |S_{N}-S_{j}| \le \alpha \rbrace  \cap \lbrace j^{*} = j \rbrace$となるのを示せば2つ目の不等号が示せる.\par
$k \in \lbrace 1,2,\dots, N \rbrace$を任意に固定しておく. $\omega \in $(右辺)をとる. $|S_{N}(\omega) - S_{j}(\omega)| \le \alpha$かつ$j^{*}(\omega) = j$であるから
$$|S_{j}(\omega)| - |S_{N}(\omega)| \le \alpha \, かつ \, |S_{j}(\omega)| > 2\alpha \Leftrightarrow |S_{j}(\omega)| - \alpha \le |S_{N}(\omega)| \, かつ \,
 |S_{j}(\omega)| > 2\alpha \quad\qquad\qquad\qquad\qquad\qquad\qquad\qquad\qquad$$
 $$\Rightarrow 2\alpha - \alpha = \alpha < |S_{N}(\omega)|$$
 \par
以上で$ \lbrace |S_{N}| > \alpha \rbrace \cap \lbrace j^{*} = j \rbrace \supset \lbrace |S_{N}-S_{j}| \le \alpha \rbrace  \cap \lbrace j^{*} = j \rbrace $
$\displaystyle\\ \ \lbrace j^{*} = j \rbrace = ( \bigcap_{k=1}^{j-1}\lbrace |S_{k}| > 2\alpha \rbrace^{c} ) \cap \lbrace |S_{j}| > 2\alpha \rbrace$なので,$\lbrace j^{*} = j \rbrace \in \sigma (X_{1}, \dots , X_{j})$,
$\\ \lbrace |S_{N} - S_{j} | \le \alpha \rbrace \in \sigma (X_{j+1}, \dots , X_{N})$であるから, $ \lbrace j^{*} = j \rbrace $と$ \lbrace |S_{N} - S_{j} | \le \alpha \rbrace $は独立である.
$\\$仮定から$\displaystyle P(|S_{N}-S_{j}| > \alpha) \le c $なので$\displaystyle 1 - P(|S_{N}-S_{j}| > \alpha) = P(|S_{N}-S_{j}| \le \alpha) \ge 1-c$となる.よって
%まとめ
$$ \sum_{j=1}^{N}P(|S_{N}-S_{j}| \le \alpha , \ j^{*}=j) = \sum_{j=1}^{N} P(|S_{N}-S_{j}| \le \alpha)P( j^{*}=j) $$
$$\ge (1-c)\sum_{j=1}^{N} P(j^{*}=j) \\ = (1-c) P(\sup_{1 \le j \le N} |S_{j}| > 2\alpha)$$ 
となる. まとめると
$$(1-c) P(\sup_{1 \le j \le N} |S_{j}| > 2\alpha) \le \sum_{j=1}^{N}P(|S_{N}-S_{j}| \le \alpha , \ j^{*}=j)  \le P(|S_{N}| > \alpha, \ \sup_{1 \le j \le N} |S_{j} | > 2\alpha) \le P(|S_{N}| > \alpha)$$
が成立する.
$\\ \quad \therefore P(\sup_{1 \le j \le N} |S_{j}| > 2\alpha) \le \frac{1}{1-c} P(|S_{N}| > \alpha)$
\end{proof}
補題 11を使って,定理10の証明をする.
%定理9の証明
\begin{proof}
$\\ (\Leftarrow) \ 概収束するならば確率収束するので成立する.$
$\displaystyle \\ (\Rightarrow) \sum_{k=1}^{n}X_{k}$は確率収束するとする. ここで$\displaystyle \sum_{k=1}^{n}X_{k}$が概収束しないと仮定する.(背理法)
$\\$ここで実数列$\displaystyle \lbrace s_{n} \rbrace$が収束しないとすれば$\displaystyle \lbrace s_{n} \rbrace$はCauchy列でないので
$\\$
ある$\varepsilon>0$が存在して, 任意の$N \in \mathbb{N}, \, n,m \ge N$に対して,$|s_{n}-s_{m}| > \varepsilon$が成立する.
$\\$
また, 
ある$\varepsilon>0$が存在して, 任意の$m \in \mathbb{N}$に対して, $\sup_{n>m} |s_{n}-s_{m}| > \varepsilon$となる.
$\\ \displaystyle \sum_{k=1}^{n}X_{k}$はほとんど確実にCauchy列でないから,
$\\$
ある $\displaystyle \varepsilon >0, \, \delta \in (0,1]$ が存在して, 任意の $m \in \mathbb{N}$に対して, 
$\\ \displaystyle P\left(\sup_{n>m} |\sum_{k=1}^{n}X_{k} - \sum_{k=1}^{m}X_{k}| > \varepsilon \right) \ge \delta$が成立する.
$\\$
この$\displaystyle \varepsilon, \ \delta$を固定する.
%Cm,Nを定める.
 $\\ \displaystyle \displaystyle \sum_{k=1}^{n}X_{k}$は確率収束するので$\displaystyle \sum_{k=1}^{N} X_{k} - \sum_{k=1}^{m}X_{k} \stackrel{P}{\rightarrow} 0 $となる. 
%何故ならば
$\displaystyle \\ \because \sum_{k=1}^{n}X_{k} \stackrel{P}{\rightarrow} S$とすると, 
$$ P\left( \left| \sum_{k=1}^{N} X_{k} - \sum_{k=1}^{m}X_{k} \right| > \varepsilon\right) = P\left( \left| \sum_{k=1}^{N} X_{k} - S +S - \sum_{k=1}^{m}X_{k} \right| > \varepsilon\right)$$
$$\le P\left( \left| \sum_{k=1}^{N} X_{k} - s\right| + \left| S - \sum_{k=1}^{m}X_{k} \right| > \varepsilon\right) \le P\left( \left\{ \left| \sum_{k=1}^{N} X_{k} - S\right| > \frac{\varepsilon}{2} \right\} \cup \left\{  \left| S - \sum_{k=1}^{m}X_{k} \right| > \frac{\varepsilon}{2} \right\} \right)$$ 
$$\le P\left( \left| \sum_{k=1}^{N} X_{k} - S\right| > \frac{\varepsilon}{2} \right) + P\left( \left| \sum_{k=1}^{m} X_{k} - S\right| > \frac{\varepsilon}{2} \right) \to 0 \ (m,\ N \to \infty)$$
となる.
$\\$
よって,ある$M \in \mathbb{N}$が存在して, 任意の $m, \ N \ge M \ (m < N)$に対して 
$\\ \displaystyle P\left( \left| \sum_{k=m+1}^{N} X_{k} \right| > \frac{\varepsilon}{2}\right) < 1$ かつ $\displaystyle P\left( \left| \sum_{k=m+1}^{N} X_{k} \right| > \frac{\varepsilon}{2}\right) \to 0 \ (m, \ N \to \infty)$この$m, \ N $を固定する.$\\$
$ \displaystyle C_{m, N} = \sup_{m < n \le N} P\left( \left| \sum_{k=n}^{N}X_{k} \right| > \frac{\varepsilon}{2} \right)$とおくと,$ \displaystyle \  C_{m, N} < 1$ かつ$\displaystyle C_{m, N} \to 0 \ (m, \ N \to \infty)$となる.

%
ここで補題11を使うと,
$\displaystyle \\ P\left(\sup_{ m < n \le N } \left| \sum_{ k = m+1 }^{n} X_{k} \right| > \varepsilon \right) \le \frac{1}{1-C_{m, N}} P\left( \left| \sum_{ k = m+1 }^{N} X_{k} \right| > \frac{\varepsilon}{2} \right)$ とかけて,
まず$\displaystyle N \to \infty$とすると,$\\$
単調性から
$$ \lim_{N \to \infty} P\left(\sup_{ m < n \le N } \left| \sum_{ k = m+1 }^{n} X_{k} \right| > \varepsilon \right) =   P\left( \lim_{N \to \infty} \sup_{ m < n \le N } \left| \sum_{ k = m+1 }^{n} X_{k} \right| > \varepsilon \right)=   P\left( \sup_{ m < n} \left| \sum_{ k = m+1 }^{n} X_{k} \right| > \varepsilon \right)$$
$$ \\ \le \lim_{N \to \infty} \frac{1}{1-C_{m, N}} P\left( \left| \sum_{ k = m+1 }^{N} X_{k} \right| > \frac{\varepsilon}{2} \right)$$であり
$$\ \displaystyle \lim_{m \to \infty} \lim_{N \to \infty} \frac{1}{1-C_{m, N}} P\left( \left| \sum_{ k = m+1 }^{N} X_{k} \right| > \frac{\varepsilon}{2} \right)=0$$だから
$\displaystyle \lim_{m \to \infty}  P\left( \sup_{ m < n} \left| \sum_{ k = m+1 }^{n} X_{k} \right| > \varepsilon \right) = 0$ 
$\\$これは任意の $m \in \mathbb{N}$ に対して 
$\displaystyle P\left(\sup_{n>m} |\sum_{k=1}^{n}X_{k} - \sum_{k=1}^{m}X_{k}| > \varepsilon \right) \ge \delta > 0$となることに矛盾する. 
$\\$背理法により $\displaystyle \sum_{k=1}^{n}X_{k}$は概収束することがわかった.
\end{proof}

%系
\begin{cor}
$\\$
任意の $k \in \mathbb{N}$に対して, $\displaystyle E[X_{k}] = 0$ かつ $\sum_{k=1}^{\infty}E[X_{k}^{2}] < \infty$であるとする.
$\\$このとき$\displaystyle \sum_{k=1}^{n}X_{ k }$は概収束する.
\end{cor}
\begin{proof}
$\displaystyle \\ X_{1}, X_{2}, \dots$は独立なので,$\displaystyle \sum_{k=1}^{n}X_{ k }$が確率収束することを示せば定理10から$\displaystyle \sum_{k=1}^{n}X_{ k }$は概収束する.
$\displaystyle \\ \sum_{k=1}^{\infty}E[X_{k}^{2}] = s^{2}$ (ただし $s \ge 0$ )とする. 任意の $\varepsilon > 0$ に対してChebyshevの不等式から $\displaystyle \\ P\left( \left| \sum_{k=1}^{n}X_{ k } - s \right| > \varepsilon \right) \le \frac{1}{ \varepsilon ^ {2} } E\left[ \left| \sum_{k=1}^{n}X_{ k } - s \right|^{2} \right]$ となる.
$\displaystyle E\left[ \left| \sum_{k=1}^{n}X_{ k } - s \right|^{2} \right] \to 0 \ (n \to \infty )$を示したい. 
$$E\left[ \left| \sum_{k=1}^{n}X_{ k } - s \right|^{2} \right] = E\left[ \left| \sum_{k=1}^{n}X_{ k } \right|^{2} \right] -2sE\left[ \sum_{k=1}^{n}X_{ k } \right] + s^{2} \\ = E\left[ \sum_{k=1}^{n}X_{ k }^{2} + 2 \sum_{i<j}X_{ i }X_{ j }  \right]-2sE\left[ \sum_{k=1}^{n}X_{ k } \right] + s^{2} $$

$ \left(
\begin{tabular}{clll}
ここで, $ X_{1}, X_{2}, \dots$は独立だから \\
$\displaystyle \sum_{i<j} E\left[ X_{ i }X_{ j }  \right] = \sum_{i<j} E\left[ X_{ i } \right] E\left[X_{ j }  \right]$が成立する. また$\displaystyle E[X_{k}] = 0$ なので
\end{tabular}
\right)$
$$= \sum_{k=1}^{n} E\left[ {X_{ k }}^{2} \right] -2s\sum_{k=1}^{n} E\left[ X_{ k } \right] + s^{2} \to s^{2} -2s^{2}+s^{2}=0 \ (n \to \infty)$$ 
$\sum_{k=1}^{n}X_{ k }$ が確率収束することがわかったので $\displaystyle \sum_{k=1}^{n}X_{ k }$は概収束する.
\end{proof}


%独立確率変数に対する大数の法則
\begin{thm}
独立確率変数に対する大数の法則$\\$
$X_{1}, X_{2}, \dots$を独立確率変数とする. 任意の$k \in \mathbb{N}$に対して $E\left[ X_{k} \right] = 0, \ E\left[ X_{k}^{2} \right] < \infty$であるとする.
$\\$正数列 $\lbrace b_{n} \rbrace_{ n \in \mathbb{N} }$が$b_{n} \uparrow \infty$かつ$\displaystyle\sum_{k=1}^{\infty} E\left[ \frac{X_{k}^{2}}{b_{k}^{2}} \right] < \infty$を満たすとき,$\displaystyle\frac{X_{1} + \dots + X_{n}}{b_{n}} \stackrel{a.s.}{\longrightarrow}  0 $が成立する.
\end{thm}
証明の前に一つ補題を示す.
\begin{lem}
Kronecker's Lemma
$\\$ $x_{1}, x_{2}, \dots$を$\displaystyle\sum_{k=1}^{n} x_{k} \to s$ (ただし$s$は有限値)を満たす実数列とする.
$\\$このとき, $b_{n} \uparrow \infty$となる整数列$\lbrace b_{n} \rbrace_{n \in \mathbb{N}}$が取れて, $\displaystyle\frac{1}{b_{n}} \sum_{k=1}^{n} b_{k} x_{k} \to 0 $となる.
\end{lem}
\begin{proof}
$\\$
$\displaystyle r_{n} = \sum_{k=n+1}^{\infty}x_{k}, \ r_{0} = s$とおく.このとき$x_{n} = r_{n-1} - r_{n}, \ n = 1,2, \dots$.また,
$$\sum_{k=1}^{n} b_{k}x_{k} = \\ \sum_{k=1}^{n} b_{k} (r_{k-1}-r_{k}) = \sum_{k=0}^{n-1} b_{k+1} r_{k} - \sum_{k=1}^{n} b_{k} r_{k}  = \sum_{k=1}^{n-1} (b_{k+1}-b_{k})r_{k} + b_{1}s - b_{n}r_{n}$$
となるから
$\\ \displaystyle\left| \sum_{k=1}^{n} b_{k}x_{k} \right| \le \sum_{k=1}^{n-1}(b_{k+1}-b_{k}) |r_{k}| + b_{1}|s| + b_{n}|r_{n}| \ $   ($ \because$ 三角不等式, $b_{n}$は単調増加なので$b_{n+1}-b_{n} \ge 0$)$\\$
ここで$\varepsilon >0$を任意にとる. $\displaystyle\sum_{k=1}^{\infty}x_{k}$は収束するから $r_{k}$の定め方から $N \in \mathbb{N}$を任意の $n \ge N$に対して, 
$\\$ 
$|r_{k}| \le \varepsilon$となるように取れる.この$N$を固定する. $\displaystyle \tilde{r} := \max \lbrace |r_{1}|, \dots, |r_{N-1}|, \varepsilon \rbrace$とする.
$n > N$において 
$\\$
$$ \sum_{k=1}^{n-1}(b_{k+1}-b_{k}) |r_{k}| \le \sum_{k=1}^{N-1}(b_{k+1}-b_{k}) |r_{k}| + \varepsilon \sum_{k=N}^{n-1}(b_{k+1}-b_{k}) \le \tilde{r} (b_{N}-b_{1}) + \varepsilon (b_{n}-b_{N})$$ よって 
$\\$
$$ \left| \frac{ \sum_{k=1}^{n} b_{k}x_{k} }{b_{n}}\right| \le \frac{1}{b_{n}} ( \tilde{r}(b_{N}-b_{1}) + \varepsilon ( b_{n} -b_{N} ) + b_{1}|s| + b_{n}\varepsilon ) \to \varepsilon $$ 
つまり
$\displaystyle \varlimsup_{n \to \infty} \left| \frac{ \sum_{k=1}^{n} b_{k}x_{k} }{b_{n}}\right| \le \varepsilon$
となるから
$\displaystyle \lim_{n \to \infty} \left| \frac{ \sum_{k=1}^{n} b_{k}x_{k} }{b_{n}}\right| \le \varepsilon$
がわかった.
$\\$
ここで$\varepsilon \downarrow 0$とすれば,$\displaystyle\frac{1}{b_{n}} \sum_{k=1}^{n} b_{k} x_{k} \to 0 $が示された.
\end{proof}
この補題14を使って定理13を証明する.
\begin{proof}
$\\$
Kronecker's Lemmaにより, $\displaystyle\sum_{k=1}^{n}\frac{X_{k}}{b_{k}}$がほとんど確実に収束すれば,$\displaystyle \frac{1}{b_{k}} \sum_{k=1}^{n} b_{k} \frac{X_{k}}{b_{k}} = \frac{1}{b_{k}} \sum_{k=1}^{n} X_{k} \stackrel{a.s.}{\longrightarrow}  0$となる.仮定から,
$X_{1}, X_{2}, \dots$は独立確率変数,
$\displaystyle E\left[ \frac{X_{k}}{b_{k}} \right] = 0, \ \sum_{k=1}^{\infty} E\left[ \frac{X_{k}^{2}}{b_{k}^{2}} \right] < \infty$であるから,補題14から $\displaystyle\sum_{k=1}^{n}\frac{X_{k}}{b_{k}}$は概収束する. 
$\displaystyle \ \therefore \frac{X_{1} + \dots + X_{n}}{b_{n}} \stackrel{a.s.}{\longrightarrow}  0 $
\end{proof}

%section 再帰状態と格子状に分布する確率変数
\section{再帰状態と格子状に分布する確率変数}
以下では$X_{1}, X_{2}, \dots$を独立同一分布に従う確率変数列とする. $\displaystyle S_{n} = \sum_{k=1}^{n} X_{k}$と定める.
%定義 
\begin{dfn} 
再帰状態 (recurrent state)
$\\ x \in \mathbb{R}$ とする. $x$が再帰状態(recurrent state)であるとは, $x$の任意の開近傍$I$に対して $P(S_{n} \in I \ i.o.)=1$
となることである.
\end{dfn}

\begin{dfn} 
格子上に分布する確率変数 
$\\ X$が格子$\displaystyle L_{d} = \left\{ nd \ | \ n \in \mathbb{Z} \right\}, \ (d >0)$上に分布するとは, $\displaystyle \sum_{n \in \mathbb{Z}} P(X = nd)=1$かつ $\displaystyle \sum_{n \in \mathbb{Z}} P(X = nl)=1$となる$l >d$が存在しないことである.
$\\ X$が格子上に分布しないとき, $L_{0} = \mathbb{R}$とかいて, $X$は$L_{0}$上に分布するという.
\end{dfn}

%定理3.33
\begin{thm}
$\\ X_{1}, X_{2}, \dots$を$L_{d} \ (d \ge 0)$上に分布する確率変数列とする. このとき$\displaystyle L_{d}$に含まれる状態は全て再帰的または全て非再帰的である.
\end{thm}
\begin{proof}
$\displaystyle \\ G = \left\{ x \in L_{d} | \ xは再帰的\right\}$
とおくと, $G$は閉集合となる.($G$が空のときは全ての状態が非再帰的なので$G \neq \emptyset$とする.)
$\displaystyle \\ \because$ 任意に $\left\{ x_{n} \right\}_{n=1}^{\infty} \subset G$をとって $x_{n} \to x $とする.このとき$x \in G$を示したい.
$\displaystyle \\ x$の開近傍$I$を任意にとる.$I$に対してnを十分大きく取れば$x_{n} \in I$となる.この$n$を固定する.$I$は$x_{n}$の近傍でもあるから,$P(S_{n} \in I \ i.o.) = 1 \ \therefore x \in G$
%候補状態
$\\$
$\displaystyle y \in \mathbb{R}$が候補状態であるとは$y$の任意の近傍$I$に対して, ある自然数$k \in \mathbb{N}$が存在して
$\\$
$P(S_{k} \in I) >0$が成立することをいう.
$\\$
「$x$が再帰的かつ$y$が候補状態$\Rightarrow \ x-y$は再帰的」が成立することを示したい. 
$\\ \because \\$
$\displaystyle \varepsilon >0$を任意にとる. このとき$y$は候補状態だから, ある $k \in \mathbb{N}$が存在して $P(\left| S_{k} - y \right| < \varepsilon) >0$となる.
$\\ x$は再帰的より$\displaystyle P(\bigcap_{m \in \mathbb{N} } \bigcup_{n > m} \left\{ \left| S_{n} - x \right| < \varepsilon \right\}) = 1$となるから
$$0=P(\bigcup_{m \in \mathbb{N} } \bigcap_{n > m} \left\{ \left| S_{n} - x \right| \ge \varepsilon \right\})$$ 
$$\ge P( \left| S_{k} - y \right| < \varepsilon, \ \bigcup_{m \in \mathbb{N} } \bigcap_{n > m} \left\{ \left| S_{k+n} - S_{k} - (x-y) \right| \ge 2\varepsilon \right\})$$
$$= P(\left| S_{k} - y \right| < \varepsilon)P(\bigcup_{m \in \mathbb{N} } \bigcap_{n > m} \left\{ \left| S_{k+n} - S_{k} - (x-y) \right| \ge 2\varepsilon \right\})= P(\left| S_{k} - y \right| < \varepsilon)P(\bigcup_{m \in \mathbb{N} } \bigcap_{n > m} \left\{ \left| S_{n} - (x-y) \right| \ge 2\varepsilon \right\})$$
\par
$\quad \because \\ \quad$  任意の$\omega \in \left\{ \left| S_{k} - y \right|  < \varepsilon \right\} \cap \bigcup_{m \in \mathbb{N} } \bigcap_{n > m} \left\{ \left| S_{k+n} - S_{k} - (x-y) \right| \ge 2\varepsilon \right\}$をとる. 
$\\$
ある $m \in \mathbb{N}$ が存在して, 任意の$n \ge m$ に対して, $\left| S_{k+n}(\omega) - S_{k}(\omega) -(x-y) \right| \ge 2 \varepsilon$となる.
$\\$この$m$ を固定する. 任意の$n \ge m$に対して
$\\$
$ 2\varepsilon \le \left| S_{k+n}(\omega) -  x \right| + \left| S_{k}(\omega) -y \right| <  \left| S_{k+n}(\omega) -  x \right|  + \varepsilon$
から
$\displaystyle \varepsilon \le \left| S_{k+n}(\omega) -  x \right|$となる. 
$\\$
ここで
$N=k+m$とおけば, 
$\displaystyle n \ge N$に対して, $ \left| S_{n}(\omega) -  x \right| \ge \varepsilon$ なので $\displaystyle \omega \in \bigcup_{m \in \mathbb{N} } \bigcap_{n > m} \left\{ \left| S_{n} - x \right| \ge \varepsilon \right\} \\ \therefore \bigcup_{m \in \mathbb{N} } \bigcap_{n > m} \left\{ \left| S_{n} - x \right| \ge \varepsilon \right\} \supset \left\{ \left| S_{k} - y \right|  < \varepsilon \right\} \cap \bigcup_{m \in \mathbb{N} } \bigcap_{n > m} \left\{ \left| S_{k+n} - S_{k} - (x-y) \right| \ge 2\varepsilon \right\}$ 
$\\$また $X_{1}, X_{2}, \dots$は独立なので,$\ S_{k}$と$\displaystyle S_{k+n} - S_{k} = \sum_{m=k+1}^{k+n} X_{m}$は独立である.
$\\$また同一分布性から
$$P(\bigcup_{m \in \mathbb{N} } \bigcap_{n > m} \left\{ \left| S_{k+n} - S_{k} - (x-y) \right| \ge 2\varepsilon \right\})= P(\bigcup_{m \in \mathbb{N} } \bigcap_{n > m} \left\{ \left| S_{n} - (x-y) \right| \ge 2\varepsilon \right\})$$も成立する.
$\displaystyle \\ P(\left| S_{k} - y \right| < \varepsilon) >0$なので$P(\bigcup_{m \in \mathbb{N} } \bigcap_{n > m} \left\{ \left| S_{n} - (x-y) \right| \ge 2\varepsilon \right\}) = 0$つまり $\displaystyle \\ P(\left\{ \left| S_{n} - (x-y) \right| < 2\varepsilon \right\} \ i.o.) = 1 \quad I_{\varepsilon} = (x-y -2\varepsilon, x-y + 2\varepsilon)$とおけば,$P( \left| S_{n} \right| \in I_{\varepsilon}  \ i.o.) = 1$
$\\ \varepsilon >0$は任意だったから$x-y$は再帰的である.
$\\$
$\\$
$x \in G$は候補状態である
%x in G$は候補状態.
$\\$
$\because$ $x \in G$を任意にとる.\ $x$の開近傍$I$を任意にとる. $\displaystyle P(S_{n} \in I \ i.o.) = 1$である. 任意の$k \in \mathbb{N}$に対して$P(S_{k} \in I )= 0$であるとすると$\displaystyle \sum_{k=1}^{\infty} P(S_{k} \in I) < \infty$よりBorel-Cantelli Lammaから$P(S_{k} \in I \ i.o.)= 0$
$\\$これは$\displaystyle P(S_{n} \in I \ i.o.) = 1$に矛盾する.よって$P(S_{k} \in I ) > 0$となる$m \in \mathbb{N}$が存在する.
$\\$よって\ $x-x =0 \in G$である.
このことから$G$は群である. $G$が$\mathbb{R}$上で閉なので$G$は$\mathbb{R}$上の閉部分群である. 
%ここで$\mathbb{R}$上の閉部分群はある格子$L_{l} \ (l \ge 0)$である.
%$X_{1}, X_{2},\dots$は$L_{d}$上に分布するので$S_{n} \in L_{d}$から任意の候補状態$y$は$y \in L_{d}$となる.よって
全ての候補状態$y$に対して $0-y = y \in G$となる.
%$L_{d} \subset G$となる.
%$S_{n} \in L_{d}$から $G \subset L_{d}$である.
%d>0
$\displaystyle \\ \bullet d>0$のとき
$\\ P(X_{1} = nd) >0$ かつ$P(X_{1}=(n+1)d)>0$となる$n \in \mathbb{Z}$が存在しないと仮定する.(背理法)
$\\$
このとき $0 \in G$は候補状態なので
$\displaystyle \sum_{n \in \mathbb{Z}} P(X_{1} = (2d) n)=1$となって, $d$の最大性に反する. 
よってある$n \in \mathbb{Z}$が取れて$nd, (n+1)d \in G$となる. $G$は群なので$(n+1)d - nd = d \in G$このことから$L_{d} \subset G$である.
%d=0
$\displaystyle \\ \bullet d=0$のとき
$\\$
このとき$G$に対して,ある$ l > 0$が存在して, $G=\left\{ nl | n \in \mathbb{Z}  \right\}$となると仮定する(背理法).
$\\$候補状態は$G$の元なので$\displaystyle \sum_{n \in \mathbb{Z}}P(X_{1}=nl)=1$となり,これは$d=0$に矛盾する.よって$G=\mathbb{R} = L_{0}$
$\\$以上で$d \ge 0$に対して$L_{d} = G$となり, $L_{d}$の全ての状態は再帰的となる.
\end{proof}

%定理3.34
%ここまで
\begin{thm}
$\displaystyle X_{1}, X_{2}, \dots$を$L_{d}$上に分布する確率変数列とする(ただし$d \ge 0$). 
$\\ (i)$もし, 有界区間$J \subset \mathbb{R}$が存在して $J \cap L_{d} \neq \emptyset$かつ $\displaystyle \sum_{n=1}^{\infty}P(S_{n} \in J) < \infty$を満たせば,再帰状態は存在しない.
$\\ (ii)$もし, 有界区間$\displaystyle J \subset \mathbb{R}$で任意の$\displaystyle \varepsilon \in (0, \frac{\left|\left|J\right|\right|}{2})$に対して, ある$x \in \mathbb{R}$が存在して $I = (x-\varepsilon, x+\varepsilon) \subset J$かつ$\displaystyle \sum_{n=1}^{\infty} P(S_{n} \in I) = \infty$となれば, $L_{d}$の全ての状態は再帰状態である.
\end{thm}
\begin{proof}
$\\$ $(i) \ J \cap L_{d} \neq \emptyset$かつ $\displaystyle \sum_{n=1}^{\infty}P(S_{n} \in J) < \infty$を満たす有界区間$J \subset \mathbb{R}$が存在することを仮定する.
$\\$
Borel-Canteli Lemmaから$P(S_{n} \in J \ i.o.)=0$となって, $L_{d}$は少なくとも再帰的でない状態が含まれる.
$\displaystyle \\ \because x \in L_{d} \cap J$をとれば, $x$の開近傍$I \subset J$がとれて, $P(S_{n} \in I \ i.o.) \le P(S_{n} \in J \ i.o.) = 0$となって$x$は再帰的でない$L_{d}$の元である.
$\displaystyle \\ $定理17から$L_{d}$の元は全て再帰状態にはならない.つまり再帰状態は存在しない.
%(ii)の証明
$\displaystyle \\ \\ (ii)$長さ$l$の有界区間$\displaystyle J \subset \mathbb{R}$で任意の$\displaystyle \varepsilon \in (0, \frac{l}{2})$に対して, ある$x \in \mathbb{R}$が存在して, $I = (x-\varepsilon, x+\varepsilon) \subset J$かつ$\displaystyle \sum_{n=1}^{\infty} P(S_{n} \in I) = \infty$となるものがとれたとする. $0 \in L_{d}$なので, $0$が再帰的であることがわかれば$\\$定理17から$L_{d}$の全ての状態が再帰的である.
%
$$A_{k} = \begin{cases} \left\{ S_{k} \in I, \ S_{n+k} \notin I \ n=1,2,\dots \right\} & (k \geq 1) \\ \left\{ S_{n} \notin I \ n=1,2,\dots \right\} & (k = 0) \end{cases}$$と定めると,
$\displaystyle \bigcup_{m \in \mathbb{N}} \bigcap_{n>m} \left\{ S_{n} \notin I \right\} = \bigcup_{k=0}^{\infty} A_{k}$となる.
%
$\\ \displaystyle \because (\subset)$  $\omega \in$ (左辺) とする. このときある $m \in \mathbb{N}$がとれて, 任意の$n > m$に対して, $S_{n}(\omega) \notin I$となる. $1 \le i \le m-1$の中で$S_{i}(\omega) \in I$となるものが存在するときその最大値を$k$とすれば, 
$\\ S_{k}(\omega) \in I, \ S_{n}(\omega) \notin I \ (n \ge k)$が成立する.
よって, $\omega \in A_{k}$となる. 
$\\$
任意の$i \in \left\{ 1, \dots , m-1 \right\}$に対して, $S_{i} \notin I $のときは, $\omega \in \left\{ S_{n} \notin I \ n=1,2,\dots \right\} = A_{0}$ 以上で
$\displaystyle \omega \in$ (右辺)
%
$\\ (\supset)$ $\omega \in$ (右辺)とする. このとき$ k \in \mathbb{N}$が存在して $\omega \in A_{k}$となる.
$\\ k \ge 1$のとき $\displaystyle \omega \in A_{k} \subset \left\{ S_{n+k} \notin n=1,2,\dots \right\} = \bigcap_{n=k+1}^{\infty} \left\{ S_{n} \notin I \right\}$
$\\ k=0$のとき $\displaystyle \omega \in A_{0} = \bigcap_{n=1}^{\infty} \left\{ S_{n}\notin I \right\}$ $\qquad \therefore \omega \in$ (左辺)
$\\$

定め方から$A_{0}, A_{1}, \dots$は非交和なので
$$\displaystyle P(\bigcup_{m \in \mathbb{N}} \bigcap_{n>m} \left\{ S_{n} \notin I \right\} ) = P(\bigcup_{k=0}^{\infty} A_{k}) = \sum_{k=0}^{\infty} P(A_{k})$$
$\\ k \ge 1$のとき
$$P(A_{k}) \ge P(S_{k} \in I, \ \left| S_{n+k} - S_{k} \right| \ge 2\varepsilon, \ n=1,2,\dots)$$
$\\$
$\displaystyle \because \omega \in \left\{ S_{k} \in I \right\} \cap \left\{ \left| S_{n+k} - S_{k} \right| \ge 2\varepsilon, \ n=1,2,\dots \right\}$を任意にとる.
$\displaystyle \\ S_{k}(\omega)\in I$かつ$ ( S_{n+k}(\omega) - S_{k}(\omega) \le -2\varepsilon$または$2\varepsilon \le S_{n+k}(\omega) - S_{k}(\omega), \ n=1,2,\dots )$
$\\ \qquad$ (このとき$I$の定め方から, $x-\varepsilon < S_{k} < x+\varepsilon $だから)
$\\$
$\displaystyle \Rightarrow S_{k}(\omega)\in I$かつ$ ( S_{n+k}(\omega) \le (x+\varepsilon) -2\varepsilon$または$2\varepsilon +(x-\varepsilon) \le S_{n+k}(\omega), \ n=1,2,\dots )$
$\\  $
$\displaystyle \Leftrightarrow S_{k}(\omega)\in I$かつ$ ( S_{n+k}(\omega) \le x -\varepsilon$または$x+\varepsilon \le S_{n+k}(\omega), \ n=1,2,\dots )$
$\displaystyle \\ \Leftrightarrow $
$\displaystyle S_{k}(\omega)\in I$かつ$ ( S_{n+k}(\omega) \notin I, \ n=1,2,\dots )$
$\\ \displaystyle \Leftrightarrow  \omega \in A_{k}$ 

$\\$独立性と同一分布性から
$$ P(S_{k}\in I, \ \left| S_{n+k} -S_{k} \right| \ge 2\varepsilon, \ n=1,2,\dots) = P(S_{k}\in I) P( \left| S_{n+k} -S_{k} \right| \ge 2\varepsilon, \ n=1,2,\dots)$$
$$ = P(S_{k}\in I) P( \left| S_{n} \right| \ge 2\varepsilon, \ n=1,2,\dots)$$
が成立する.よって
$$ P(  \bigcup_{m \in \mathbb{N}} \bigcap_{n>m} \left\{ S_{n} \notin I \right\} ) = \sum_{k=0}^{\infty} P(A_{k}) \ge P(A_{0}) + \sum_{k=1}^{\infty} P(A_{k})$$
$$ \ge P(A_{0}) + P( \left| S_{n} \right| \ge 2\varepsilon, \ n=1,2,\dots) \sum_{k=1}^{\infty} P(S_{k}\in I) \quad $$
ここで, $\displaystyle \sum_{k=1}^{\infty} P(S_{k}\in I) = \infty$
であるから
$P( \left| S_{n} \right| \ge 2\varepsilon, \ n=1,2,\dots)  = 0$が示せた.
$\\$
以上で任意の$ \varepsilon \in ( 0, \frac{l}{2})$に対して $P( \left| S_{n} \right| \ge 2\varepsilon, \ n=1,2,\dots)  = 0 \ $となる $(*)$
$\\ \displaystyle \varepsilon \in (0, \frac{l}{2})$を新しく固定する. $I = (-\varepsilon, \ \varepsilon)$として, $\left\{ A_{k} \right\}_{k=0}^{\infty}$を先と同様にとる. $I_{\delta} = (-\delta, \delta)$ (ただし, $\delta < \varepsilon$)とする. 任意の$ k \in \mathbb{N}$に対して $\displaystyle A_{k} = \lim_{\delta \uparrow \varepsilon} \left\{ S_{k} \in I_{\delta}, \ S_{n+k} \notin I \ n=1,2,\dots \right\}$となるから
$$ P(A_{k}) = P(\lim_{\delta \uparrow \varepsilon} \left\{ S_{k} \in I_{\delta}, \ S_{n+k} \notin I \ n=1,2,\dots \right\})$$ 連続性から
$$ = \lim_{\delta \uparrow \varepsilon} P(\left\{ S_{k} \in I_{\delta}, \ S_{n+k} \notin I \ n=1,2,\dots \right\})$$
$\displaystyle P(S_{k} \in I_{\delta}, \ S_{n+k} \notin I \ n=1,2,\dots) \le P(S_{k}\in I_{\delta}, \ \left| S_{n+k} - S_{k} \right| \ge \varepsilon - \delta \ n = 1,2,\dots )$となる.
%because 
$\\ \displaystyle \because \ \omega \in \left\{ S_{k} \in I_{\delta}, \ S_{n+k} \notin I \ n=1,2,\dots \right\}$を任意にとる. $\\ \displaystyle -\delta < S_{k}(\omega) < \delta, \ S_{n+k}(\omega) \le -\varepsilon または\ \varepsilon \le S_{n+k}(\omega) \ (n =1,2,\dots)$となる.
$\\ S_{n+k}(\omega) - S_{k}(\omega) \le -\varepsilon - S_{k}(\omega) < -\varepsilon + \delta$または $\varepsilon - \delta < \varepsilon - S_{k}(\omega) \le S_{n+k}(\omega) - S_{k}(\omega)$
$\\$よって $\displaystyle \left| S_{n+k}(\omega) - S_{k}(\omega) \right| \ge \varepsilon - \delta \quad \therefore \ \omega \in \left\{ S_{k} \in I_{\delta}, \ \left| S_{n+k} - S_{k} \right| \ge \varepsilon - \delta \ n = 1,2,\dots \right\}  $
%
$\\$
独立性と同一分布性から
$$ P(S_{k}\in I_{\delta}, \ \left| S_{n+k} - S_{k} \right| \ge \varepsilon - \delta \ n = 1,2,\dots ) = P(S_{k}\in I_{\delta}) P(\left| S_{n+k} - S_{k} \right| \ge \varepsilon - \delta \ n = 1,2,\dots)$$
$$ = P(S_{k}\in I_{\delta}) P(\left| S_{n} \right| \ge \varepsilon - \delta \ n = 1,2,\dots) = 0$$
$\because \displaystyle  0 < \frac{\varepsilon - \delta}{2} < \frac{l}{2}$なので$(*)$より $P(\left| S_{n} \right| \ge \varepsilon - \delta \ n = 1,2,\dots) =0 \\$
以上で$P(A_{k})=0 \ ( k \ge 1)$となる. 
$\\ 0< \varepsilon<\frac{l}{2}$だから$0<\frac{\varepsilon}{2} < \frac{l}{2}$なので $P(A_{0}) = P(S_{n} \notin I \ n=1,2,\dots) = P(\left| S_{n} \right| \ge \varepsilon \ n=1,2,\dots) = 0$
以上から任意の$\varepsilon \in  (0 , \frac{l}{2}), \ I= (-\varepsilon, \varepsilon)$に対して, $P (S_{n} \in I \ i.o.)=1 $ 
よって, $0$は再帰的となる. $\therefore L_{d}$の全ての状態は再帰的である.
\end{proof}

\begin{cor}
$\\ \displaystyle \left\{ S_{n} \right\}_{n \in \mathbb{N}}$に対して, 次の$(i), \ (ii)$ のいずれかが成立する.
$\\  (i) \ L_{d} \cap I \neq \emptyset$ を満たす全ての有界区間$I$に対して, $P(S_{n} \in I \ i.o.)=1$ 
$\\ (ii) \ L_{d} \cap I \neq \emptyset$を満たす全ての有界区間$I$ に対して, $P(S_{n} \in I  \ i.o.)=0$
\end{cor}
\begin{proof}
$\\ L_{d} \cap I \neq \emptyset$となる全ての有界区間$I$に対して$\displaystyle \sum_{n=1}^{\infty} P(S_{n} \in I)= \infty$となるとき, 定理17から$L_{d}$の全ての状態は再帰的である.つまり $L_{d} \cap I \neq \emptyset$となる任意の有界区間$I$とすれば, $ x \in L_{d} \cap I$を任意にとって 
$ P(S_{n} \in I \ i.o.) \ge P(S_{n} = x \ i.o.) =1 \quad (\because \ x$は再帰的)となる.
$\\ L_{d} \cap I \neq \emptyset$となる有界区間$I$が存在して$\displaystyle \sum_{n=1}^{\infty} P(S_{n} \in I) < \infty$となるとき, 定理17から再帰状態は存在しない. よって任意の有界区間$I \ ( L_{d} \cap I \neq \emptyset )$ に対して $I$を含む開区間$J$とすれば $P(S_{n}\in I \ i.o.) \le P(S_{n} \in J \ i.o. ) =0$ ( $\because \ J$は $L_{d}$のある元の開近傍)
\end{proof}

\begin{dfn}
$\\ \displaystyle \left\{ S_{n} \right\}_{n \in \mathbb{N}}$が系$18 \ (i)$を満たすとき, 再帰的という.また, 系$18 \ (ii)$を満たすとき, 非再帰的という.
\end{dfn}

%定理
\begin{thm}
$\displaystyle \\ E\left[ X_{1} \right] = 0$となるとき,$\displaystyle \ S_{1}, S_{2}, \dots$は再帰的となる.
\end{thm}

証明の前にまず命題を一つ示す.
%命題
\begin{prop}
$\displaystyle \\ I$が長さ$a$の区間であるとき, 
$\displaystyle \sum_{n=1}^{\infty} P \left(S_{n} \in I \right) \le 1 + \sum_{n=1}^{\infty} P \left( \left| S_{n} \right| \le a  \right)$となる.
\end{prop}
\begin{proof}
$\displaystyle \\ I$を長さ$a$の任意の区間としてとる.
$\displaystyle N := \sum_{n=1}^{\infty} 1_{ \left\{ S_{n} \in I \right\} }$とおくと, $N$は和$S_{1}, S_{2}, \dots$の中で$I$に含まれるものの個数を数えたものとなる.
$\displaystyle \\ n^{*} (\omega) := \begin{cases} S_{n}(\omega) \in I となる最小の n \in \mathbb{N} \\ \infty & (もしS_{n}(\omega) \notin I \ , n = 1,2,\dots ) \end{cases}$とおく. $\displaystyle \\ \left\{ n^{*} = k \right\}$ 上では, $n < k$となる $n\in \mathbb{N}$に対しては $1_{ \left\{ S_{n} \in I \right\} } = 0$,また $1_{ \left\{ S_{k} \in I \right\} } = 1$ となるから
$$ N = \sum_{n=1}^{k-1} 1_{ \left\{ S_{n} \in I \right\} } + 1_{ \left\{ S_{k} \in I \right\} } + \sum_{n=k+1}^{\infty} 1_{ \left\{ S_{n} \in I \right\} } = 1 + \sum_{n=1}^{\infty} 1_{ \left\{ S_{k+n} \in I \right\} } = 1 + \sum_{n=1}^{\infty} 1_{ \left\{ S_{k+n} - S_{k} \in I - S_{k} \right\} } \\ \le 1 + \sum_{n=1}^{\infty} 1_{ \left\{ S_{k+n} - S_{k} \in \left[-a,a\right] \right\} }$$
$\displaystyle \\ \quad \because$ 任意に $\omega \in \left\{ n^{*} = k \right\}$をとる.このとき $I - S_{k}(\omega) \subset [-a, a] $となる. つまり $1_{ \left\{ I - S_{k}(\omega)  \right\} } \le 1_{ \left[ -a, a \right]}$が成立する.
$\\$なぜなら, $\overline{I} = \left[ I_{1}, I_{1}+a \right]$とかくと$(I_{1} \in \mathbb{R})$ 
$\\ -I_{1} - a \le -S_{k}(\omega) \le -I_{1}$ であって, $I - I_{1} - a = \left[-a, 0\right], \ I - I_{1}= \left[0,a\right]$であるから,
$\\I - S_{k}(\omega) \subset \left[-a, a\right]$だから, $1_{ \left\{ I - S_{k}  \right\} } \le 1_{ \left[ -a, a \right]}$となる.

$\\ \left\{ n^{*} = k \right\} \in \sigma (X_{1}, X_{2}, \dots, X_{k}) $なので, $\left\{ n^{*} = k \right\}$と$\left\{ S_{k+n} - S_{k} \in [-a,a] \right\}$は独立となる.
$$\int_{ \left\{ n^{*} = k \right\} }{N(\omega) P (d\omega)} \le \int_{ \left\{ n^{*} = k \right\} } { \left( 1 + \sum_{n=1}^{\infty} 1_{ \left\{ (S_{k+n} - S_{k})(\omega) \in \left[-a,a\right] \right\}  } \right) P (d\omega) } $$ 単調収束定理から, 
$$ = P( n^{*} = k ) + \sum_{n=1}^{\infty} \int_{ \left\{ n^{*} = k \right\} } {1_{ \left\{ S_{k+n}(\omega) - S_{k}(\omega) \in \left[-a,a\right] \right\}  } P (d\omega) } = P( n^{*} = k ) + \sum_{n=1}^{\infty} P\left( \left\{ n^{*}=k \right\} \cap  \left\{ S_{k+n} - S_{k} \in \left[-a,a\right] \right\}  \right) $$ 
独立性から
$$=  P( n^{*} = k ) +  \sum_{n=1}^{\infty} P(n^{*} = k) P (S_{k+n} - S_{k} \in \left[-a,a\right])$$ 
同一分布性から 
$$ = P( n^{*} = k ) + P(n^{*} = k) \sum_{n=1}^{\infty} P ( \left| S_{n} \right| \le a)$$ ここで両辺$k$で和をとると, 左辺は単調収束定理を使って,
$$ \sum_{k=1}^{\infty} \int_{ \left\{ n^{*} = k \right\} }{N(\omega) P (d\omega)} = \int_{ \Omega }{\sum_{n=1}^{\infty} 1_{ \left\{ S_{n}(\omega) \in I \right\} } P (d\omega)} = \sum_{n=1}^{\infty} P(S_{n} \in I )$$
となって, $\displaystyle \sum_{n=1}^{\infty} P(S_{n} \in I ) \le 1 + \sum_{n=1}^{\infty} P ( \left| S_{n} \right| \le a)$が成立する.
\end{proof}
 
 命題22を使って, 定理21を証明する.
 \begin{proof}
 $\displaystyle \\ M \in \mathbb{N}$を任意にとる. 
 $$\sum_{n=1}^{\infty} P (\left| S_{n} \right| < M) =  \sum_{k=-M}^{M-1} \sum_{n=1}^{\infty} P (S_{n} \in (k, k+1))$$
 命題22より
 $$ \le \sum_{k=-M}^{M-1}(1 + \sum_{n=1}^{\infty} P (\left| S_{n} \right| \le 1) ) =  2M (1 + \sum_{n=1}^{\infty} P( \left| S_{n} \right| \le 1))$$
 つまり $\displaystyle \varlimsup_{M \to \infty} \frac{1}{2M} \sum_{n=1}^{\infty} P( \left| S_{n} \right| < M) \le 1 + \sum_{n=1}^{\infty}P( \left| S_{n} \right| \le 1) - (*)$が成立する.
 $\\$大数の強法則から $\frac{S_{n}}{n} \stackrel{P}{\rightarrow} 0 \ (\because 概収束 \Rightarrow 確率収束)$となる. 
 $ \varepsilon >0$を任意にとる.このとき
 $\\$
 $\displaystyle  m \in \mathbb{N}$が存在して, $P(\left|S_{n} \right| < \varepsilon n) > \frac{1}{2} \ ( n > m)$となる. 
 $M \in \mathbb{N}$を $\displaystyle \varepsilon (m+1) < M$となるようにとれば,
 $\displaystyle \\ P(\left|S_{n} \right| < M ) \ge P(\left|S_{n} \right| < \varepsilon (m+1) ) > \frac{1}{2} $ となる.
 $\displaystyle \\ \therefore \sum_{n=1}^{\infty} P(\left|S_{n} \right| < M) \ge \sum_{n=m+1}^{\infty} P(\left|S_{n} \right| < M) > \frac{1}{2} (\frac{M}{\varepsilon}-m) \ $から $\displaystyle \frac{1}{2M} \sum_{n=1}^{\infty} P(|S_{n}| < M) \ge \frac{1}{4 \varepsilon} - \frac{m}{4M}$
 $\displaystyle \\ (*)$から $\displaystyle 1+\sum_{n=1}^{\infty} P ( |S_{n}| \le 1 ) \ge \varlimsup_{M \to \infty} \frac{1}{2M} \sum_{n=1}^{\infty} P( |S_{n}| < M ) \ge \frac{1}{4 \varepsilon}$ となり, $\varepsilon \downarrow 0$とすれば, 
 $\\ \displaystyle \sum_{n=1}^{\infty}P(|S_{n}| \le 1) = \infty$ となる. 
 ここで任意の$\displaystyle \delta \in  (0, \frac{1}{2})$ を固定する. 
 $\displaystyle \frac{1}{N} < \delta$となる $N \in \mathbb{N}$をとると, 
 $$\sum_{n=1}^{\infty} P (\left| S_{n} \right| < M) =  \sum_{k= -MN}^{N(M-1)} \sum_{n=1}^{\infty} P(S_{n} \in (\frac{k}{N}, \frac{k+1}{N})) \le 2MN (1+ \sum_{n=1}^{\infty}P(\left|S_{n} \right| \le \frac{1}{N}))$$
 となって
 $$ 1 + \sum_{n=1}^{\infty} P (|S_{n}| < \frac{1}{\delta}) \ge  1 + \sum_{n=1}^{\infty} P (|S_{n}| \le \frac{1}{N}) $$
 $$\ge \varlimsup_{M \to \infty} \frac{1}{MN} \sum_{n=1}^{\infty} P(\left|S_{n} \right| < M) \ge \frac{1}{4 \varepsilon N} \rightarrow \infty \ (\varepsilon \downarrow 0)$$
 だから, $ \displaystyle \sum_{n=1}^{\infty} P (|S_{n}| < \frac{1}{\delta}) = \infty$となる.このことから, 定理18$(ii)$より$L_{d}$の全ての元は再帰的である.よって系19から$\displaystyle { \left\{ S_{n} \right\} }_{n}^{\infty}$は再帰的となる.
 \end{proof}
 
 例
 $\\$コイントスを考える. $\\$
 $\displaystyle P(Head)=\frac{1}{2}, \
 Y_{i}(\omega)= \left \{
\begin{array}{ll}
1 & (\omega_{i}がHのとき) \\
-1 & (\omega_{i}がTのとき)
\end{array}
\right., Z_{n} = Y_{1} + \dots + Y_{n}$とする.
$\\$このとき$\displaystyle Y_{1}, Y_{2}, \dots$は独立同一分布に従い,$L_{1}$上に分布する. $\displaystyle E\left[Y_{1} \right] = 0$なので,定理21により $\left\{ Z_{n} \right\}_{n \in \mathbb{N}}$は再帰的になる. また, 定理9が従うので,  $\displaystyle P(Z_{n} = 0 \ i.o.) = 1$より,系19から$\left\{ Z_{n} \right\}_{n \in \mathbb{N}}$は再帰的になることがわかる.
 
 \end{document}